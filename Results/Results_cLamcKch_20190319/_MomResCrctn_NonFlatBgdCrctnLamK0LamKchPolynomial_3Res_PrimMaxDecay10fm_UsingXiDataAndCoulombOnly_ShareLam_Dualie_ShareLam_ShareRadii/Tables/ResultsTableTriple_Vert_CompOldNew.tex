%%% NOTE: If I want to run this standalone, uncomment out lines below
%%%       But lines must be commented out to run within larger project
\begin{comment}
\makeatletter
\def\input@path{{/home/jesse/Analysis/FemtoAnalysis/AnalysisNotes/}}
\makeatother

\documentclass[ALICE,manyauthors]{ALICE_analysis_notes}

\usepackage{MyStyle}
\usepackage{chngpage}  % for adjustwidth
\usepackage{boldline}  % to make lines in table bold
                       % V{<factor>} vertical rule in \begin{tabular} command
                       % also \clinB{<spec>}{<factor>} and \hlineB{<factor>}
\usepackage{arrayjobx} % To use the array structures stored in FitResults_cLamcKch_20180505.tex   
\end{comment}  


%%%% UPDATE ME FOR EACH FILE!!!!!!!!!!!!!!!!!!!!!!!!!!!!!!!!!!!!!!!!!!!!!!!!!!!!!!!!!!!
\newarray\ArrLamKchP
\readarray{ArrLamKchP}{
                       1.14 & 0.29 & 0.18 & 
                       6.02 & 0.82 & 0.65 & 
                       0.82 & 0.18 & 0.16 & 
                       4.50 & 0.51 & 0.45 & 
                       0.90 & 0.22 & 0.19 & 
                       3.61 & 0.44 & 0.30 & 
                       $-$0.60 & 0.12 & 0.11 & 
                       0.51 & 0.15 & 0.12 & 
                       0.83 & 0.47 & 1.23}

\newarray\ArrALamKchM
\readarray{ArrALamKchM}{
                       1.14 & 0.29 & 0.18 & 
                       6.02 & 0.82 & 0.65 & 
                       0.82 & 0.18 & 0.16 & 
                       4.50 & 0.51 & 0.45 & 
                       0.90 & 0.22 & 0.19 & 
                       3.61 & 0.44 & 0.30 & 
                       $-$0.60 & 0.12 & 0.11 & 
                       0.51 & 0.15 & 0.12 & 
                       0.83 & 0.47 & 1.23}

\newarray\ArrLamKchM
\readarray{ArrLamKchM}{
                       1.14 & 0.29 & 0.18 & 
                       6.02 & 0.82 & 0.65 & 
                       0.82 & 0.18 & 0.16 & 
                       4.50 & 0.51 & 0.45 & 
                       0.90 & 0.22 & 0.19 & 
                       3.61 & 0.44 & 0.30 & 
                       0.27 & 0.12 & 0.07 & 
                       0.40 & 0.11 & 0.07 & 
                       $-$5.23 & 2.13 & 4.80}

\newarray\ArrALamKchP
\readarray{ArrALamKchP}{
                       1.14 & 0.29 & 0.18 & 
                       6.02 & 0.82 & 0.65 & 
                       0.82 & 0.18 & 0.16 & 
                       4.50 & 0.51 & 0.45 & 
                       0.90 & 0.22 & 0.19 & 
                       3.61 & 0.44 & 0.30 & 
                       0.27 & 0.12 & 0.07 & 
                       0.40 & 0.11 & 0.07 & 
                       $-$5.23 & 2.13 & 4.80}

\newarray\ArrLamKs
\readarray{ArrLamKs}{
                       1.14 & 0.29 & 0.18 & 
                       6.02 & 0.82 & 0.65 & 
                       0.82 & 0.18 & 0.16 & 
                       4.50 & 0.51 & 0.45 & 
                       0.90 & 0.22 & 0.19 & 
                       3.61 & 0.44 & 0.30 & 
                       0.10 & 0.13 & 0.07 & 
                       0.58 & 0.15 & 0.13 & 
                       $-$1.85 & 1.71 & 2.77}

\newarray\ArrALamKs
\readarray{ArrALamKs}{
                       1.14 & 0.29 & 0.18 & 
                       6.02 & 0.82 & 0.65 & 
                       0.82 & 0.18 & 0.16 & 
                       4.50 & 0.51 & 0.45 & 
                       0.90 & 0.22 & 0.19 & 
                       3.61 & 0.44 & 0.30 & 
                       0.10 & 0.13 & 0.07 & 
                       0.58 & 0.15 & 0.13 & 
                       $-$1.85 & 1.71 & 2.77}


%!!!!!!!!!!!!!!!!!!!!!!!!!!!!!!!!!!!!!!!!!!!!!!!!!!!!!!!!!!!!!!!!!!!!!!!!!!!!!!!!!!!!!!

\begin{document}

\begin{table}[htbp]
{\color{red}{
 \centering
 \caption{
 Extracted fit parameters.
 The uncertainties marked as ``stat." are those returned by MINUIT, and those marked as ``syst." result from the systematic analysis.
 }
 \resizebox{\columnwidth}{!}{
 \begin{tabular}{|cV{4.0}c|c|c|}
  \cline{1-3}
  \multirow{2}{*}{\textbf{Centrality}} & \multirow{2}{*}{$\boldsymbol{\lambda}$} & \multirow{2}{*}{$\boldsymbol{R}$} & \multicolumn{1}{c}{} \\
   & & & \multicolumn{1}{c}{} \\
  \clineB{1-3}{4.0}  
   \multirow{2}{*}{0--10\%} 
     & \multirow{2}{*}{\ArrLamKchP(1) $\pm$ \ArrLamKchP(2) (stat.) $\pm$ \ArrLamKchP(3) (syst.)}    %Lambda (LamKchP 0010)
     & \multirow{2}{*}{\ArrLamKchP(4) $\pm$ \ArrLamKchP(5) (stat.) $\pm$ \ArrLamKchP(6) (syst.)}    %Radius (LamKchP & ALamKchM 0010)
     & \multicolumn{1}{c}{} \\    
    & & & \multicolumn{1}{c}{} \\     
   \cline{1-3}
   
   \multirow{2}{*}{10--30\%}
     & \multirow{2}{*}{\ArrLamKchP(7) $\pm$ \ArrLamKchP(8) (stat.) $\pm$ \ArrLamKchP(9) (syst.)}       %Lambda (LamKchP 1030)
     & \multirow{2}{*}{\ArrLamKchP(10) $\pm$ \ArrLamKchP(11) (stat.) $\pm$ \ArrLamKchP(12) (syst.)}    %Radius (LamKchP & ALamKchM 1030)    
     & \multicolumn{1}{c}{} \\  
    & & & \multicolumn{1}{c}{} \\  
   \cline{1-3}           
   
   \multirow{2}{*}{30--50\%}
     & \multirow{2}{*}{\ArrLamKchP(13) $\pm$ \ArrLamKchP(14) (stat.) $\pm$ \ArrLamKchP(15) (syst.)}    %Lambda (LamKchP 3050)
     & \multirow{2}{*}{\ArrLamKchP(16) $\pm$ \ArrLamKchP(17) (stat.) $\pm$ \ArrLamKchP(18) (syst.)}    %Radius (LamKchP & ALamKchM 3050)   
     & \multicolumn{1}{c}{} \\  
    & & & \multicolumn{1}{c}{} \\  
   %\clineB{1-3}{6.0}
   \cline{1-3}
  %%%%%%%%%%%%%%%%%%%%%%%%%%%%%%%%%%%%%%%%%%%%%%%%%%%%%%%%%%%%%%%%%%%%%%%%%%%%%%%%%%%%%%%%%%%%%%%%%%%%%%%%%%%%%%%%%%%%%%%%%%%%%%%%% 
   \multicolumn{1}{c}{} & \multicolumn{1}{c}{} & \multicolumn{1}{c}{} & \multicolumn{1}{c}{} \\
   %\hlineB{6.0}
   \hline
  \multirow{2}{*}{\textbf{System}} & \multirow{2}{*}{$\boldsymbol{\Re f_{0}}$} & \multirow{2}{*}{$\boldsymbol{\Im f_{0}}$} & \multirow{2}{*}{$\boldsymbol{d_{0}}$} \\
   & & & \\
  \clineB{1-4}{4.0}  
   \multirow{2}{*}{\LamKchP \& \ALamKchM} 
     & \multirow{2}{*}{\ArrLamKchP(19) $\pm$ \ArrLamKchP(20) (stat.) $\pm$ \ArrLamKchP(21) (syst.)}   %Ref0   (LamKchP & ALamKchM)
     & \multirow{2}{*}{\ArrLamKchP(22) $\pm$ \ArrLamKchP(23) (stat.) $\pm$ \ArrLamKchP(24) (syst.)}    %Imf0   (LamKchP & ALamKchM)
     & \multirow{2}{*}{\ArrLamKchP(25) $\pm$ \ArrLamKchP(26) (stat.) $\pm$ \ArrLamKchP(27) (syst.)} \\ %d0     (LamKchP & ALamKchM)
     
     & & & \\          
   \hline
   
   \multirow{2}{*}{\LamKchM \& \ALamKchP}
     & \multirow{2}{*}{\ArrLamKchM(19) $\pm$ \ArrLamKchM(20) (stat.) $\pm$ \ArrLamKchM(21) (syst.)}   %Ref0   (LamKchM & ALamKchP)
     & \multirow{2}{*}{\ArrLamKchM(22) $\pm$ \ArrLamKchM(23) (stat.) $\pm$ \ArrLamKchM(24) (syst.)}    %Imf0   (LamKchM & ALamKchP)
     & \multirow{2}{*}{\ArrLamKchM(25) $\pm$ \ArrLamKchM(26) (stat.) $\pm$ \ArrLamKchM(27) (syst.)} \\ %d0     (LamKchM & ALamKchP)     
     & & & \\  
   \hline           
   
   \multirow{2}{*}{\LamKs \& \ALamKs}
     & \multirow{2}{*}{\ArrLamKs(19) $\pm$ \ArrLamKs(20) (stat.) $\pm$ \ArrLamKs(21) (syst.)}   %Ref0   (LamKchM & ALamKchP)
     & \multirow{2}{*}{\ArrLamKs(22) $\pm$ \ArrLamKs(23) (stat.) $\pm$ \ArrLamKs(24) (syst.)}    %Imf0   (LamKchM & ALamKchP)
     & \multirow{2}{*}{\ArrLamKs(25) $\pm$ \ArrLamKs(26) (stat.) $\pm$ \ArrLamKs(27) (syst.)} \\ %d0     (LamKchM & ALamKchP)     
     & & & \\  
   \hline   
   
   
 \end{tabular}
 }
 %\caption{
 %Extracted fit parameters.
 %The uncertainties marked as ``stat." are those returned by MINUIT, and those marked as ``syst." result from the systematic analysis.
 %}
 \label{tab:FitResultsLamK_3Res}
 }}
\end{table}






%%%%%%%%%%%%%%%%%%%%%%Single rows.  For multi rows, see below
%\begin{comment}
\begin{table}[htbp]
{\color{blue}{
 \centering
 \caption{
 Extracted fit parameters.
 The uncertainties marked as ``stat." are those returned by MINUIT, and those marked as ``syst." result from the systematic analysis.
 }
 \resizebox{\columnwidth}{!}{
 \renewcommand{\arraystretch}{1.3}
 \begin{tabular}{c|c|c|c}
  \clineB{1-3}{3.0}
  Centrality & $\lambda_{\mathrm{Fit}}$ & \multicolumn{1}{c}{$R_{\mathrm{inv}}$} & \multicolumn{1}{c}{} \\
  \clineB{1-3}{3.0}  
   0--10\%
     & \ArrLamKchP(1) $\pm$ \ArrLamKchP(2) (stat.) $\pm$ \ArrLamKchP(3) (syst.)    %Lambda (LamKchP 0010)
     & \multicolumn{1}{c}{\ArrLamKchP(4) $\pm$ \ArrLamKchP(5) (stat.) $\pm$ \ArrLamKchP(6) (syst.)}    %Radius (LamKchP & ALamKchM 0010)
     & \multicolumn{1}{c}{} \\    
   \cline{1-3}
   
   10--30\%
     & \ArrLamKchP(7) $\pm$ \ArrLamKchP(8) (stat.) $\pm$ \ArrLamKchP(9) (syst.)      %Lambda (LamKchP 1030)
     & \multicolumn{1}{c}{\ArrLamKchP(10) $\pm$ \ArrLamKchP(11) (stat.) $\pm$ \ArrLamKchP(12) (syst.)}    %Radius (LamKchP & ALamKchM 1030)    
     & \multicolumn{1}{c}{} \\  
   \cline{1-3}           
   
   30--50\%
     & \ArrLamKchP(13) $\pm$ \ArrLamKchP(14) (stat.) $\pm$ \ArrLamKchP(15) (syst.)    %Lambda (LamKchP 3050)
     & \multicolumn{1}{c}{\ArrLamKchP(16) $\pm$ \ArrLamKchP(17) (stat.) $\pm$ \ArrLamKchP(18) (syst.)}    %Radius (LamKchP & ALamKchM 3050)   
     & \multicolumn{1}{c}{} \\  
   %\clineB{1-3}{6.0}
   \cline{1-3}
  %%%%%%%%%%%%%%%%%%%%%%%%%%%%%%%%%%%%%%%%%%%%%%%%%%%%%%%%%%%%%%%%%%%%%%%%%%%%%%%%%%%%%%%%%%%%%%%%%%%%%%%%%%%%%%%%%%%%%%%%%%%%%%%%% 
   \multicolumn{1}{c}{} & \multicolumn{1}{c}{} & \multicolumn{1}{c}{} & \multicolumn{1}{c}{} \\
   %\hlineB{6.0}
   \hlineB{3.0}
  System & $\Re f_{0}$ & $\Im f_{0}$ & $d_{0}$ \\
  \clineB{1-4}{3.0}  
   \LamKchP \& \ALamKchM 
     & \ArrLamKchP(19) $\pm$ \ArrLamKchP(20) (stat.) $\pm$ \ArrLamKchP(21) (syst.)   %Ref0   (LamKchP & ALamKchM)
     & \ArrLamKchP(22) $\pm$ \ArrLamKchP(23) (stat.) $\pm$ \ArrLamKchP(24) (syst.)    %Imf0   (LamKchP & ALamKchM)
     & \ArrLamKchP(25) $\pm$ \ArrLamKchP(26) (stat.) $\pm$ \ArrLamKchP(27) (syst.) \\ %d0     (LamKchP & ALamKchM)
     
   \hline
   
   \LamKchM \& \ALamKchP
     & \ArrLamKchM(19) $\pm$ \ArrLamKchM(20) (stat.) $\pm$ \ArrLamKchM(21) (syst.)   %Ref0   (LamKchM & ALamKchP)
     & \ArrLamKchM(22) $\pm$ \ArrLamKchM(23) (stat.) $\pm$ \ArrLamKchM(24) (syst.)    %Imf0   (LamKchM & ALamKchP)
     & \ArrLamKchM(25) $\pm$ \ArrLamKchM(26) (stat.) $\pm$ \ArrLamKchM(27) (syst.) \\ %d0     (LamKchM & ALamKchP)     
   \hline           
   
   \LamKs \& \ALamKs
     & \ArrLamKs(19) $\pm$ \ArrLamKs(20) (stat.) $\pm$ \ArrLamKs(21) (syst.)   %Ref0   (LamKchM & ALamKchP)
     & \ArrLamKs(22) $\pm$ \ArrLamKs(23) (stat.) $\pm$ \ArrLamKs(24) (syst.)    %Imf0   (LamKchM & ALamKchP)
     & \ArrLamKs(25) $\pm$ \ArrLamKs(26) (stat.) $\pm$ \ArrLamKs(27) (syst.) \\ %d0     (LamKchM & ALamKchP)     
   \hline   
   
   
 \end{tabular}
 }
 %\caption{
 %Extracted fit parameters.
 %The uncertainties marked as ``stat." are those returned by MINUIT, and those marked as ``syst." result from the systematic analysis.
 %}
 \label{tab:FitResultsLamK_3Res}
 }}
\end{table}
%\end{comment}




\end{document}
