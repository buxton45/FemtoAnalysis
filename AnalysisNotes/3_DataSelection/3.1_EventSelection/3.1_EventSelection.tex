\documentclass[/home/jesse/Analysis/FemtoAnalysis/AnalysisNotes/AnalysisNoteJBuxton.tex]{subfiles}
\begin{document}

\subsection{Event Selection and Mixing}
\label{EventSelection}

The events used in this study were selected with the class AliFemtoEventCutEstimators according to the following criteria:

\begin{itemize}
 \itemsep0em
 \item Triggers
 \begin{itemize}
  \itemsep0em
  \item minimum bias (kMB)
  \item central (kCentral)
  \item semi-central (kSemiCentral)
 \end{itemize}
 \item z-position of reconstructed event vertex must be within 10 cm of the center of the ALICE detector
 \item the event must contain at least one particle of each type from the pair of interest
\end{itemize}

The event mixing was handled by the AliFemtoVertexMultAnalysis class, which only mixes events with like vertex position and centrality.
The following criteria were used for event mixing:

\begin{itemize}
 \itemsep0em
 \item Number of events to mix = 5
 \item Vertex position bin width = 2 cm
 \item Centrality bin width = 5%
\end{itemize}

The AliFemtoEventReaderAODChain class is used to read the events.
Event flatteneing is not currently used.
FilterBit(7).
The centrality is determined by the ``V0M" method of AliCentrality, set by calling AliFemtoEventReaderAOD::SetUseMultiplicity(kCentrality).
We utilize the SetPrimaryVertexCorrectionTPCPoints switch, which causes the reader to shift all TPC points to be relative to the event vertex.

\end{document}