\documentclass[../AnalysisNoteJBuxton.tex]{subfiles}
\begin{document}

\subsection{V0 Selection}
\label{V0Selection}

$\Lambda$ ($\bar{\Lambda}$) and K$^{0}_{S}$ are neutral particles which cannot be directly detected, but must instead be reconstructed through detection of their decay products, or daughters.  
This process is illustrated in Figure \ref{fig:V0Reconstruction}.
In general, particles which are topologically reconstructed in this fashion are called V0 particles.
The class AliFemtoV0TrackCutNSigmaFilter (which is an extension of AliFemtoV0TrackCut) is used to reconstruct the V0s.

In order to obtain a true and reliable signal, one must ensure good purity of the V0 collection.  The purity of the collection is calculated as:

\begin{equation}
 Purity = \frac{Signal}{Signal + Background}
\label{eqn:Purity}
\end{equation}

In order to obtain both the signal and background, the invariant mass distribution (M$_{inv}$) of all V0 candidates must be constructed immediately before the final invariant mass cut.
Examples of such distribtions can be found in Figures \ref{fig:cLamPurity} and \ref{fig:K0Purity}.
It is vital that this distribution be constructed immediately before the final M$_{inv}$ cut, otherwise it would be impossible to estimate the background.
As shown in Figures \ref{fig:cLamPurity} and \ref{fig:K0Purity}, the background is fit (with a polynomial) outside of the peak region of interest to obtain an estimate for the backfround within the region.
Within the M$_{inv}$ cut limits, the background is the region below the fit while the signal is the region above the fit.

\begin{figure}[h]
  \centering
  \includegraphics[width=0.5\textwidth]{3_DataSelection/Figures/V0CutsGeneral.pdf}
  \caption[V0 Reconstruction]{V0 Reconstruction}
  \label{fig:V0Reconstruction}
\end{figure}

\subfile{3_DataSelection/3.3.1_LambdaReconstruction.tex}
\subfile{3_DataSelection/3.3.2_K0sReconstruction.tex}

\end{document}
