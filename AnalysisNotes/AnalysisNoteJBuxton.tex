%Need the following so subdirectories can all find ALICE_analysis_note.cls and
%other files needed from main directory
\makeatletter
\def\input@path{{/home/jesse/Analysis/FemtoAnalysis/AnalysisNotes/}}
%or: \def\input@path{{/path/to/folder/}{/path/to/other/folder/}}
\makeatother

%\documentclass[ALICE,manyauthors]{ALICE_analysis_notes}
\documentclass[ALICE,manyauthors]{ALICE_analysis_notes}


%\documentclass[ALICE,manyauthors]{ALICE_scientific_notes}
%
\usepackage{MyStyle}


%
\begin{document}%
%%%%%%%%%%%%% ptdr definitions %%%%%%%%%%%%%%%%%%%%%
%
%%%%%%%%%%%%%%%  Title page %%%%%%%%%%%%%%%%%%%%%%%%
%
\begin{titlepage}
%
\PHnumber{ALICE-ANA-2018-xxx} 
\PHdate{\today}
%
%%% Put your own title + short title here:
\title{\LamK and $\Xi^{-}$\Kpm Femtoscopy in Pb-Pb Collisions at $\sqrt{s_{\mathrm{NN}}}=2.76$ TeV from the LHC ALICE Experiment}
\ShortTitle{\LamK and $\Xi$K Femtoscopy with ALICE}   % appears on right page headers
%
\author{Jesse T. Buxton$^{1}$}
\author{
1. Department of Physics, The Ohio State University, Columbus, Ohio, USA\\
}
\author{Email: jesse.thomas.buxton@cern.ch}
%
\ShortAuthor{ALICE Analysis Note 2016}      % appears on left page headers, do not change
%
\begin{abstract}

\begin{comment}
My abstract will be contained here.  The abstract will introduce my study and inform the reader about the content of this paper.  I will state the problem I tackle, and summarize (in one sentence) why no one else has yet to adequately answered the research question.  Next, I will explain (again, in one sentence) how I tackled the research question, and (in one sentence) how I went about doing the research which followed from this big idea (i.e. elaborate on previous sentence).  Finally, as a single sentence, I will state the key impact of my research.
\end{comment}

\linenumbers

\begin{comment}
We present results from a femtoscopic analysis of \LamK and $\Xi^{-}$\Kpm correlations in Pb-Pb collisions at $\sqrt{s_{\mathrm{NN}}}$ = 2.76 TeV by the ALICE experiment at the LHC.  
All pair combinations of \Lam and \ALam with \KchP, \KchM and \Ks are analyzed.  
The femtoscopic correlations are the result of strong final-state interactions, and are fit with a parametrization based on a model by R. Lednicky and V. L. Lyuboshitz \cite{Lednicky:82}.  
This allows us to both characterize the emission source and measure the scattering parameters for the particle pairs.  
We observe a large difference in the \LamKchP and \LamKchM correlations in pairs with low relative momenta (\kstar $\lesssim$ 100 MeV).   
The results suggest an effect arising from different quark-antiquark interactions in the pairs, i.e. $\rm s\bar{s}$ in \LamKchP and $\rm u\bar{u}$ in \LamKchM, or from different net strangeness for each system.  
To gain further insight into this hypothesis, we currently are conducting a $\Xi$K femtoscopic analysis.
\end{comment}

I present the first femtoscopic analysis of \LamK correlations in Pb-Pb collisions at $\sqrt{s_{\mathrm{NN}}}$ = 2.76 TeV measured by the ALICE experiment at the Large Hadron Collider (LHC).  
All pair combinations of \Lam and \ALam with \KchP, \KchM and \Ks are analyzed.  
The femtoscopic correlations result from strong final-state interactions, and are fit with a parametrization allowing us to both characterize the emission source and measure the scattering parameters for the particle pairs.  
We observe a large difference in the \LamKchP and \LamKchM correlations in pairs with low relative momenta (\kstar $\lesssim$ 100 MeV/$c$).   
This might suggest an effect arising from different quark-antiquark interactions within the pairs ($\rm s\bar{s}$ in \LamKchP and $\rm u\bar{u}$ in \LamKchM), or from different net strangeness for each system.  
To gain further insight into this hypothesis, we are currently conducting a $\Xi^{-}$\Kpm femtoscopic analysis, for which I present preliminary results.

\end{abstract}
\end{titlepage}
%
%\input{alice_mynote.tex}               %%%%%%%%%%% put the body of the article here
%%\documentclass[10pt,a4paper]{article}
%\usepackage[latin1]{inputenc}
%\usepackage{amsmath}
%\usepackage{amsfonts}
%\usepackage{amssymb}
%\author{}
%\title{}
%\begin{document}

\providecommand{\abs}[1]{\left\lvert#1\right\rvert}
\tableofcontents
%\pagebreak  %using pagebreak makes uniform spacing over entire page
\newpage %using newpage instead makes things look better
\listoffigures
%\pagebreak
\newpage

\documentclass[../AnalysisNoteJBuxton.tex]{subfiles}
\begin{document}

\section{Introduction}

This will be my introduction.
Remember, Jai suggested to make each sentence a separate line to make changes easier to track in git.
Otherwise, git will treat an entire paragraph as a single line!

And a new paragraph begins with an empty line.

\end{document}
\section{Data Sample and Software}
\label{sec:DataSampleAndSoftware}

\subsection{Data Sample}
\label{sec:DataSample}
%\subsection{Analytical model}
Talk about the data sample 



\documentclass[../AnalysisNoteJBuxton.tex]{subfiles}
\begin{document}

\subsection{Software}
\label{sec:Software}
%\subsection{Analytical model}

The analysis was performed on the PWGCF analysis train using AliRoot v5-08-18-1 and AliPhysics vAN-20161027-1.

The main classes utilized include: AliFemtoVertexMultAnalysis, AliFemtoEventCutEstimators, AliFemtoESDTrackCutNSigmaFilter, AliFemtoV0TrackCutNSigmaFilter, AliFemtoXiTrackCut, AliFemtoV0PairCut, AliFemtoV0TrackPairCut, AliFemtoXiTrackPairCut, and AliFemtoAnalysisLambdaKaon.
All of these classes are contained in /AliPhysics/PWGCF/FEMTOSCOPY/AliFemto and .../AliFemtoUser.

\end{document}



\section{Data Selection}
\label{DataSelection}

\subsection{Event Selection and Mixing}
\label{EventSelection}

The events used in this study were selected with the following criteria:

\begin{itemize}
 \item Triggers
 \begin{itemize}
  \item minimum bias (kMB)
  \item central (kCentral)
  \item semi-central (kSemiCentral)
 \end{itemize}
 \item z-position of reconstructed event vertex must be within 10 cm of the center of the ALICE detector
 \item the event must contain at least one particle of each type from the pair of interest
\end{itemize}

The event mixing was handled by the AliFemtoVertexMultAnalysis class, which only mixes events with like vertex position and centrality.
The following criteria were used for event mixing:

\begin{itemize}
 \item Number of events to mix = 5
 \item Vertex position bin width = 2 cm
 \item Centrality bin width = 5%
\end{itemize}

The AliFemtoEventReaderAODChain class is used to read the events.
Event flatteneing is not currently used.
FilterBit(7).
The centrality is determined by the "V0M" method of AliCentrality, set by calling AliFemtoEventReaderAOD::SetUseMultiplicity(kCentrality).
I utilize the SetPrimaryVertexCorrectionTPCPoints switch, which causes the reader to shift all TPC points to be relative to the event vertex.
\subsection{Track Selection}
\label{TrackSelection}

Some general remarks on identifying particles

\input{3.2.1_ChargedKaonIdentification.tex}
\input{3.2.2_V0Selection.tex}
\input{3.2.3_CascadeReconstruction.tex}
\subsection{Pair Selection}
\label{PairSelection}

Some general remarks on forming pairs
\documentclass[../AnalysisNoteJBuxton.tex]{subfiles}
\begin{document}

\section{Correlation Functions}
\label{CorrelationFunctions}

%General remarks about formaton of correlation functions and what information they provide.

This analysis studies the momentum correlations of both $\Lambda$-K and $\Xi$-K pairs using the two-particle correlation function, defined as $C(k^{*}) = A(k^{*})/B(k^{*})$, where $A(k^{*})$ is the signal distribution, $B(k^{*})$ is the reference (or background) distribution, and $k^{*}$ is the momentum of one of the particles in the pair rest frame.
In practice, $A(k^{*})$ is constructed by binning in $k^{*}$ pairs from the same event.
Ideally, $B(k^{*})$ is similar to $A(k^{*})$ in all respects excluding the presence of femtoscopic correlations \cite{Lisa:2005dd}; as such, $B(k^{*})$ is used to divide out the phase-space effects, leaving only the femtoscopic effects in the correlation function. 

This analysis presents correlation functions for three centrality bins (0-10\%, 10-30\%, and 30-50\%), and is currently pair transverse momentum ($k_{T} = 0.5|\mathbf{p}_{T,1}+\mathbf{p}_{T,2}|$) integrated (i.e. not binned in $k_{T}$).  
The correlation functions are constructed separately for the two magnetic field configurations, and are combined using a weighted average:

\begin{equation}
  C_{combined}(k^{*}) = \frac{\sum\limits_{i}w_{i}C_{i}(k^{*})}{\sum\limits_{i}w_{i}} 
\label{eqn:CombineCfs}
\end{equation}

where the sum runs over the correlation functions to be combined, and the weight, $w_{i}$, is the number of numerator pairs in $C_{i}(k^{*})$.
Here, the sum is over the two field configurations.

\subfile{4_CorrelationFunctions/4.1_NormalCfs.tex}
\subfile{4_CorrelationFunctions/4.2_StavCfs.tex}

\clearpage

\end{document}
\documentclass[../AnalysisNoteJBuxton.tex]{subfiles}
\begin{document}

\section{Fitting}
\label{Fitting}

This section will include the Lednicky model and the method used to fit the Cascade study.  It will also include momentum resolution, residual correlations, and any other aspects to obtain a good fit

\subfile{5_Fitting/5.1_ModelLambdaKaon.tex}
\subfile{5_Fitting/5.2_ModelCascadeKaon.tex}
\subfile{5_Fitting/5.3_MomentumResolutionCorrections.tex}
\subfile{5_Fitting/5.4_ResidualCorrelations.tex}

\end{document}


\documentclass[../AnalysisNoteJBuxton.tex]{subfiles}
\begin{document}

\section{Systematic Errors}
\label{SystematicErrors}

This study is currently ongoing.  See Table \ref{tab:LamDcaLamK0}.

\subfile{6_SystematicErrors/6.1_SysErrsLamK0.tex}
\subfile{6_SystematicErrors/6.2_SysErrsLamKch.tex}

\end{document}
\documentclass[../AnalysisNoteJBuxton.tex]{subfiles}
\begin{document}
\clearpage

\section{Results and Discussion}
\label{ResultsAndDiscussion}

\subfile{7_ResultsAndDiscussion/7.1_ResultsLamK.tex}
\subfile{7_ResultsAndDiscussion/7.2_ResultsXiK.tex}


\end{document}
\documentclass[../AnalysisNoteJBuxton.tex]{subfiles}
\begin{document}

\section{To Do}
\label{ToDo}

\end{document}

%\end{document}

\subfile{0_Contents.tex}
\clearpage
\bibliographystyle{unsrt}
\bibliography{bibfile}
\end{document}
