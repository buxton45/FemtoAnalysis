\documentclass[/home/jesse/Analysis/FemtoAnalysis/AnalysisNotes/AnalysisNoteJBuxton.tex]{subfiles}

\renewcommand{\NonFlatBgdLamKch}{_NonFlatBgdCrctnLamK0LinearLamKchPolynomial}
\renewcommand{\NonFlatBgdLamKs}{_NonFlatBgdCrctnLamK0LinearLamKchPolynomial}

\renewcommand{\ResNum}{_3Res}
\renewcommand{\PrimMaxDecay}{_PrimMaxDecay10fm}

\renewcommand{\SaveNameModLamKch}{\MomRes\NonFlatBgdLamKch\ResNum\PrimMaxDecay\ResMethod\ParamFixAndShareLamKch}
\renewcommand{\SaveNameModLamKs}{\MomRes\NonFlatBgdLamKs\ResNum\PrimMaxDecay\ResMethod\ParamFixAndShareLamKch}

\begin{document}

\subsection{3 Residual Contributors Included in Fit}
\label{App_ResultsLamK_3Res}

This section presents our final results, for which three residual contributors were assumed.
These three contributors include: ($\Sigma^{0}\mathrm{K}$, $\Xi^{0}\mathrm{K}$, $\Xi^{-}\mathrm{K}$) $\rightarrow \Lambda\mathrm{K}$.
The figures presented do contain supplemental information to that presented in Sec. \ref{ResultsLamK}, but are here largely for convenience in comparing to the cases of including 10 (App. \ref{App_ResultsLamK_10Res}) and no (App. \ref{App_ResultsLamK_NoRes}) residual contributors.


%%%%%%%%%%%%%%%%%%%%%%%%%%%%%%%%%%%%%%%%%%%%%%%%%%%%%%%%%%%%%%%%%%%%%%
\begin{figure}[h]
  \centering
  \includegraphics[width=0.80\textwidth]{\ResultsDirBaseLamKch\SaveNameModLamKch/Comparisons/mTscaling_MinvCalc_OutlinedPoints_OthersTransparent_wJaiAndHans_3Res.pdf}
  \caption[$m_{\mathrm{T}}$ scaling of radii: 3 residuals]
  {
  3 residual correlations in \LamK fits.  
  Extracted fit $R_{\mathrm{inv}}$ parameters as a function of pair transverse mass ($m_{\mathrm{T}}$) for various pair systems over several centralities. 
  The ALICE published data \cite{Adam:2015vja} are shown with transparent, open symbols.  
  The new \LamK results are shown with opaque, filled symbols.  
  The \mt value for the \LamK system is an average of those for the \LamKchP, \ALamKchM, and \LamKs systems.
  }
  \label{figApp:mTScalingOfRadii_3Res}
\end{figure}


\begin{figure}[h]
  \centering
  \includegraphics[width=0.65\textwidth]{\ResultsDirBaseLamKch\SaveNameModLamKch/Comparisons/FinalResults_Comp3An_Vertical.pdf}
  \caption[Extracted scattering parameters: 3 residuals]
  {
  Extracted fit parameters for the case of 3 residual contributors for all of our \LamK systems.  
  [Top]: $\Im f_{0}$ vs. $\Re f_{0}$, together with $d_{0}$ to the right.  
  [Bottom]: $\lambda$ vs. Radius for the 0-10\% (blue), 10-30\% (green), and 30-50\% (red) centrality bins.  
  In the fit, all \LamK systems share common radii.
  The color scheme used in the panel are to be consistent with those in Fig. \ref{figApp:mTScalingOfRadii_3Res}.
  The cyan ([A] = Ref. \cite{Liu:2006xja}) and magenta ([B] = Ref. \cite{Mai:2009ce}) points show theoretical predictions made using chiral perturbation theory.
  }
  \label{figApp:ScattParams_3Res}
\end{figure}

%%%%%%%%%%%%%%%%%%%%%%%%%%%%%%%%%%%%%%%%%%%%%%%%%%%%%%%%%%%%%%%%%%%%%%
\begin{figure}[h]
  \centering
  \includegraphics[width=0.75\textwidth]{\ResultsDirBaseLamKch\SaveNameModLamKch/canKStarCfwFitsLamKchPwConj_0010_1030_3050UnZoomed\SaveNameModLamKch.pdf}
  \caption[\LamKchPALamKchM data with fits: 3 residuals]
  {
  Fit results, with 3 residual correlations included, for the \LamKchP and \ALamKchM data.
  The \LamKchP data is shown in the left column, the \ALamKchM in the right, and the rows differentiate the different centrality bins (0-10\% in the top, 10-30\% in the middle, and 30-50\% in the bottom).
  }
  \label{figApp:LamKchPwConjFits_3Res}
\end{figure}


\begin{figure}[h]
  \centering
  \includegraphics[width=0.75\textwidth]{\ResultsDirBaseLamKch\SaveNameModLamKch/Residuals\ResNum/LamKchP/canKStarCfwFitsAndResidualsLamKchPwConj_0010_1030_3050UnZoomed_ZoomResiduals\SaveNameModLamKch.pdf}
  \caption[\LamKchPALamKchM fit contribution from residuals: 3 residuals]
  {
  Fit results with the 3 residual contributions shown, for the \LamKchP and \ALamKchM data.
  The \LamKchP data is shown in the left column, the \ALamKchM in the right, and the rows differentiate the different centrality bins (0-10\% in the top, 10-30\% in the middle, and 30-50\% in the bottom).
  }
  \label{figApp:LamKchPwConjFitsAndResiduals_3Res}
\end{figure}





%%%%%%%%%%%%%%%%%%%%%%%%%%%%%%%%%%%%%%%%%%%%%%%%%%%%%%%%%%%%%%%%%%%%%%
\begin{figure}[h]
  \centering
  \includegraphics[width=0.75\textwidth]{\ResultsDirBaseLamKch\SaveNameModLamKch/canKStarCfwFitsLamKchMwConj_0010_1030_3050UnZoomed\SaveNameModLamKch.pdf}
  \caption[\LamKchMALamKchP data with fits: 3 residuals]
  {
  Fit results, with 3 residual correlations included, for the \LamKchM and \ALamKchP data.
  The \LamKchM data is shown in the left column, the \ALamKchP in the right, and the rows differentiate the different centrality bins (0-10\% in the top, 10-30\% in the middle, and 30-50\% in the bottom).
  }
  \label{figApp:LamKchMwConjFits_3Res}
\end{figure}


\begin{figure}[h]
  \centering
  \includegraphics[width=0.75\textwidth]{\ResultsDirBaseLamKch\SaveNameModLamKch/Residuals\ResNum/LamKchM/canKStarCfwFitsAndResidualsLamKchMwConj_0010_1030_3050UnZoomed_ZoomResiduals\SaveNameModLamKch.pdf}
  \caption[\LamKchMALamKchP fit contribution from residuals: 3 residuals]
  {
  Fit results with the 3 residual contributions shown, for the \LamKchM and \ALamKchP data.
  The \LamKchM data is shown in the left column, the \ALamKchP in the right, and the rows differentiate the different centrality bins (0-10\% in the top, 10-30\% in the middle, and 30-50\% in the bottom).
  }
  \label{figApp:LamKchMwConjFitsAndResiduals_3Res}
\end{figure}




%%%%%%%%%%%%%%%%%%%%%%%%%%%%%%%%%%%%%%%%%%%%%%%%%%%%%%%%%%%%%%%%%%%%%%
\begin{figure}[h]
  \centering
  \includegraphics[width=0.75\textwidth]{\ResultsDirBaseLamKs\SaveNameModLamKs/canKStarCfwFitsLamK0wConj_0010_1030_3050UnZoomed\SaveNameModLamKs.pdf}
  \caption[\LamALamKs data with fits: 3 residuals]
  {
  Fit results, with 3 residual correlations included, for the \LamKs and \ALamKs data.
  The \LamKs data is shown in the left column, the \ALamKs in the right, and the rows differentiate the different centrality bins (0-10\% in the top, 10-30\% in the middle, and 30-50\% in the bottom).
  }
  \label{figApp:LamKswConjFits_3Res}
\end{figure}


\begin{figure}[h]
  \centering
  \includegraphics[width=0.75\textwidth]{\ResultsDirBaseLamKs\SaveNameModLamKs/Residuals\ResNum/LamK0/canKStarCfwFitsAndResidualsLamK0wConj_0010_1030_3050UnZoomed_ZoomResiduals\SaveNameModLamKs.pdf}
  \caption[\LamKsALamKs fit contribution from residuals: 3 residuals]
  {
  Fit results with the 3 residual contributions shown, for the \LamKs and \ALamKs data.
  The \LamKs data is shown in the left column, the \ALamKs in the right, and the rows differentiate the different centrality bins (0-10\% in the top, 10-30\% in the middle, and 30-50\% in the bottom).
  }
  \label{figApp:LamKswConjFitsAndResiduals_3Res}
\end{figure}

\begin{comment}
%%%%%%%%%%%%%%%%%%%%%%%%%%%%%%%%%%%%%%%%     TABLES!!!!!     %%%%%%%%%%%%%%%%%%%%%%%%%%%%%%%%%%%%%%%%
%\pagestyle{empty}
\begin{landscape}

\subfile{\ResultsDirBaseLamKs\SaveNameModLamKs/Tables/ResultsTable_cLamK0.tex}
\subfile{\ResultsDirBaseLamKch\SaveNameModLamKch/Tables/ResultsTable_cLamcKch.tex}
\subfile{\ResultsDirBaseLamKch\SaveNameModLamKch/Tables/ResultsTableTriple.tex}

\end{landscape}
%\pagestyle{plain}
%%%%%%%%%%%%%%%%%%%%%%%%%%%%%%%%%%%%%%%%%%%%%%%%%%%%%%%%%%%%%%%%%%%%%%%%%%%%%%%%%%%%%%%%%%%%%%%%%%%%%
\end{comment}



\clearpage

\end{document}