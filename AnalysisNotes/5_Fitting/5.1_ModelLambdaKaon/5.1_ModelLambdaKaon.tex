\documentclass[/home/jesse/Analysis/FemtoAnalysis/AnalysisNotes/AnalysisNoteJBuxton.tex]{subfiles}
\begin{document}

\subsection{Model: \LamKs, \LamKpm, $\Xi^{-}$\Ks}
\label{ModelLambdaKaon}

%%%%%%%%%%%%%%%%%%%%%%%%%%%%%%%%%%%%%%%%%%%%%%%%%%%%%%%%%%%%%%%%%%%%%%%%%%%%%%%%%%%%%%%%%%%%%%%%%%%%%%%%%%%%%%%%%%%%%%%
\subsubsection*{Strong Interaction Only}
\label{StrongIntOnly}

%%%%%%%%%%%%%%%%%%%%%%%%%%%%%%%%%%%%%%%%%%%%%%%%%%%%%%%%%%%%%%%%%%%%%%%%%%%%%%%%%%%%%%%%%%%%%%%%%%%%%%%
\begin{comment}
The Bethe-Salpeter(?) amplitude for interacting particles is given by

\begin{equation}
\psi = e^{-iP(x_{1}+x_{2})/2}\left( e^{-ikr} + \phi_{p_{1}p_{2}}(r) \right)
\label{eqn:TempEq1}
\end{equation}
where $\phi_{p_{1}p_{2}}(r)$ describes the FSI of the particles. 
More specifically,  $\phi_{p_{1}p_{2}}(r)$ can be calculated as:

\begin{equation}
\begin{aligned}
\phi_{p_{1}p_{2}}(x) = \frac{8\pi\sqrt{p^{2}}}{(2\pi)^{4}i}e^{ipx}\int \frac{e^{-i\kappa x}f(p_{1}, p_{2}, \kappa, 2p-\kappa)}{(\kappa^{2}-m_{1}^{2})((2p-\kappa)^2 - m_{2}^{2})}d^{4}\kappa
\end{aligned}
\end{equation}


Range of potential smaller than r $\rightarrow$ $f(p_{1}, p_{2}, \kappa, 2p-\kappa) \approx f(k^{*})$, and we can pull it out of the integral.

\begin{equation}
\begin{aligned}
\phi_{p_{1}p_{2}}(x) = f(k^{*}) \Phi_{p_{1}p_{2}}(x)
\end{aligned}
\end{equation}

In the effective range approximation, $f(k^{*})$ is of the form:

\begin{equation}
\begin{aligned}
f(k^{*}) = \left( \frac{1}{f_{0}} + \frac{1}{2}d_{0}k^{*2} - ik^{*} \right)^{-1}
\end{aligned}
\label{eqn:ScatteringParam}
\end{equation}



Furthermore, for $k* << m$,  and particles emitted at same time in the PRF, $\Phi_{p_{1}p_{2}}(x)$ reduces to
\begin{equation}
\begin{aligned}
\Phi_{p_{1}p_{2}}(x) = \frac{e^{ik^{*}r^{*}}}{r^{*}}
\end{aligned}
\label{eqn:PhiP1P2SphericalWave}
\end{equation}
\end{comment}
%%%%%%%%%%%%%%%%%%%%%%%%%%%%%%%%%%%%%%%%%%%%%%%%%%%%%%%%%%%%%%%%%%%%%%%%%%%%%%%%%%%%%%%%%%%%%%%%%%%%%%%%%%%%%%

For the case of interacting particles, the non-symmetrized two-particle wave-function, $\psi$, may be written as \cite{Lednicky:82}
\begin{equation}
\psi = e^{-iP(x_{1}+x_{2})/2}\left( e^{-ikr} + \phi_{p_{1}p_{2}}(r) \right)
\label{eqn:TempEq1}
\end{equation}
where $\phi_{p_{1}p_{2}}(r)$ describes the final state interactions (FSI) of the particles. 
Assuming that the range of the interaction potential is smaller than the distance between the emission points, $\phi_{p_{1}p_{2}}(r)$ may be expressed as 
\begin{equation}
\begin{aligned}
\phi_{p_{1}p_{2}}(x) = f(k^{*}) \Phi_{p_{1}p_{2}}(x)
\end{aligned}
\end{equation}
where, in the effective range approximation, $f(k^{*})$ is of the form
\begin{equation}
\begin{aligned}
f(k^{*}) = \left( \frac{1}{f_{0}} + \frac{1}{2}d_{0}k^{*2} - ik^{*} \right)^{-1}
\end{aligned}
\label{eqn:ScatteringParam}
\end{equation}
Furthermore, if we assume the magnitude of the momentum difference is much smaller than the mass of the particles ($k* \ll m$), and that the particles are emitted at the same time in the pair rest frame, $\Phi_{p_{1}p_{2}}(x)$ reduces to
\begin{equation}
\begin{aligned}
\Phi_{p_{1}p_{2}}(x) = \frac{e^{ik^{*}r^{*}}}{r^{*}}
\end{aligned}
\label{eqn:PhiP1P2SphericalWave}
\end{equation}
In other words, $\Phi_{p_{1}p_{2}}(x)$ is a diverging spherical wave.
Therefore, we have $\psi_{p_{1}p_{2}} = e^{-ikr} + \phi_{p_{1}p_{2}}(r)$, where $\phi_{p_{1}p_{2}}(r)$ is independent of the directions of the vectors $\boldsymbol k^{*}$ and $\boldsymbol r$ \cite{Lednicky:1995vk}.


Assuming the final state interactions are independent of the spin state, for identical interacting particles,

\begin{equation}
\begin{aligned}
|\Psi|^{2} &= 1 + g_{0}\cos(2kx) + g_{i}\left\lbrace|\phi_{p_{1}p_{2}}(r)|^{2} + 2\Re[\phi_{p_{1}p_{2}}(r)]\cos(kx)\right\rbrace
\end{aligned}
\label{eqn:IntIdPart_SpinIndependent}
\end{equation} 
where $g_{0} = (-1)^{2j}/(2j+1)$ and $g_{i} = 1+g_{0}$.

For non-identical interacting particles, where we need not worry about the symmetrization of the wave function, we have:

\begin{equation}
\begin{aligned}
|\Psi_{p_{1}p_{2}}|^{2} &= 1 + |f(k^{*})\Phi_{p_{1}p_{2}}|^{2} + e^{ikx}f(k^{*})\Phi_{p_{1}p_{2}} + e^{-ikx}\tilde{f}(k^{*})\tilde{\Phi}_{p_{1}p_{2}} \\
&= 1 + |f(k^{*})\Phi_{p_{1}p_{2}}|^{2} + 2\Re[f(k^{*})\Phi_{p_{1}p_{2}}]\cos(kx) - 2\Im[f(k^{*})\Phi_{p_{1}p_{2}}]\sin(kx)
\end{aligned}
\label{eqn:IntNonIdPart_SpinIndependent}
\end{equation}



If we allow the interaction, and therefore the two-particle wave function, to depend on the summary spin of the system (but not on the spin projections), things change slightly.
Instead of grouping all odd and even summary spins together, as was done in deriving Eqns. \ref{eqn:IntIdPart_SpinIndependent} and \ref{eqn:IntNonIdPart_SpinIndependent}, we need to keep track of their individual contributions, as their wave functions will be different.
When combining all of the various states with different summary spin $S$, they must be combined with a weight factor $\rho_{S}$, the normalized emission probability for such a state ($\sum_{S}\rho_{s} = 1$).
In this case, we find 

\begin{equation}
\begin{aligned}
|\Psi|^{2} &= 1 + \left(\sum_{S=even}\rho_{S} - \sum_{S=odd}\rho_{S} \right)\cos(2kx) + ... \\
&~~~+ 2\sum_{S=even}\rho_{S}\left\lbrace |\phi^{S}_{p_{1}p_{2}}(r)|^{2} + 2\Re[\phi^{S}_{p_{1}p_{2}}(r)]\cos(kx) \right\rbrace \\
&= 1 + g_{0}^{\prime}\cos(2kx) + 2\sum_{S=even}\rho_{S}\left\lbrace |\phi^{S}_{p_{1}p_{2}}(r)|^{2} + 2\Re[\phi^{S}_{p_{1}p_{2}}(r)]\cos(kx) \right\rbrace
\end{aligned}
\end{equation}
where $g_{0}^{\prime} = \sum_{S=even}\rho_{S} - \sum_{S=odd}\rho_{S}$.
In the case of unpolarized emission for identical particle pairs, $\rho_{s} = (2S+1)/(2j+1)^{2}$, and we have $g_{0}^{\prime} = g_{0} = (-1)^{2j}/(2j+1)$.

The case of interacting non-identical particles can be written down immediately from Eq. \ref{eqn:IntNonIdPart_SpinIndependent}

\begin{equation}
\begin{aligned}
|\Psi_{p_{1}p_{2}}|^{2} &= \sum_{S}\rho_{s}\left\lbrace 1 + |\phi^{S}_{p_{1}p_{2}}|^{2} + 2\Re[\phi^{S}_{p_{1}p_{2}}]\cos(kx) - 2\Im[\phi^{S}_{p_{1}p_{2}}]\sin(kx) \right\rbrace
\end{aligned}
\label{eqn:IntNonIdPart}
\end{equation}
where, for the case of unpolarized emission, $\rho_{s} = (2S+1)/[(2j_{1}+1)(2j_{2}+1)]$.


Now, up to this point, we have discussed only the wave-functions.
Within femtoscopy, this essentially amounts to assuming point-like sources.
In the more realistic scenario, the correlation function should be averaged over the space-time distribution of particle sources.
The two-particle relative momentum correlation function may be written theoretically by the Koonin-Pratt equation \cite{Koonin:1977fh, Pratt:1990zq}:

\begin{equation}
 C(\mathbf{k^{*}}) = \int S(\mathbf{r^{*}})|\Psi_{\mathbf{k^{*}}}(\mathbf{r^{*}})|^{2}d^{3}\mathbf{r^{*}}
\label{eqn:KooninPrattEqn}
\end{equation}
where $S(\mathbf{r^{*}})$ is the pair source distribution, $\Psi_{\mathbf{k^{*}}}(\mathbf{r^{*}})$ is the two-particle wave-function, and \kstar is the momentum of one particle in the pair rest frame.  
In the case on one-dimensional analyses, a spherically symmetric Gaussian pair emission source in the PRF with size $R_{\mathrm{inv}}$ is often assumed:

\begin{equation}
\begin{aligned}
S(x) \propto \exp\left[ -\frac{r^{2}}{4R_{\mathrm{inv}}^{2}}\right]
\end{aligned}
\label{eqn:Gaussian1D}
\end{equation}
In the case of identical particles, where the Gaussian offsets are zero and the single particle sources are obviously described by the same radii, the two-particle source is related by a factor $\sqrt{2}$ to the single particle sizes. 

As a side note, a more realistic source would allow for different radii in the out, side and long directions.
Furthermore, for non-identical particle femtoscopy, each single-particle source should be given a unique offset in the out direction.
In such a case, the pair emission source would be
\begin{equation}
\begin{aligned}
S_{AB}(\mathbf{r}) &\propto \int \exp\left(-\frac{(r_{a, out}-\mu_{a, out})^{2}}{2R^{2}_{a, out}} - \frac{r^{2}_{a, side}}{2R^{2}_{a, side}} - \frac{r^{2}_{a, long}}{2R^{2}_{a, long}} \right) \times ... \\
&~~~~~\times \exp\left(-\frac{(r_{b, out}-\mu_{b, out})^{2}}{2R^{2}_{b, out}} - \frac{r^{2}_{b, side}}{2R^{2}_{b, side}} - \frac{r^{2}_{b, long}}{2R^{2}_{b, long}} \right) \\
&~~~~~\times \delta(r_{out}-r_{a,out}+r_{b,out})dr_{a,out}dr_{b,out} \\
&~~~~~\times \delta(r_{side}-r_{a,side}+r_{b,side})dr_{a,side}dr_{b,side} \\
&~~~~~\times \delta(r_{long}-r_{a,long}+r_{b,long})dr_{a,long}dr_{b,long}
\end{aligned}
\label{eqn:SingleGaussSourceWithShift}
\end{equation}
Using the results from Appendix \ref{app:ProdTwoGauss}, we find
\begin{equation}
\begin{aligned}
S_{AB}(\mathbf{r}) &\propto \exp\left(-\frac{[r_{out}-(\mu_{a, out}-\mu_{b, out})]^{2}}{2(R_{a,out}^{2}+R_{b,out}^{2})}\right) \times ... \\
&~~~~\times \exp\left(-\frac{r_{side}^{2}}{2(R_{a,side}^{2}+R_{b,side}^{2})}\right) \times ... \\
&~~~~\times \exp\left(-\frac{r_{long}^{2}}{2(R_{a,long}^{2}+R_{b,long}^{2})}\right)
\end{aligned}
\label{eqn:PairGaussSourceWithShift}
\end{equation}
which demonstrates $\mu_{ab, out} = \mu_{a, out}-\mu_{b, out}$, and $R_{ab, i}^{2} = R_{a, i}^{2} + R_{b, i}^{2}$ \cite{Kisiel:2009eh}.

Returning back to our treatment of a one dimensional femtoscopic study with a spherically symmetric source of width $R$ (Eq. \ref{eqn:Gaussian1D}).
In the absence of Coulomb effects, with the assumptions and approximations used above (including unpolarized emission), the 1D femtoscopic correlation function can be calculated analytically as

\begin{equation}
 C(k^{*}) = 1 + C_{QI}(k^{*}) + C_{FSI}(k^{*})
\label{eqn:LednickyEqn}
\end{equation}
$C_{QI}$ describes plane-wave quantum interference:

\begin{equation}
 C_{QI}(k^{*}) = \alpha\exp(-4k^{*2}R^{2})
\label{eqn:CQI}
\end{equation}
where $\alpha = (-1)^{2j}/(2j+1)$ (i.e. $= g_{0}$) for identical particles with spin j, and $\alpha = 0$ for non-identical particles.  For all analyses presented in this note, $\alpha = 0$.  $C_{FSI}$ describes the s-wave strong final state interaction between the particles:

\begin{equation}
\begin{aligned}
C_{FSI}(k^{*}) &= \sum_{S}\rho_{S}\left[\frac{1}{2}\left|\frac{f^{S}(k^{*})}{R}\right|^2\left(1-\frac{d^{S}_{0}}{2\sqrt{\pi}R}\right)+\frac{2\Re f^{S}(k^{*})}{\sqrt{\pi}R}F_{1}(2k^{*}R)-\frac{\Im f^{S}(k^{*})}{R}F_{2}(2k^{*}R)\right]
\end{aligned}  
\label{eqn:CFSI}
\end{equation}
where $\rho_{s} = (2S+1)/[(2j_{1}+1)(2j_{2}+1)]$ and
\begin{equation}
\begin{aligned}
F_{1}(z) &= \int_{0}^{z} \frac{e^{x^{2}-z^{2}}}{z}dx \qquad \qquad F_{2}(z) &= \frac{1-e^{-z^{2}}}{z}
\end{aligned}  
\label{eqn:CFSI2}
\end{equation}
where $R$ is the source size, $f(k^{*})$ is the s-wave scattering amplitude, $f_{0}$ is the complex scattering length, and $d_{0}$ is the effective range of the interaction.

An additional parameter $\lambda$ is typically included in the femtoscopic fit function to account for the purity of the pair sample.  In the case of no residual correlations (to be discussed in Section \ref{ResidualCorrelations}), the fit function becomes:

\begin{equation}
 C(k^{*}) = 1 + \lambda[C_{QI}(k^{*}) + C_{FSI}(k^{*})]
\label{eqn:LednickyEqnwLambda}
\end{equation}

%%%%%%%%%%%%%%%%%%%%%%%%%%%%%%%%%%%%%%%%%%%%%%%%%%%%%%%%%%%%%%%%%%%%%%%%%%%%%%%%%%%%%%%%%%%%%%%%%%%%%%%%%%%%%%%%%%%%%%%


\end{document}