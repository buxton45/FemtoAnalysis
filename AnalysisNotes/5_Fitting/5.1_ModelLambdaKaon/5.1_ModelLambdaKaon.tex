\documentclass[../AnalysisNoteJBuxton.tex]{subfiles}
\begin{document}

\subsection{Model: \LamKs, \LamKpm, $\Xi^{-}$\Ks}
\label{ModelLambdaKaon}

The two-particle relative momentum correlation function may be written theoretically by the Koonin-Pratt equation \cite{Koonin:1977fh, Pratt:1990zq}:

\begin{equation}
 C(\mathbf{k^{*}}) = \int S(\mathbf{r^{*}})|\Psi_{\mathbf{k^{*}}}(\mathbf{r^{*}})|^{2}d^{3}\mathbf{r^{*}}
\label{eqn:KooninPrattEqn}
\end{equation}

where $S(\mathbf{r^{*}})$ is the pair source distribution, $\Psi_{\mathbf{k^{*}}}(\mathbf{r^{*}})$ is the two-particle wave-function, and \kstar is the momentum of one particle in the pair rest frame.  
In the absence of Coulomb effects, and assuming a spherically Gaussian source of width $R$, and s-wave scattering, the 1D femtoscopic correlation function can be calculated analytically using:

\begin{equation}
 C(k^{*}) = 1 + C_{QI}(k^{*}) + C_{FSI}(k^{*})
\label{eqn:LednickyEqn}
\end{equation}

$C_{QI}$ describes plane-wave quantum interference:

\begin{equation}
 C_{QI}(k^{*}) = \alpha\exp(-4k^{*2}R^{2})
\label{eqn:CQI}
\end{equation}

where $\alpha = (-1)^{2j}/(2j+1)$ for identical particles with spin j, and $\alpha = 0$ for non-identical particles.  For all analyses presented in this note, $\alpha = 0$.  $C_{FSI}$ describes the s-wave strong final state interaction between the particles:

\begin{equation}
\begin{array}{l}
\vspace{2mm}  %%space between C_{FSI}(k^{*}) and f(k^{*})
  C_{FSI}(k^{*}) = (1+\alpha)[\frac{1}{2}|\frac{f(k^{*})}{R}|^2(1-\frac{d_{0}}{2\sqrt{\pi}R})+\frac{2\mathbb{R}f(k^{*})}{\sqrt{\pi}R}F_{1}(2k^{*}R)-\frac{\mathbb{I}f(k^{*})}{R}F_{2}(2k^{*}R)] \\
\vspace{2mm}  %%space after f(k^{*})  
  ~~~~~f(k^{*}) = (\frac{1}{f_{0}}+\frac{1}{2}d_{0}k^{*2}-ik^{*})^{-1};~~~
  F_{1}(z) = \int_{0}^{z} \frac{e^{x^{2}-z^{2}}}{z}dx;~~~
  F_{2}(z) = \frac{1-e^{-z^{2}}}{z}
\end{array}  
\label{eqn:CFSI}
\end{equation}

where $R$ is the source size, $f(k^{*})$ is the s-wave scattering amplitude, $f_{0}$ is the complex scattering length, and $d_{0}$ is the effective range of the interaction.

An additional parameter $\lambda$ is typically included in the femtoscopic fit function to account for the purity of the pair sample.  In the case of no residual correlations (to be discussed in Section \ref{ResidualCorrelations}), the fit function becomes:

\begin{equation}
 C(k^{*}) = 1 + \lambda[C_{QI}(k^{*}) + C_{FSI}(k^{*})]
\label{eqn:LednickyEqnwLambda}
\end{equation}

\end{document}