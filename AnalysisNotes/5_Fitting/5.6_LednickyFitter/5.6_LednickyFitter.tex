\documentclass[/home/jesse/Analysis/FemtoAnalysis/AnalysisNotes/AnalysisNoteJBuxton.tex]{subfiles}
\begin{document}

\subsection{LednickyFitter}
\label{LednickyFitter}


The code developed to fit the data is called ``LednickyFitter", and utilizes the ROOT TMinuit implementation of the MINUIT fitting package.
In short, given a function with a number of parameters, the fitter explores the parameter space searching for the minimum of the function.
In this implementation, the function to be minimized should represent the difference between the measured and theoretical correlation functions.
However, a simple $\chi^{2}$ test is inappropriate for fitting correlation functions, as the ratio of two Poisson distributions does not result in a Poisson distribution.
Instead, a log-likelihood fit function of the following form is used \cite{Lisa:2005dd}:

\begin{equation}
 \chi^{2}_{PML} = -2\left[A\ln\left(\frac{C(A+B)}{A(C+1)}\right) + B\ln\left(\frac{A+B}{B(C+1)}\right)\right]
\label{eqn:Chi2PML}
\end{equation}

where $A$ is the experimental signal distribution (numerator), $B$ is the experimental background distribution (denominator), and $C$ is the theoretical fit correlation function.
Therefore, we use Eq. \ref{eqn:Chi2PML} as the statistic quantifying the quality of the fit.
The parameters of the fit are: $\lambda$, $R$, $f_{0}$ ($\Re f_{0}$ and $\Im f_{0}$ separately), $d_{0}$, and normalization $N$.

With our procedure, we are able to share parameters between different analyses and fit all simultaneously.
A given pair and its conjugate (e.g. \LamKchP and \ALamKchM) always share scattering parameters ($\Re f_{0}$, $\Im f_{0}$, $d_{0}$).
However, the three distinct analyses (\LamKchP, \LamKchM, and \LamKs) are assumed to have scattering parameters unique from each other.
We assume the pair emission source for a given centrality class is similar between all analyses; therefore, for each centrality, all \LamK analyses share a common radius parameter.
We assume the same is true for the overall normalization $\lambda$ parameters in Eq. \ref{eqn:CfwRes}.
Finally, each correlation function has a unique normalization parameter.

All correlation functions were normalized in the range 0.32 $< k^{*} <$ 0.40 GeV/c, and fit in the range 0.0 $< k^{*} <$ 0.30 GeV/c.
For the \LamKchM analysis, the region 0.19 $< k^{*} <$ 0.23 GeV/$c$ was excluded from the fit to exclude the bump caused by the $\Omega^{-}$ resonance.
For each pair system, we account for contributions from three residual contributors, as discussed in Sec. \ref{ResidualCorrelations}, and whose individual $\lambda$ values are listed in Table \ref{tab:LambdaValues_All} (the cases of zero and ten residual contributors were also investigated, but the case of three contributors was deemed most reasonable).
We account for effects of finite track momentum resolution, as outlined in Sec. \ref{MomentumResolutionCorrections}.
The non-femtoscopic backgrounds are modeled using the THERMINATOR 2 simulation for the \LamKpm analyses, and with a linear form for the \LamKs system, as described in Sec. \ref{NonFlatBackground}.
In general, corrections are applied to the fit function, the raw data is never touched.

To summarize, the complete fit function is constructed as follows:
\begin{enumerate}
 \item The uncorrected, primary, correlation function, $C_{\Lambda\mathrm{K}}$(\ktrue), is constructed using Eqns. \ref{eqn:LednickyEqn} and \ref{eqn:CFSI}
 \item The correlation functions describing the parent systems which contribute residually are obtained using:
 \begin{itemize}
  \item Eqns. \ref{eqn:LednickyEqn} and \ref{eqn:CFSI} for the case of Coulomb-neutral pairs
  \item \XiKpm experimental data for \XiKpm contributions
  \item a Coulomb-only curve, with the help of Appendix \ref{App:CoulombFitter}, for other pairs including the Coulomb interaction 
 \end{itemize} 
 \item The residual contributions to the \LamK correlation function is found by running each parent correlation function through the appropriate transform matrix, via Eq.\ref{eqn:ResidualsTransform}
 \item The primary and residual correlations are combined, via Eq.\ref{eqn:Residuals} with Tab. \ref{tab:LambdaValues_All}, to form $C'_{Fit}$(\ktrue)
 \item The correlation function is corrected to account for momentum resolution effects using Eq. \ref{eqn:MomResCorrection}, to obtain $C'_{\mathrm{Fit}}(k^{*}_{\mathrm{Rec}})$
 \item Finally, the non-flat background correction, $F_{\mathrm{Bgd}}(k^{*}_{\mathrm{Rec}})$ is applied, and the final fit function is obtained, $C_{\mathrm{Fit}}(k^{*}_{\mathrm{Rec}}) = C'_{\mathrm{Fit}}(k^{*}_{\mathrm{Rec}})*F_{\mathrm{Bgd}}(k^{*}_{\mathrm{Rec}})$
\end{enumerate}

Figures \ref{fig:LamK0wConjFits_NoRes}, \ref{fig:LamKchPwConjFits_NoRes}, and \ref{fig:LamKchMwConjFits_NoRes} (\ref{fig:LamK0wConjFits_3Res}, \ref{fig:LamKchPwConjFits_3Res}, and \ref{fig:LamKchMwConjFits_3Res}, or \ref{fig:LamK0wConjFits_10Res}, \ref{fig:LamKchPwConjFits_10Res}, and \ref{fig:LamKchMwConjFits_10Res}), in Section \ref{ResultsAndDiscussion}, show experimental data with fits for all studied centralities for \LamKsALamKs, \LamKchPALamKchM, and \LamKchMALamKchP, respectively.  In the figures, the black solid line represents the ``raw" fit, i.e. not corrected for momentum resolution effects nor non-flat background.  The green line shows the fit to the non-flat background.  The purple points show the fit after momentum resolution, non-flat background, and residual correlations (if applicable) corrections have been applied.  The extracted fit values with uncertainties are also printed on the figures.

\end{document}