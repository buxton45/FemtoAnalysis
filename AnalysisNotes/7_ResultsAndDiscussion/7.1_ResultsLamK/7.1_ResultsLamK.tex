\documentclass[/home/jesse/Analysis/FemtoAnalysis/AnalysisNotes/AnalysisNoteJBuxton.tex]{subfiles}
\begin{document}

%\subsection{Results: \texorpdfstring{$\Lambda$K$^{0}_{S}$ and $\Lambda$K$^{\pm}$}{TEXT}}
\subsection{Results: \LamKs and \LamKpm}
\label{ResultsLamK}

In the following sections, we present results assuming (i) three residual contributors (Sec. \ref{ResultsLamK_3Res}), (ii) ten residual contributors (Sec. \ref{ResultsLamK_10Res}), and (iii) no residual correlations (Sec. \ref{ResultsLamK_NoRes}).

For the results shown, unless otherwise noted, the following hold true:
All correlation functions were normalized in the range 0.32 $< k^{*} <$ 0.40 GeV/c, and fit in the range 0.0 $< k^{*} <$ 0.30 GeV/c.
For the \LamKchM and \ALamKchP analyses, the region 0.19 $< k^{*} <$ 0.23 GeV/$c$ was excluded from the fit to exclude the bump caused by the $\Omega^{-}$ resonance.
The non-femtoscopic backgrounds for the \LamKchP and \LamKchM systems were modeled by a (6$^{\mathrm{th}}$-)order polynomial fit to THERMINATOR simulation, while those for the \LamKs were fit with a simple linear form.
The \LamKchPALamKchM radii are shared with \LamKchMALamKchP, while the \LamKsALamKs radii are unique.
In the figures showing experimental correlation functions with fits, the black solid line represents the primary (\LamK) correlation's contribution to the fit.
The green line shows the fit to the non-flat background.  
The purple points show the fit after all residuals' contributions have been included, and momentum resolution and non-flat background corrections have been applied.


Before beginning, I first collect a summary of my final results in Figure \ref{fig:ScattParams_3Res}.  
In the summary plot, we show the extracted scattering parameters in the form of a $\Im f_{0}$ vs $\Re f_{0}$ plot, which includes the $d_{0}$ values to the right side.  
We also show the $\lambda$ vs. radius parameters for all three of our studied centrality bins.  
In Fig. \ref{fig:ScattParams_3Res}, three residual contributors were used.
For the \LamKs results shown in the figure, the \LamKs and \ALamKs analyses were fit simultaneously across all centralities (0-10\%, 10-30\%, 30-50\%); scattering parameters and a single $\lambda$ parameter were shared amongst all, the radii were shared amongst results of like-centrality, and each has a unique normalization parameter.  
For the \LamKpm results shown, all four pair combinations were fit simultaneously (\LamKchP, \ALamKchM, \LamKchM, \ALamKchP) across all centralities.  
Scattering parameters were shared between pair-conjugate systems (i.e. a parameter set describing \LamKchP \& \ALamKchM, and a separate set describing \LamKchM \& \ALamKchP).  
For each centrality, a radius and $\lambda$ parameters were shared between all pairs.  Each analysis has a unique normalization parameter.


\clearpage

\subfile{7_ResultsAndDiscussion/7.1_ResultsLamK/7.1.1_ResultsLamK_3Res/7.1.1_ResultsLamK_3Res.tex}
\subfile{7_ResultsAndDiscussion/7.1_ResultsLamK/7.1.2_ResultsLamK_10Res/7.1.2_ResultsLamK_10Res.tex}
\subfile{7_ResultsAndDiscussion/7.1_ResultsLamK/7.1.3_ResultsLamK_NoRes/7.1.3_ResultsLamK_NoRes.tex}
\subfile{7_ResultsAndDiscussion/7.1_ResultsLamK/7.1.4_ResultsLamK_FitMethodComparisons/7.1.4_ResultsLamK_FitMethodComparisons.tex}
\subfile{7_ResultsAndDiscussion/7.1_ResultsLamK/7.1.5_ResultsLamK_DiscussionOfmTScaling/7.1.5_ResultsLamK_DiscussionOfmTScaling.tex}

\end{document}