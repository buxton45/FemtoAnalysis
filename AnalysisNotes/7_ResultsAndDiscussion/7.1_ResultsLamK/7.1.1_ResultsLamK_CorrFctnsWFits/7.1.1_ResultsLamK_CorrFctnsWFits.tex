\documentclass[/home/jesse/Analysis/FemtoAnalysis/AnalysisNotes/AnalysisNoteJBuxton.tex]{subfiles}

\renewcommand{\NonFlatBgdLamKch}{_NonFlatBgdCrctnLamK0LinearLamKchPolynomial}
\renewcommand{\NonFlatBgdLamKs}{_NonFlatBgdCrctnLamK0LinearLamKchPolynomial}

\renewcommand{\ResNum}{_3Res}
\renewcommand{\PrimMaxDecay}{_PrimMaxDecay10fm}

\renewcommand{\SaveNameModLamKch}{\MomRes\NonFlatBgdLamKch\ResNum\PrimMaxDecay\ResMethod\ParamFixAndShareLamKch}
\renewcommand{\SaveNameModLamKs}{\MomRes\NonFlatBgdLamKs\ResNum\PrimMaxDecay\ResMethod\ParamFixAndShareLamKch}

\begin{document}

\subsubsection{Correlation functions with fits}
\label{ResultsLamK_CorrFctnsWFits}

Figures \ref{fig:LamKchPwConjFits_3Res}, \ref{fig:LamKchMwConjFits_3Res}, and \ref{fig:LamK0wConjFits_3Res} show the experimental correlation functions with fits, assuming 3 residual contributors, for all \LamK systems (\LamKchP, \LamKchM, and \LamKs, respectively) in all studied centralities.
The parameter sets extracted from the fits can be found in Table \ref{tab:FitResultsLamK_3Res}.
Figures with a wider range in \kstar, showing better the non-femtoscopic background, may be found in Appendix \ref{App_Results}.
Also contained in Appendix \ref{App_Results} are plots demonstrating the contributions from the residuals, as well as results assuming 10 and no residual contributors.

In Figures \ref{fig:LamKchPwConjFits_3Res} - \ref{fig:LamK0wConjFits_3Res}, the pair system (e.g. \LamKchP) data is shown in the left column, and the conjugate pair system (e.g. \ALamKchM) in the right
The rows differentiate the different centrality bins (0-10\% in the top, 10-30\% in the middle, and 30-50\% in the bottom).
The lines on the data represent the statistical errors, while the boxes represent the systematic errors.  
The fit procedure is as described in the text; in short, all systems are fit simultaneously with shared radii, while each [\LamKchP, \LamKchM, \LamKs] maintains a unique set of scattering parameters.
The black solid line represents the primary \LamK component of the fit.  
The green line shows the fit to the non-flat background.
The purple points show the fit after all residuals' contributions have been included, and momentum resolution and non-flat background corrections have been applied.
The extracted fit values with uncertainties are printed in the top left panel of each figure.

\begin{figure}[h]
  \centering
  \includegraphics[width=\textwidth]{\ResultsDirBaseLamKch\SaveNameModLamKch/canKStarCfwFitsLamKchPwConj_0010_1030_3050\SaveNameModLamKch.pdf}
  \caption[\LamKchPALamKchM data with fits]
  {
  Fit results, with 3 residual correlations included, for the \LamKchP and \ALamKchM data.
  The \LamKchP data is shown in the left column, the \ALamKchM in the right, and the rows differentiate the different centrality bins (0-10\% in the top, 10-30\% in the middle, and 30-50\% in the bottom).
  See text for further details.
  }
  \label{fig:LamKchPwConjFits_3Res}
\end{figure}



\begin{figure}[h!]
  \centering
  \includegraphics[width=\linewidth]{\ResultsDirBaseLamKch\SaveNameModLamKch/canKStarCfwFitsLamKchMwConj_0010_1030_3050\SaveNameModLamKch.pdf}
  \caption[\LamKchMALamKchP data with fits]
  {
  Fit results, with 3 residual correlations included, for the \LamKchM and \ALamKchP data.
  The \LamKchM data is shown in the left column, the \ALamKchP in the right, and the rows differentiate the different centrality bins (0-10\% in the top, 10-30\% in the middle, and 30-50\% in the bottom).
 See text for further details.
 }
  \label{fig:LamKchMwConjFits_3Res}
\end{figure}


\begin{figure}[h!]
  \centering
  \includegraphics[width=\linewidth]{\ResultsDirBaseLamKs\SaveNameModLamKs/canKStarCfwFitsLamK0wConj_0010_1030_3050\SaveNameModLamKs.pdf}
  \caption[\LamALamKs data with fits]
  {
  Fit results, with 3 residual correlations included, for the \LamKs and \ALamKs data.
  The \LamKs data is shown in the left column, the \ALamKs in the right, and the rows differentiate the different centrality bins (0-10\% in the top, 10-30\% in the middle, and 30-50\% in the bottom).
 See text for further details.
 }
  \label{fig:LamK0wConjFits_3Res}
\end{figure}

\clearpage

\end{document}