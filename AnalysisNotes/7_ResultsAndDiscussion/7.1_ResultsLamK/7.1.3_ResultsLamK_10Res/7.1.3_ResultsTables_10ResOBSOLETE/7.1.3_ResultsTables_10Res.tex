%%% NOTE: If I want to run this standalone, uncomment out lines below
%%%       But lines must be commented out to run within larger project
\begin{comment}
\makeatletter
\def\input@path{{/home/jesse/Analysis/FemtoAnalysis/AnalysisNotes/}}
\makeatother

\documentclass[ALICE,manyauthors]{ALICE_analysis_notes}

\usepackage{MyStyle}
\usepackage{chngpage}  % for adjustwidth
\usepackage{boldline}  % to make lines in table bold
                       % V{<factor>} vertical rule in \begin{tabular} command
                       % also \clinB{<spec>}{<factor>} and \hlineB{<factor>}
\usepackage{arrayjobx} % To use the array structures stored in FitResults_cLamcKch_20180505.tex   
\end{comment}  

%NOTE: Arrays seem to only be able to be named with alphabetical characters
%      No '_' or event numbers

% ----------Shorthand definitions----------
% A = FitGen_NoLamShare
% B = FitGen_ShareLamConj

% C = Dualie_NoLamShare
% D = Dualie_ShareLamConj
% E = Dualie_ShareSingleLam

%   ------------------
% a = PolyNFB
% b = LinrNFB
% c = LinrNFB_StavCf
% d = NoNFB_StavCf

% ----------How arrays are stored----------
% aArray = {lam0010(1), lam0010StatErr(2), lam0010SysErr(3)
%           R0010(4),   R0010StatErr(5),   R0010SysErr(6)
%           lam1030(7), lam1030StatErr(8), lam1030SysErr(9)
%           R1030(10),  R1030StatErr(11),  R1030SysErr(12)
%           lam3050(13), lam3050StatErr(14), lam3050SysErr(15)
%           R3050(16),   R3050StatErr(17),   R3050SysErr(18)
%           ReF0(19),   ReF0StatErr(20),   ReF0SysErr(21)
%           ImF0(22),   ImF0StatErr(23),   ImF0SysErr(24)
%           d0(25),     d0StatErr(26),     d0SysErr(27)}


%%%%%%%%%%%%%%%%%%%%%%%%%%%%%%%%%%%%%%%%%%%%%%%%%%%%%%%%%%%%%%%%%%%%%%%%%%%%%%%%%%%%%%%%%%%%%%%%%%%%%

% --------------- Aa = FitGen_PolyNFB_NoLamShare_10Res ---------------
\newarray\AaLamKchP
\readarray{AaLamKchP}{
                       1.35 & 0.46 & 0.28 & 
                       5.40 & 1.00 & 0.54 & 
                       1.53 & 0.54 & 0.36 & 
                       5.05 & 0.83 & 0.42 & 
                       1.18 & 0.32 & 0.31 & 
                       3.44 & 0.46 & 0.32 & 
                       -1.37 & 0.29 & 0.36 & 
                       0.47 & 0.30 & 0.23 & 
                       0.97 & 0.50 & 0.53}

\newarray\AaALamKchM
\readarray{AaALamKchM}{
                       1.37 & 0.47 & 0.33 & 
                       5.40 & 1.00 & 0.54 & 
                       1.37 & 0.48 & 0.29 & 
                       5.05 & 0.83 & 0.42 & 
                       1.07 & 0.29 & 0.19 & 
                       3.44 & 0.46 & 0.32 & 
                       -1.37 & 0.29 & 0.36 & 
                       0.47 & 0.30 & 0.23 & 
                       0.97 & 0.50 & 0.53}

\newarray\AaLamKchM
\readarray{AaLamKchM}{
                       2.06 & 0.61 & 0.24 & 
                       6.88 & 1.09 & 0.81 & 
                       1.24 & 0.38 & 0.27 & 
                       4.72 & 0.82 & 0.60 & 
                       1.58 & 0.84 & 0.57 & 
                       3.14 & 0.63 & 0.38 & 
                       0.48 & 0.17 & 0.14 & 
                       0.47 & 0.13 & 0.11 & 
                       -4.46 & 1.78 & 1.33}

\newarray\AaALamKchP
\readarray{AaALamKchP}{
                       2.07 & 0.58 & 0.27 & 
                       6.88 & 1.09 & 0.81 & 
                       1.28 & 0.39 & 0.26 & 
                       4.72 & 0.82 & 0.60 & 
                       0.93 & 0.32 & 0.37 & 
                       3.14 & 0.63 & 0.38 & 
                       0.48 & 0.17 & 0.14 & 
                       0.47 & 0.13 & 0.11 & 
                       -4.46 & 1.78 & 1.33}

% --------------- Ba = FitGen_PolyNFB_ShareLamConj_10Res ---------------
\newarray\BaLamKchP
\readarray{BaLamKchP}{
                       1.35 & 0.55 & 0.28 & 
                       5.39 & 1.17 & 0.54 & 
                       1.40 & 0.47 & 0.36 & 
                       4.96 & 0.80 & 0.42 & 
                       1.12 & 0.30 & 0.31 & 
                       3.43 & 0.45 & 0.32 & 
                       -1.37 & 0.29 & 0.36 & 
                       0.46 & 0.31 & 0.23 & 
                       0.99 & 0.52 & 0.53}

\newarray\BaALamKchM
\readarray{BaALamKchM}{
                       1.35 & 0.55 & 0.33 & 
                       5.39 & 1.17 & 0.54 & 
                       1.40 & 0.47 & 0.29 & 
                       4.96 & 0.80 & 0.42 & 
                       1.12 & 0.30 & 0.19 & 
                       3.43 & 0.45 & 0.32 & 
                       -1.37 & 0.29 & 0.36 & 
                       0.46 & 0.31 & 0.23 & 
                       0.99 & 0.52 & 0.53}

\newarray\BaLamKchM
\readarray{BaLamKchM}{
                       1.72 & 0.44 & 0.24 & 
                       3.42 & 0.34 & 0.81 & 
                       0.93 & 0.32 & 0.27 & 
                       2.34 & 0.36 & 0.60 & 
                       0.39 & 0.21 & 0.57 & 
                       1.13 & 0.33 & 0.38 & 
                       -0.04 & 0.02 & 0.14 & 
                       0.09 & 0.04 & 0.11 & 
                       29.55 & 17.10 & 1.33}

\newarray\BaALamKchP
\readarray{BaALamKchP}{
                       1.72 & 0.44 & 0.27 & 
                       3.42 & 0.34 & 0.81 & 
                       0.93 & 0.32 & 0.26 & 
                       2.34 & 0.36 & 0.60 & 
                       0.39 & 0.21 & 0.37 & 
                       1.13 & 0.33 & 0.38 & 
                       -0.04 & 0.02 & 0.14 & 
                       0.09 & 0.04 & 0.11 & 
                       29.55 & 17.10 & 1.33}

% --------------- Ca = Dualie_PolyNFB_NoLamShare_10Res ---------------
\newarray\CaLamKchP
\readarray{CaLamKchP}{
                       1.90 & 0.46 & 0.28 & 
                       6.63 & 0.78 & 0.54 & 
                       1.36 & 0.35 & 0.36 & 
                       4.97 & 0.57 & 0.42 & 
                       1.06 & 0.26 & 0.31 & 
                       3.37 & 0.37 & 0.32 & 
                       -1.34 & 0.25 & 0.36 & 
                       0.64 & 0.30 & 0.23 & 
                       0.86 & 0.54 & 0.53}

\newarray\CaALamKchM
\readarray{CaALamKchM}{
                       1.93 & 0.48 & 0.33 & 
                       6.63 & 0.78 & 0.54 & 
                       1.23 & 0.32 & 0.29 & 
                       4.97 & 0.57 & 0.42 & 
                       0.97 & 0.24 & 0.19 & 
                       3.37 & 0.37 & 0.32 & 
                       -1.34 & 0.25 & 0.36 & 
                       0.64 & 0.30 & 0.23 & 
                       0.86 & 0.54 & 0.53}

\newarray\CaLamKchM
\readarray{CaLamKchM}{
                       1.92 & 0.45 & 0.24 & 
                       6.63 & 0.78 & 0.81 & 
                       1.31 & 0.30 & 0.27 & 
                       4.97 & 0.57 & 0.60 & 
                       1.78 & 0.65 & 0.57 & 
                       3.37 & 0.37 & 0.38 & 
                       0.52 & 0.15 & 0.14 & 
                       0.49 & 0.12 & 0.11 & 
                       -4.05 & 1.48 & 1.33}

\newarray\CaALamKchP
\readarray{CaALamKchP}{
                       1.94 & 0.43 & 0.27 & 
                       6.63 & 0.78 & 0.81 & 
                       1.34 & 0.30 & 0.26 & 
                       4.97 & 0.57 & 0.60 & 
                       0.98 & 0.25 & 0.37 & 
                       3.37 & 0.37 & 0.38 & 
                       0.52 & 0.15 & 0.14 & 
                       0.49 & 0.12 & 0.11 & 
                       -4.05 & 1.48 & 1.33}

% --------------- Da = Dualie_PolyNFB_ShareLamConj_10Res ---------------
\newarray\DaLamKchP
\readarray{DaLamKchP}{
                       1.82 & 0.46 & 0.28 & 
                       6.22 & 0.69 & 0.54 & 
                       1.26 & 0.33 & 0.36 & 
                       4.72 & 0.52 & 0.42 & 
                       0.93 & 0.23 & 0.31 & 
                       3.14 & 0.32 & 0.32 & 
                       -1.30 & 0.28 & 0.36 & 
                       0.52 & 0.26 & 0.23 & 
                       0.82 & 0.52 & 0.53}

\newarray\DaALamKchM
\readarray{DaALamKchM}{
                       1.82 & 0.46 & 0.33 & 
                       6.22 & 0.69 & 0.54 & 
                       1.26 & 0.33 & 0.29 & 
                       4.72 & 0.52 & 0.42 & 
                       0.93 & 0.23 & 0.19 & 
                       3.14 & 0.32 & 0.32 & 
                       -1.30 & 0.28 & 0.36 & 
                       0.52 & 0.26 & 0.23 & 
                       0.82 & 0.52 & 0.53}

\newarray\DaLamKchM
\readarray{DaLamKchM}{
                       1.71 & 0.36 & 0.24 & 
                       6.22 & 0.69 & 0.81 & 
                       1.19 & 0.24 & 0.27 & 
                       4.72 & 0.52 & 0.60 & 
                       1.01 & 0.22 & 0.57 & 
                       3.14 & 0.32 & 0.38 & 
                       0.50 & 0.14 & 0.14 & 
                       0.53 & 0.12 & 0.11 & 
                       -3.74 & 1.31 & 1.33}

\newarray\DaALamKchP
\readarray{DaALamKchP}{
                       1.71 & 0.36 & 0.27 & 
                       6.22 & 0.69 & 0.81 & 
                       1.19 & 0.24 & 0.26 & 
                       4.72 & 0.52 & 0.60 & 
                       1.01 & 0.22 & 0.37 & 
                       3.14 & 0.32 & 0.38 & 
                       0.50 & 0.14 & 0.14 & 
                       0.53 & 0.12 & 0.11 & 
                       -3.74 & 1.31 & 1.33}

% --------------- Ea = Dualie_PolyNFB_ShareSingleLam_10Res ---------------
\newarray\EaLamKchP
\readarray{EaLamKchP}{
                       1.70 & 0.32 & 0.28 & 
                       6.18 & 0.63 & 0.54 & 
                       1.19 & 0.22 & 0.36 & 
                       4.72 & 0.45 & 0.42 & 
                       0.97 & 0.19 & 0.31 & 
                       3.25 & 0.30 & 0.32 & 
                       -1.34 & 0.17 & 0.36 & 
                       0.58 & 0.17 & 0.23 & 
                       0.86 & 0.42 & 0.53}

\newarray\EaALamKchM
\readarray{EaALamKchM}{
                       1.70 & 0.32 & 0.33 & 
                       6.18 & 0.63 & 0.54 & 
                       1.19 & 0.22 & 0.29 & 
                       4.72 & 0.45 & 0.42 & 
                       0.97 & 0.19 & 0.19 & 
                       3.25 & 0.30 & 0.32 & 
                       -1.34 & 0.17 & 0.36 & 
                       0.58 & 0.17 & 0.23 & 
                       0.86 & 0.42 & 0.53}

\newarray\EaLamKchM
\readarray{EaLamKchM}{
                       1.70 & 0.32 & 0.24 & 
                       6.18 & 0.63 & 0.81 & 
                       1.19 & 0.22 & 0.27 & 
                       4.72 & 0.45 & 0.60 & 
                       0.97 & 0.19 & 0.57 & 
                       3.25 & 0.30 & 0.38 & 
                       0.50 & 0.14 & 0.14 & 
                       0.53 & 0.11 & 0.11 & 
                       -3.72 & 1.28 & 1.33}

\newarray\EaALamKchP
\readarray{EaALamKchP}{
                       1.70 & 0.32 & 0.27 & 
                       6.18 & 0.63 & 0.81 & 
                       1.19 & 0.22 & 0.26 & 
                       4.72 & 0.45 & 0.60 & 
                       0.97 & 0.19 & 0.37 & 
                       3.25 & 0.30 & 0.38 & 
                       0.50 & 0.14 & 0.14 & 
                       0.53 & 0.11 & 0.11 & 
                       -3.72 & 1.28 & 1.33}

%%%%%%%%%%%%%%%%%%%%%%%%%%%%%%%%%%%%%%%%%%%%%%%%%%%%%%%%%%%%%%%%%%%%%%%%%%%%%%%%%%%%%%%%%%%%%%%%%%%%%

% --------------- Ab = FitGen_LinrNFB_NoLamShare_10Res ---------------
\newarray\AbLamKchP
\readarray{AbLamKchP}{
                       1.33 & 0.48 & 0.28 & 
                       5.67 & 0.95 & 0.54 & 
                       1.52 & 0.58 & 0.36 & 
                       5.15 & 0.89 & 0.42 & 
                       1.08 & 0.29 & 0.31 & 
                       3.44 & 0.46 & 0.32 & 
                       -1.46 & 0.35 & 0.36 & 
                       0.62 & 0.33 & 0.23 & 
                       1.08 & 0.75 & 0.53}

\newarray\AbALamKchM
\readarray{AbALamKchM}{
                       1.34 & 0.49 & 0.33 & 
                       5.67 & 0.95 & 0.54 & 
                       1.33 & 0.50 & 0.29 & 
                       5.15 & 0.89 & 0.42 & 
                       1.05 & 0.28 & 0.19 & 
                       3.44 & 0.46 & 0.32 & 
                       -1.46 & 0.35 & 0.36 & 
                       0.62 & 0.33 & 0.23 & 
                       1.08 & 0.75 & 0.53}

\newarray\AbLamKchM
\readarray{AbLamKchM}{
                       1.89 & 0.66 & 0.24 & 
                       6.91 & 1.24 & 0.81 & 
                       1.30 & 0.44 & 0.27 & 
                       4.91 & 0.90 & 0.60 & 
                       1.66 & 0.92 & 0.57 & 
                       3.27 & 0.65 & 0.38 & 
                       0.52 & 0.19 & 0.14 & 
                       0.53 & 0.16 & 0.11 & 
                       -4.44 & 1.83 & 1.33}

\newarray\AbALamKchP
\readarray{AbALamKchP}{
                       1.91 & 0.63 & 0.27 & 
                       6.91 & 1.24 & 0.81 & 
                       1.33 & 0.45 & 0.26 & 
                       4.91 & 0.90 & 0.60 & 
                       0.93 & 0.33 & 0.37 & 
                       3.27 & 0.65 & 0.38 & 
                       0.52 & 0.19 & 0.14 & 
                       0.53 & 0.16 & 0.11 & 
                       -4.44 & 1.83 & 1.33}

% --------------- Bb = FitGen_LinrNFB_ShareLamConj_10Res ---------------
\newarray\BbLamKchP
\readarray{BbLamKchP}{
                       1.32 & 0.46 & 0.28 & 
                       5.64 & 0.92 & 0.54 & 
                       1.35 & 0.48 & 0.36 & 
                       5.03 & 0.82 & 0.42 & 
                       1.06 & 0.27 & 0.31 & 
                       3.44 & 0.46 & 0.32 & 
                       -1.48 & 0.34 & 0.36 & 
                       0.62 & 0.34 & 0.23 & 
                       1.11 & 0.72 & 0.53}

\newarray\BbALamKchM
\readarray{BbALamKchM}{
                       1.32 & 0.46 & 0.33 & 
                       5.64 & 0.92 & 0.54 & 
                       1.35 & 0.48 & 0.29 & 
                       5.03 & 0.82 & 0.42 & 
                       1.06 & 0.27 & 0.19 & 
                       3.44 & 0.46 & 0.32 & 
                       -1.48 & 0.34 & 0.36 & 
                       0.62 & 0.34 & 0.23 & 
                       1.11 & 0.72 & 0.53}

\newarray\BbLamKchM
\readarray{BbLamKchM}{
                       1.46 & 0.57 & 0.24 & 
                       5.95 & 1.19 & 0.81 & 
                       1.02 & 0.36 & 0.27 & 
                       4.29 & 0.81 & 0.60 & 
                       0.79 & 0.29 & 0.57 & 
                       2.69 & 0.54 & 0.38 & 
                       0.44 & 0.20 & 0.14 & 
                       0.58 & 0.20 & 0.11 & 
                       -4.12 & 1.94 & 1.33}

\newarray\BbALamKchP
\readarray{BbALamKchP}{
                       1.46 & 0.57 & 0.27 & 
                       5.95 & 1.19 & 0.81 & 
                       1.02 & 0.36 & 0.26 & 
                       4.29 & 0.81 & 0.60 & 
                       0.79 & 0.29 & 0.37 & 
                       2.69 & 0.54 & 0.38 & 
                       0.44 & 0.20 & 0.14 & 
                       0.58 & 0.20 & 0.11 & 
                       -4.12 & 1.94 & 1.33}

% --------------- Cb = Dualie_LinrNFB_NoLamShare_10Res ---------------
\newarray\CbLamKchP
\readarray{CbLamKchP}{
                       1.76 & 0.46 & 0.28 & 
                       6.58 & 0.77 & 0.54 & 
                       1.42 & 0.39 & 0.36 & 
                       5.04 & 0.60 & 0.42 & 
                       1.03 & 0.26 & 0.31 & 
                       3.38 & 0.36 & 0.32 & 
                       -1.39 & 0.31 & 0.36 & 
                       0.73 & 0.33 & 0.23 & 
                       0.95 & 0.73 & 0.53}

\newarray\CbALamKchM
\readarray{CbALamKchM}{
                       1.79 & 0.47 & 0.33 & 
                       6.58 & 0.77 & 0.54 & 
                       1.25 & 0.34 & 0.29 & 
                       5.04 & 0.60 & 0.42 & 
                       1.00 & 0.25 & 0.19 & 
                       3.38 & 0.36 & 0.32 & 
                       -1.39 & 0.31 & 0.36 & 
                       0.73 & 0.33 & 0.23 & 
                       0.95 & 0.73 & 0.53}

\newarray\CbLamKchM
\readarray{CbLamKchM}{
                       1.73 & 0.41 & 0.24 & 
                       6.58 & 0.77 & 0.81 & 
                       1.32 & 0.30 & 0.27 & 
                       5.04 & 0.60 & 0.60 & 
                       1.71 & 0.60 & 0.57 & 
                       3.38 & 0.36 & 0.38 & 
                       0.55 & 0.16 & 0.14 & 
                       0.56 & 0.15 & 0.11 & 
                       -4.07 & 1.55 & 1.33}

\newarray\CbALamKchP
\readarray{CbALamKchP}{
                       1.76 & 0.40 & 0.27 & 
                       6.58 & 0.77 & 0.81 & 
                       1.35 & 0.31 & 0.26 & 
                       5.04 & 0.60 & 0.60 & 
                       0.94 & 0.24 & 0.37 & 
                       3.38 & 0.36 & 0.38 & 
                       0.55 & 0.16 & 0.14 & 
                       0.56 & 0.15 & 0.11 & 
                       -4.07 & 1.55 & 1.33}

% --------------- Db = Dualie_LinrNFB_ShareLamConj_10Res ---------------
\newarray\DbLamKchP
\readarray{DbLamKchP}{
                       1.66 & 0.47 & 0.28 & 
                       6.14 & 0.74 & 0.54 & 
                       1.27 & 0.36 & 0.36 & 
                       4.74 & 0.54 & 0.42 & 
                       0.93 & 0.23 & 0.31 & 
                       3.15 & 0.32 & 0.32 & 
                       -1.37 & 0.33 & 0.36 & 
                       0.60 & 0.29 & 0.23 & 
                       0.86 & 0.69 & 0.53}

\newarray\DbALamKchM
\readarray{DbALamKchM}{
                       1.66 & 0.47 & 0.33 & 
                       6.14 & 0.74 & 0.54 & 
                       1.27 & 0.36 & 0.29 & 
                       4.74 & 0.54 & 0.42 & 
                       0.93 & 0.23 & 0.19 & 
                       3.15 & 0.32 & 0.32 & 
                       -1.37 & 0.33 & 0.36 & 
                       0.60 & 0.29 & 0.23 & 
                       0.86 & 0.69 & 0.53}

\newarray\DbLamKchM
\readarray{DbLamKchM}{
                       1.51 & 0.36 & 0.24 & 
                       6.14 & 0.74 & 0.81 & 
                       1.17 & 0.25 & 0.27 & 
                       4.74 & 0.54 & 0.60 & 
                       0.97 & 0.22 & 0.57 & 
                       3.15 & 0.32 & 0.38 & 
                       0.54 & 0.16 & 0.14 & 
                       0.62 & 0.15 & 0.11 & 
                       -3.67 & 1.39 & 1.33}

\newarray\DbALamKchP
\readarray{DbALamKchP}{
                       1.51 & 0.36 & 0.27 & 
                       6.14 & 0.74 & 0.81 & 
                       1.17 & 0.25 & 0.26 & 
                       4.74 & 0.54 & 0.60 & 
                       0.97 & 0.22 & 0.37 & 
                       3.15 & 0.32 & 0.38 & 
                       0.54 & 0.16 & 0.14 & 
                       0.62 & 0.15 & 0.11 & 
                       -3.67 & 1.39 & 1.33}

% --------------- Eb = Dualie_LinrNFB_ShareSingleLam_10Res ---------------
\newarray\EbLamKchP
\readarray{EbLamKchP}{
                       1.52 & 0.31 & 0.28 & 
                       6.11 & 0.65 & 0.54 & 
                       1.19 & 0.23 & 0.36 & 
                       4.76 & 0.47 & 0.42 & 
                       0.95 & 0.18 & 0.31 & 
                       3.26 & 0.30 & 0.32 & 
                       -1.44 & 0.21 & 0.36 & 
                       0.70 & 0.20 & 0.23 & 
                       1.00 & 0.48 & 0.53}

\newarray\EbALamKchM
\readarray{EbALamKchM}{
                       1.52 & 0.31 & 0.33 & 
                       6.11 & 0.65 & 0.54 & 
                       1.19 & 0.23 & 0.29 & 
                       4.76 & 0.47 & 0.42 & 
                       0.95 & 0.18 & 0.19 & 
                       3.26 & 0.30 & 0.32 & 
                       -1.44 & 0.21 & 0.36 & 
                       0.70 & 0.20 & 0.23 & 
                       1.00 & 0.48 & 0.53}

\newarray\EbLamKchM
\readarray{EbLamKchM}{
                       1.52 & 0.31 & 0.24 & 
                       6.11 & 0.65 & 0.81 & 
                       1.19 & 0.23 & 0.27 & 
                       4.76 & 0.47 & 0.60 & 
                       0.95 & 0.18 & 0.57 & 
                       3.26 & 0.30 & 0.38 & 
                       0.54 & 0.16 & 0.14 & 
                       0.61 & 0.13 & 0.11 & 
                       -3.69 & 1.33 & 1.33}

\newarray\EbALamKchP
\readarray{EbALamKchP}{
                       1.52 & 0.31 & 0.27 & 
                       6.11 & 0.65 & 0.81 & 
                       1.19 & 0.23 & 0.26 & 
                       4.76 & 0.47 & 0.60 & 
                       0.95 & 0.18 & 0.37 & 
                       3.26 & 0.30 & 0.38 & 
                       0.54 & 0.16 & 0.14 & 
                       0.61 & 0.13 & 0.11 & 
                       -3.69 & 1.33 & 1.33}

%%%%%%%%%%%%%%%%%%%%%%%%%%%%%%%%%%%%%%%%%%%%%%%%%%%%%%%%%%%%%%%%%%%%%%%%%%%%%%%%%%%%%%%%%%%%%%%%%%%%%


%NOTE: Arrays seem to only be able to be named with alphabetical characters
%      No '_' or event numbers

% ----------Shorthand definitions----------
% A = FitGen_NoLamShare
% B = FitGen_ShareLamConj

% C = Dualie_NoLamShare
% D = Dualie_ShareLamConj
% E = Dualie_ShareSingleLam

%   ------------------
% a = PolyNFB
% b = LinrNFB
% c = LinrNFB_StavCf
% d = NoNFB_StavCf

% ----------How arrays are stored----------
% aArray = {lam0010(1), lam0010StatErr(2), lam0010SysErr(3)
%           R0010(4),   R0010StatErr(5),   R0010SysErr(6)
%           lam1030(7), lam1030StatErr(8), lam1030SysErr(9)
%           R1030(10),  R1030StatErr(11),  R1030SysErr(12)
%           lam3050(13), lam3050StatErr(14), lam3050SysErr(15)
%           R3050(16),   R3050StatErr(17),   R3050SysErr(18)
%           ReF0(19),   ReF0StatErr(20),   ReF0SysErr(21)
%           ImF0(22),   ImF0StatErr(23),   ImF0SysErr(24)
%           d0(25),     d0StatErr(26),     d0SysErr(27)}


%%%%%%%%%%%%%%%%%%%%%%%%%%%%%%%%%%%%%%%%%%%%%%%%%%%%%%%%%%%%%%%%%%%%%%%%%%%%%%%%%%%%%%%%%%%%%%%%%%%%%

% --------------- Aa = FitGen_PolyNFB_NoLamShare_10Res ---------------
\newarray\AaLamKs
\readarray{AaLamKs}{
                       0.60 & 0.86 & 0.16 & 
                       2.89 & 0.48 & 0.33 & 
                       0.60 & 0.86 & 0.16 & 
                       2.59 & 0.46 & 0.23 & 
                       0.60 & 0.86 & 0.16 & 
                       1.93 & 0.33 & 0.11 & 
                       -0.39 & 0.12 & 0.16 & 
                       0.13 & 0.08 & 0.13 & 
                       1.38 & 1.06 & 0.62}

\newarray\AaALamKs
\readarray{AaALamKs}{
                       0.60 & 0.86 & 0.16 & 
                       2.89 & 0.48 & 0.33 & 
                       0.60 & 0.86 & 0.16 & 
                       2.59 & 0.46 & 0.23 & 
                       0.60 & 0.86 & 0.16 & 
                       1.93 & 0.33 & 0.11 & 
                       -0.39 & 0.12 & 0.16 & 
                       0.13 & 0.08 & 0.13 & 
                       1.38 & 1.06 & 0.62}

%%%%%%%%%%%%%%%%%%%%%%%%%%%%%%%%%%%%%%%%%%%%%%%%%%%%%%%%%%%%%%%%%%%%%%%%%%%%%%%%%%%%%%%%%%%%%%%%%%%%%

% --------------- Ab = FitGen_LinrNFB_NoLamShare_10Res ---------------
\newarray\AbLamKs
\readarray{AbLamKs}{
                       0.60 & 0.70 & 0.16 & 
                       2.97 & 0.54 & 0.33 & 
                       0.60 & 0.70 & 0.16 & 
                       2.41 & 0.45 & 0.23 & 
                       0.60 & 0.70 & 0.16 & 
                       1.78 & 0.32 & 0.11 & 
                       -0.37 & 0.11 & 0.16 & 
                       0.19 & 0.11 & 0.13 & 
                       1.79 & 0.57 & 0.62}

\newarray\AbALamKs
\readarray{AbALamKs}{
                       0.60 & 0.70 & 0.16 & 
                       2.97 & 0.54 & 0.33 & 
                       0.60 & 0.70 & 0.16 & 
                       2.41 & 0.45 & 0.23 & 
                       0.60 & 0.70 & 0.16 & 
                       1.78 & 0.32 & 0.11 & 
                       -0.37 & 0.11 & 0.16 & 
                       0.19 & 0.11 & 0.13 & 
                       1.79 & 0.57 & 0.62}

%%%%%%%%%%%%%%%%%%%%%%%%%%%%%%%%%%%%%%%%%%%%%%%%%%%%%%%%%%%%%%%%%%%%%%%%%%%%%%%%%%%%%%%%%%%%%%%%%%%%%




\begin{document}
\pagestyle{empty}

\pagestyle{empty}
\begin{landscape}


%%%%%%%%%%%%%%%%%%%%%%%%%%%%%%%%%%%%%%%%%%%%%%%%%%%%%%%%%%%%%%%%%%%%%%%%%%%%%%%%%%%%%%%%%%%%%%%%%%%%%%%%%%%%%%%%%%%%%%%%%%%%%%%%%
%%%%%%%%%%%%%%%%%%%%%%%%%%%%%%%%%%%%%%%%%          LamK0             %%%%%%%%%%%%%%%%%%%%%%%%%%%%%%%%%%%%%%%%%%%%%%%%%%%%%%%%%%%
%%%%%%%%%%%%%%%%%%%%%%%%%%%%%%%%%%%%%%%%%%%%%%%%%%%%%%%%%%%%%%%%%%%%%%%%%%%%%%%%%%%%%%%%%%%%%%%%%%%%%%%%%%%%%%%%%%%%%%%%%%%%%%%%%
%%%%%%%%%%%%%%%%%%%% LamK0, polynomial background, single lambda parameter
\begin{table}[htbp]
 \centering
 \renewcommand{\arraystretch}{1.25}
 \resizebox{\paperwidth}{!}{
 \begin{tabular}{|c|c|c|c|c|c|c|}
  \multicolumn{7}{c}{Fit Results \LamALamKs} \\
  \hline
  \multirow{2}{*}{System} & \multirow{2}{*}{Centrality} & \multicolumn{5}{c|}{Fit Parameters} \\
  \cline{3-7}
   & & $\lambda$ & $R$ & $\mathbb{R}f_{0}$ & $\mathbb{I}f_{0}$ & $d_{0}$ \\
  \hline  
  \multirow{3}{*}{\LamKs \& \ALamKs}  
     & 0-10\%
     & \multirow{3}{*}{\AaLamKs(1) $\pm$ \AaLamKs(2) (stat.) $\pm$ \AaLamKs(3) (sys.)}    %Lambda (LamK0 & ALamK0 AllCent)
     & \AaLamKs(4) $\pm$ \AaLamKs(5) (stat.) $\pm$ \AaLamKs(6) (sys.)                     %Radius (LamK0 & ALamK0 0010)
     & \multirow{3}{*}{\AaLamKs(19) $\pm$ \AaLamKs(20) (stat.) $\pm$ \AaLamKs(21) (sys.)}   %Ref0   (LamK0 & ALamK0)
     & \multirow{3}{*}{\AaLamKs(22) $\pm$ \AaLamKs(23) (stat.) $\pm$ \AaLamKs(24) (sys.)}    %Imf0   (LamK0 & ALamK0)
     & \multirow{3}{*}{\AaLamKs(25) $\pm$ \AaLamKs(26) (stat.) $\pm$ \AaLamKs(27) (sys.)} \\ %d0     (LamK0 & ALamK0)
   
     & 10-30\%
     & & \AaLamKs(10) $\pm$ \AaLamKs(11) (stat.) $\pm$ \AaLamKs(12) (sys.)                   %Radius (LamK0 & ALamK0 1030)
     & & & \\
   
     & 30-50\%
     & & \AaLamKs(16) $\pm$ \AaLamKs(17) (stat.) $\pm$ \AaLamKs(18) (sys.)                   %Radius (LamK0 & ALamK0 3050)
     & & & \\
   \hline
 \end{tabular}}
 \caption[Fit Results \LamALamKs, with 10 residual correlations included]{Fit Results \LamALamKs, with 10 residual correlations included. 
 Each pair is fit simultaneously with its conjugate (ie. \LamKs with \ALamKs) across all centralities (0-10\%, 10-30\%, 30-50\%), for a total of 6 simultaneous analyses in the fit.
 A single $\lambda$ parameter is shared amongst all.
 Each analysis has a unique normalization parameter.
 The radii are shared between analyses of like centrality, as these should have similar source sizes.
 The scattering parameters ($\mathbb{R}f_{0}$, $\mathbb{I}f_{0}$, $d_{0}$) are shared amongst all.
 The background is modeled by a (6$^{\mathrm{th}}$-)degree polynomial fit to THERMINATOR simulation.
 The fit is done on the data with only statistical error bars.
 The errors marked as ``stat." are those returned by MINUIT.
 The errors marked as ``sys." are those which result from my systematic analysis (as outlined in Section \ref{SystematicErrors}).}
 \label{tab:FitResultsLamK0_10Res}
\end{table}  


%%%%%%%%%%%%%%%%%%%%%%%%%%%%%%%%%%%%%%%%%%%%%%%%%%%%%%%%%%%%%%%%%%%%%%%%%%%%%%%%%%%%%%%%%%%%%%%%%%%%%%%%%%%%%%%%%%%%%%%%%%%%%%%%%
%%%%%%%%%%%%%%%%%%%% LamK0, linear background, single lambda parameter
\begin{comment}
\begin{table}[htbp]
 \centering
 \renewcommand{\arraystretch}{1.25}
 \resizebox{\paperwidth}{!}{
 \begin{tabular}{|c|c|c|c|c|c|c|}
  \multicolumn{7}{c}{Fit Results \LamALamKs} \\
  \hline
  \multirow{2}{*}{System} & \multirow{2}{*}{Centrality} & \multicolumn{5}{c|}{Fit Parameters} \\
  \cline{3-7}
   & & $\lambda$ & $R$ & $\mathbb{R}f_{0}$ & $\mathbb{I}f_{0}$ & $d_{0}$ \\
  \hline  
  \multirow{3}{*}{\LamKs \& \ALamKs}  
     & 0-10\%
     & \multirow{3}{*}{\AbLamKs(1) $\pm$ \AbLamKs(2) (stat.) $\pm$ \AbLamKs(3) (sys.)}    %Lambda (LamK0 & ALamK0 AllCent)
     & \AbLamKs(4) $\pm$ \AbLamKs(5) (stat.) $\pm$ \AbLamKs(6) (sys.)                     %Radius (LamK0 & ALamK0 0010)
     & \multirow{3}{*}{\AbLamKs(19) $\pm$ \AbLamKs(20) (stat.) $\pm$ \AbLamKs(21) (sys.)}   %Ref0   (LamK0 & ALamK0)
     & \multirow{3}{*}{\AbLamKs(22) $\pm$ \AbLamKs(23) (stat.) $\pm$ \AbLamKs(24) (sys.)}    %Imf0   (LamK0 & ALamK0)
     & \multirow{3}{*}{\AbLamKs(25) $\pm$ \AbLamKs(26) (stat.) $\pm$ \AbLamKs(27) (sys.)} \\ %d0     (LamK0 & ALamK0)
   
     & 10-30\%
     & & \AbLamKs(10) $\pm$ \AbLamKs(11) (stat.) $\pm$ \AbLamKs(12) (sys.)                   %Radius (LamK0 & ALamK0 1030)
     & & & \\
   
     & 30-50\%
     & & \AbLamKs(16) $\pm$ \AbLamKs(17) (stat.) $\pm$ \AbLamKs(18) (sys.)                   %Radius (LamK0 & ALamK0 3050)
     & & & \\
   \hline
 \end{tabular}}
 \caption[Fit Results \LamALamKs, with 10 residual correlations included]{Fit Results \LamALamKs, with 10 residual correlations included. 
 Each pair is fit simultaneously with its conjugate (ie. \LamKs with \ALamKs) across all centralities (0-10\%, 10-30\%, 30-50\%), for a total of 6 simultaneous analyses in the fit.
 A single $\lambda$ parameter is shared amongst all.
 Each analysis has a unique normalization parameter.
 The radii are shared between analyses of like centrality, as these should have similar source sizes.
 The scattering parameters ($\mathbb{R}f_{0}$, $\mathbb{I}f_{0}$, $d_{0}$) are shared amongst all.
 The background is modeled with a linear form.
 The fit is done on the data with only statistical error bars.
 The errors marked as ``stat." are those returned by MINUIT.
 The errors marked as ``sys." are those which result from my systematic analysis (as outlined in Section \ref{SystematicErrors}).}
 \label{tab:FitResultsLamK0_10Res}
\end{table} 
\end{comment} 

%%%%%%%%%%%%%%%%%%%%%%%%%%%%%%%%%%%%%%%%%%%%%%%%%%%%%%%%%%%%%%%%%%%%%%%%%%%%%%%%%%%%%%%%%%%%%%%%%%%%%%%%%%%%%%%%%%%%%%%%%%%%%%%%%
%%%%%%%%%%%%%%%%%%%% LamK0, Stav method with no bgd, single lambda parameter
\begin{comment}
\begin{table}[htbp]
 \centering
 \renewcommand{\arraystretch}{1.25}
 \resizebox{\paperwidth}{!}{
 \begin{tabular}{|c|c|c|c|c|c|c|}
  \multicolumn{7}{c}{Fit Results \LamALamKs} \\
  \hline
  \multirow{2}{*}{System} & \multirow{2}{*}{Centrality} & \multicolumn{5}{c|}{Fit Parameters} \\
  \cline{3-7}
   & & $\lambda$ & $R$ & $\mathbb{R}f_{0}$ & $\mathbb{I}f_{0}$ & $d_{0}$ \\
  \hline  
  \multirow{3}{*}{\LamKs \& \ALamKs}  
     & 0-10\%
     & \multirow{3}{*}{\AcLamKs(1) $\pm$ \AcLamKs(2) (stat.) $\pm$ \AcLamKs(3) (sys.)}    %Lambda (LamK0 & ALamK0 AllCent)
     & \AcLamKs(4) $\pm$ \AcLamKs(5) (stat.) $\pm$ \AcLamKs(6) (sys.)                     %Radius (LamK0 & ALamK0 0010)
     & \multirow{3}{*}{\AcLamKs(19) $\pm$ \AcLamKs(20) (stat.) $\pm$ \AcLamKs(21) (sys.)}   %Ref0   (LamK0 & ALamK0)
     & \multirow{3}{*}{\AcLamKs(22) $\pm$ \AcLamKs(23) (stat.) $\pm$ \AcLamKs(24) (sys.)}    %Imf0   (LamK0 & ALamK0)
     & \multirow{3}{*}{\AcLamKs(25) $\pm$ \AcLamKs(26) (stat.) $\pm$ \AcLamKs(27) (sys.)} \\ %d0     (LamK0 & ALamK0)
   
     & 10-30\%
     & & \AcLamKs(10) $\pm$ \AcLamKs(11) (stat.) $\pm$ \AcLamKs(12) (sys.)                   %Radius (LamK0 & ALamK0 1030)
     & & & \\
   
     & 30-50\%
     & & \AcLamKs(16) $\pm$ \AcLamKs(17) (stat.) $\pm$ \AcLamKs(18) (sys.)                   %Radius (LamK0 & ALamK0 3050)
     & & & \\
   \hline
 \end{tabular}}
 \caption[Fit Results \LamALamKs, with 10 residual correlations included, using the Stavinsky method]{Fit Results \LamALamKs, with 10 residual correlations included, using the Stavinsky method to reduce the non-femtoscopic background. 
 Each pair is fit simultaneously with its conjugate (ie. \LamKs with \ALamKs) across all centralities (0-10\%, 10-30\%, 30-50\%), for a total of 6 simultaneous analyses in the fit.
 A single $\lambda$ parameter is shared amongst all.
 Each analysis has a unique normalization parameter.
 The radii are shared between analyses of like centrality, as these should have similar source sizes.
 The scattering parameters ($\mathbb{R}f_{0}$, $\mathbb{I}f_{0}$, $d_{0}$) are shared amongst all.
 The background is assumed flat.
 The fit is done on the data with only statistical error bars.
 The errors marked as ``stat." are those returned by MINUIT.
 The errors marked as ``sys." are those which result from my systematic analysis (as outlined in Section \ref{SystematicErrors}).}
 \label{tab:FitResultsLamK0_10Res}
\end{table} 
\end{comment} 

%%%%%%%%%%%%%%%%%%%%%%%%%%%%%%%%%%%%%%%%%%%%%%%%%%%%%%%%%%%%%%%%%%%%%%%%%%%%%%%%%%%%%%%%%%%%%%%%%%%%%%%%%%%%%%%%%%%%%%%%%%%%%%%%%
%%%%%%%%%%%%%%%%%%%%%%%%%%%%%%%%%%%%%%%%%          LamKch             %%%%%%%%%%%%%%%%%%%%%%%%%%%%%%%%%%%%%%%%%%%%%%%%%%%%%%%%%%%
%%%%%%%%%%%%%%%%%%%%%%%%%%%%%%%%%%%%%%%%%%%%%%%%%%%%%%%%%%%%%%%%%%%%%%%%%%%%%%%%%%%%%%%%%%%%%%%%%%%%%%%%%%%%%%%%%%%%%%%%%%%%%%%%%
%%%%%%%%%%%%%%%%%%%% LamKch, polynomial background, unique radii and unique lambda
\begin{comment}
\clearpage
\begin{table}[htbp]
 \centering
 \renewcommand{\arraystretch}{1.25}
 \resizebox{\paperwidth}{!}{
 \begin{tabular}{|c|c|c|c|c|c|c|c|}
  \multicolumn{8}{c}{Fit Results \LamALamKpm} \\
  \hline
  \multirow{2}{*}{System} & \multirow{2}{*}{Centrality} & \multirow{2}{*}{Pair Type} & \multicolumn{5}{c|}{Fit Parameters} \\
  \cline{4-8}
   & & & $\lambda$ & $R$ & $\mathbb{R}f_{0}$ & $\mathbb{I}f_{0}$ & $d_{0}$ \\
  \hline  
  \multirow{6}{*}{\LamKchP \& \ALamKchM}  
   & \multirow{2}{*}{0-10\%} 
     & \LamKchP
     & \AaLamKchP(1) $\pm$ \AaLamKchP(2) (stat.) $\pm$ \AaLamKchP(3) (sys.)                     %Lambda (LamKchP 0010)
     & \multirow{2}{*}{\AaLamKchP(4) $\pm$ \AaLamKchP(5) (stat.) $\pm$ \AaLamKchP(6) (sys.)}    %Radius (LamKchP & ALamKchM 0010)
     & \multirow{6}{*}{\AaLamKchP(19) $\pm$ \AaLamKchP(20) (stat.) $\pm$ \AaLamKchP(21) (sys.)}   %Ref0   (LamKchP & ALamKchM)
     & \multirow{6}{*}{\AaLamKchP(22) $\pm$ \AaLamKchP(23) (stat.) $\pm$ \AaLamKchP(24) (sys.)}    %Imf0   (LamKchP & ALamKchM)
     & \multirow{6}{*}{\AaLamKchP(25) $\pm$ \AaLamKchP(26) (stat.) $\pm$ \AaLamKchP(27) (sys.)} \\ %d0     (LamKchP & ALamKchM)
     
     & & \ALamKchM 
     & \AaALamKchM(1) $\pm$ \AaALamKchM(2) (stat.) $\pm$ \AaALamKchM(3) (sys.)                     %Lambda (ALamKchM 0010)
     & & & & \\          
   \cline{2-5}
   
   & \multirow{2}{*}{10-30\%}
     & \LamKchP
     & \AaLamKchP(7) $\pm$ \AaLamKchP(8) (stat.) $\pm$ \AaLamKchP(9) (sys.)                     %Lambda (LamKchP 1030)
     & \multirow{2}{*}{\AaLamKchP(10) $\pm$ \AaLamKchP(11) (stat.) $\pm$ \AaLamKchP(12) (sys.)}    %Radius (LamKchP & ALamKchM 1030) 
     & & & \\
             
     & & \ALamKchM 
     & \AaALamKchM(7) $\pm$ \AaALamKchM(8) (stat.) $\pm$ \AaALamKchM(9) (sys.)                     %Lambda (ALamKchM 1030)
     & & & & \\  
   \cline{2-5}
   
   & \multirow{2}{*}{30-50\%}
     & \LamKchP
     & \AaLamKchP(13) $\pm$ \AaLamKchP(14) (stat.) $\pm$ \AaLamKchP(15) (sys.)                     %Lambda (LamKchP 3050)
     & \multirow{2}{*}{\AaLamKchP(16) $\pm$ \AaLamKchP(17) (stat.) $\pm$ \AaLamKchP(18) (sys.)}    %Radius (LamKchP & ALamKchM 3050)
     & & & \\
             
     & & \ALamKchM
     & \AaALamKchM(13) $\pm$ \AaALamKchM(14) (stat.) $\pm$ \AaALamKchM(15) (sys.)                     %Lambda (ALamKchM 3050)
     & & & & \\  
   \hline
   \hline
  \multirow{6}{*}{\LamKchM \& \ALamKchP}  
   & \multirow{2}{*}{0-10\%} 
     & \LamKchM
     & \AaLamKchM(1) $\pm$ \AaLamKchM(2) (stat.) $\pm$ \AaLamKchM(3) (sys.)                      %Lambda (LamKchM 0010)
     & \multirow{2}{*}{\AaLamKchM(4) $\pm$ \AaLamKchM(5) (stat.) $\pm$ \AaLamKchM(6) (sys.)}     %Radius (LamKchM & ALamKchP 0010)
     & \multirow{6}{*}{\AaLamKchM(19) $\pm$ \AaLamKchM(20) (stat.) $\pm$ \AaLamKchM(21) (sys.)}     %Ref0   (LamKchM & ALamKchP)
     & \multirow{6}{*}{\AaLamKchM(22) $\pm$ \AaLamKchM(23) (stat.) $\pm$ \AaLamKchM(24) (sys.)}     %Imf0   (LamKchM & ALamKchP)
     & \multirow{6}{*}{\AaLamKchM(25) $\pm$ \AaLamKchM(26) (stat.) $\pm$ \AaLamKchM(27) (sys.)} \\ %d0     (LamKchM & ALamKchP)
     
     & & \ALamKchP 
     & \AaALamKchP(1) $\pm$ \AaALamKchP(2) (stat.) $\pm$ \AaALamKchP(3) (sys.)                      %Lambda (ALamKchP 0010)
     & & & & \\          
   \cline{2-5}
   
   & \multirow{2}{*}{10-30\%}
     & \LamKchM
     & \AaLamKchM(7) $\pm$ \AaLamKchM(8) (stat.) $\pm$ \AaLamKchM(9) (sys.)                      %Lambda (LamKchM 1030) 
     & \multirow{2}{*}{\AaLamKchM(10) $\pm$ \AaLamKchM(11) (stat.) $\pm$ \AaLamKchM(12) (sys.)}     %Radius (LamKchM & ALamKchP 1030)
     & & & \\
             
     & & \ALamKchP 
     & \AaALamKchP(7) $\pm$ \AaALamKchP(8) (stat.) $\pm$ \AaALamKchP(9) (sys.)                      %Lambda (ALamKchP 1030)
     & & & & \\  
   \cline{2-5}
   
   & \multirow{2}{*}{30-50\%}
     & \LamKchM
     & \AaLamKchM(13) $\pm$ \AaLamKchM(14) (stat.) $\pm$ \AaLamKchM(15) (sys.)                      %Lambda (LamKchM 3050) 
     & \multirow{2}{*}{\AaLamKchM(16) $\pm$ \AaLamKchM(17) (stat.) $\pm$ \AaLamKchM(18) (sys.)}     %Radius (LamKchM & ALamKchP 3050)
     & & & \\
             
     & & \ALamKchP 
     & \AaALamKchP(13) $\pm$ \AaALamKchP(14) (stat.) $\pm$ \AaALamKchP(15) (sys.)                      %Lambda (ALamKchP 3050)
     & & & & \\     
   \hline
 \end{tabular}}
 \caption[Fit Results \LamALamKpm, with 10 residual correlations included]{Fit Results \LamALamKpm, with 10 residual correlations included.
 Each pair is fit simultaneously with its conjugate (ie. \LamKchP with \ALamKchM and \LamKchM with \ALamKchP) across all centralities (0-10\%, 10-30\%, 30-50\%), for a total of 6 simultaneous analyses in the fit.
 Each analysis has a unique $\lambda$ and normalization parameter.
 The radii are shared between analyses of like centrality, as these should have similar source sizes.
 The scattering parameters ($\mathbb{R}f_{0}$, $\mathbb{I}f_{0}$, $d_{0}$) are shared amongst all.
 The background is modeled by a (6$^{\mathrm{th}}$-)degree polynomial fit to THERMINATOR simulation.
 The fit is done on the data with only statistical error bars.
 The errors marked as ``stat." are those returned by MINUIT.
 The errors marked as ``sys." are those which result from my systematic analysis (as outlined in Section \ref{SystematicErrors}).}
 \label{tab:FitResultsLamKch_10Res}
\end{table}  
\end{comment}

%%%%%%%%%%%%%%%%%%%%%%%%%%%%%%%%%%%%%%%%%%%%%%%%%%%%%%%%%%%%%%%%%%%%%%%%%%%%%%%%%%%%%%%%%%%%%%%%%%%%%%%%%%%%%%%%%%%%%%%%%%%%%%%%%
%%%%%%%%%%%%%%%%%%%% LamKch, polynomial background, unique radii and share lamconj
\begin{comment}
\clearpage
\begin{table}[htbp]
 \centering
 \renewcommand{\arraystretch}{1.25}
 \resizebox{\paperwidth}{!}{
 \begin{tabular}{|c|c|c|c|c|c|c|}
  \multicolumn{7}{c}{Fit Results \LamALamKpm} \\
  \hline
  \multirow{2}{*}{System} & \multirow{2}{*}{Centrality} & \multicolumn{5}{c|}{Fit Parameters} \\
  \cline{3-7}
   & & $\lambda$ & $R$ & $\mathbb{R}f_{0}$ & $\mathbb{I}f_{0}$ & $d_{0}$ \\
  \hline  
  \multirow{6}{*}{\LamKchP \& \ALamKchM}  
   & \multirow{2}{*}{0-10\%} 
     & \multirow{2}{*}{\BaLamKchP(1) $\pm$ \BaLamKchP(2) (stat.) $\pm$ \BaLamKchP(3) (sys.)}    %Lambda (LamKchP 0010)
     & \multirow{2}{*}{\BaLamKchP(4) $\pm$ \BaLamKchP(5) (stat.) $\pm$ \BaLamKchP(6) (sys.)}    %Radius (LamKchP & ALamKchM 0010)
     & \multirow{6}{*}{\BaLamKchP(19) $\pm$ \BaLamKchP(20) (stat.) $\pm$ \BaLamKchP(21) (sys.)}   %Ref0   (LamKchP & ALamKchM)
     & \multirow{6}{*}{\BaLamKchP(22) $\pm$ \BaLamKchP(23) (stat.) $\pm$ \BaLamKchP(24) (sys.)}    %Imf0   (LamKchP & ALamKchM)
     & \multirow{6}{*}{\BaLamKchP(25) $\pm$ \BaLamKchP(26) (stat.) $\pm$ \BaLamKchP(27) (sys.)} \\ %d0     (LamKchP & ALamKchM)
     
     & & & & & & \\          
   \cline{2-4}
   
   & \multirow{2}{*}{10-30\%}
     & \multirow{2}{*}{\BaLamKchP(7) $\pm$ \BaLamKchP(8) (stat.) $\pm$ \BaLamKchP(9) (sys.)}       %Lambda (LamKchP 1030)
     & \multirow{2}{*}{\BaLamKchP(10) $\pm$ \BaLamKchP(11) (stat.) $\pm$ \BaLamKchP(12) (sys.)}    %Radius (LamKchP & ALamKchM 1030) 
     & & & \\
             
     & & & & & & \\  
   \cline{2-4}
   
   & \multirow{2}{*}{30-50\%}
     & \multirow{2}{*}{\BaLamKchP(13) $\pm$ \BaLamKchP(14) (stat.) $\pm$ \BaLamKchP(15) (sys.)}    %Lambda (LamKchP 3050)
     & \multirow{2}{*}{\BaLamKchP(16) $\pm$ \BaLamKchP(17) (stat.) $\pm$ \BaLamKchP(18) (sys.)}    %Radius (LamKchP & ALamKchM 3050)
     & & & \\
             
     & & & & & & \\  
   \hline
   \hline
  \multirow{6}{*}{\LamKchM \& \ALamKchP}  
   & \multirow{2}{*}{0-10\%} 
     & \multirow{2}{*}{\BaLamKchM(1) $\pm$ \BaLamKchM(2) (stat.) $\pm$ \BaLamKchM(3) (sys.)}     %Lambda (LamKchM 0010)
     & \multirow{2}{*}{\BaLamKchM(4) $\pm$ \BaLamKchM(5) (stat.) $\pm$ \BaLamKchM(6) (sys.)}     %Radius (LamKchM & ALamKchP 0010)
     & \multirow{6}{*}{\BaLamKchM(19) $\pm$ \BaLamKchM(20) (stat.) $\pm$ \BaLamKchM(21) (sys.)}     %Ref0   (LamKchM & ALamKchP)
     & \multirow{6}{*}{\BaLamKchM(22) $\pm$ \BaLamKchM(23) (stat.) $\pm$ \BaLamKchM(24) (sys.)}     %Imf0   (LamKchM & ALamKchP)
     & \multirow{6}{*}{\BaLamKchM(25) $\pm$ \BaLamKchM(26) (stat.) $\pm$ \BaLamKchM(27) (sys.)} \\ %d0     (LamKchM & ALamKchP)
     
     & & & & & & \\          
   \cline{2-4}
   
   & \multirow{2}{*}{10-30\%}
     & \multirow{2}{*}{\BaLamKchM(7) $\pm$ \BaLamKchM(8) (stat.) $\pm$ \BaLamKchM(9) (sys.)}        %Lambda (LamKchM 1030) 
     & \multirow{2}{*}{\BaLamKchM(10) $\pm$ \BaLamKchM(11) (stat.) $\pm$ \BaLamKchM(12) (sys.)}     %Radius (LamKchM & ALamKchP 1030)
     & & & \\
             
     & & & & & & \\  
   \cline{2-4}
   
   & \multirow{2}{*}{30-50\%}
     & \multirow{2}{*}{\BaLamKchM(13) $\pm$ \BaLamKchM(14) (stat.) $\pm$ \BaLamKchM(15) (sys.)}     %Lambda (LamKchM 3050) 
     & \multirow{2}{*}{\BaLamKchM(16) $\pm$ \BaLamKchM(17) (stat.) $\pm$ \BaLamKchM(18) (sys.)}     %Radius (LamKchM & ALamKchP 3050)
     & & & \\
             
     & & & & & & \\     
   \hline
 \end{tabular}}
 \caption[Fit Results \LamALamKpm, with 10 residual correlations included]{Fit Results \LamALamKpm, with 10 residual correlations included.
 Each pair is fit simultaneously with its conjugate (ie. \LamKchP with \ALamKchM and \LamKchM with \ALamKchP) across all centralities (0-10\%, 10-30\%, 30-50\%), for a total of 6 simultaneous analyses in the fit.
 A $\lambda$ parameter is shared between a pair and its conjugate for each centrality.
 Each analysis has a unique normalization parameter.
 The radii are shared between analyses of like centrality, as these should have similar source sizes.
 The scattering parameters ($\mathbb{R}f_{0}$, $\mathbb{I}f_{0}$, $d_{0}$) are shared amongst all.
 The background is modeled by a (6$^{\mathrm{th}}$-)degree polynomial fit to THERMINATOR simulation.
 The fit is done on the data with only statistical error bars.
 The errors marked as ``stat." are those returned by MINUIT.
 The errors marked as ``sys." are those which result from my systematic analysis (as outlined in Section \ref{SystematicErrors}).}
 \label{tab:FitResultsLamKch_10Res}
\end{table}  
\end{comment}


%%%%%%%%%%%%%%%%%%%%%%%%%%%%%%%%%%%%%%%%%%%%%%%%%%%%%%%%%%%%%%%%%%%%%%%%%%%%%%%%%%%%%%%%%%%%%%%%%%%%%%%%%%%%%%%%%%%%%%%%%%%%%%%%%
%%%%%%%%%%%%%%%%%%%% LamKch, polynomial background, share radii and single lambda
%\begin{comment}
\begin{table}[htbp]
 \centering
 \renewcommand{\arraystretch}{1.25}
 \resizebox{\paperwidth}{!}{
 \begin{tabular}{|c|c|c|c|c|c|c|}
  \multicolumn{7}{c}{Fit Results \LamALamKpm} \\
  \hline
  \multirow{2}{*}{System} & \multirow{2}{*}{Centrality} & \multicolumn{5}{c|}{Fit Parameters} \\
  \cline{3-7}
   & & $\lambda$ & $R$ & $\mathbb{R}f_{0}$ & $\mathbb{I}f_{0}$ & $d_{0}$ \\
  \hline  
  \multirow{3}{*}{\LamKchP \& \ALamKchM}  
   & \multirow{2}{*}{0-10\%} 
     & \multirow{2}{*}{\EaLamKchP(1) $\pm$ \EaLamKchP(2) (stat.) $\pm$ \EaLamKchP(3) (sys.)}    %Lambda (LamKchP 0010)
     & \multirow{2}{*}{\EaLamKchP(4) $\pm$ \EaLamKchP(5) (stat.) $\pm$ \EaLamKchP(6) (sys.)}    %Radius (LamKchP & ALamKchM 0010)
     & \multirow{3}{*}{\EaLamKchP(19) $\pm$ \EaLamKchP(20) (stat.) $\pm$ \EaLamKchP(21) (sys.)}   %Ref0   (LamKchP & ALamKchM)
     & \multirow{3}{*}{\EaLamKchP(22) $\pm$ \EaLamKchP(23) (stat.) $\pm$ \EaLamKchP(24) (sys.)}    %Imf0   (LamKchP & ALamKchM)
     & \multirow{3}{*}{\EaLamKchP(25) $\pm$ \EaLamKchP(26) (stat.) $\pm$ \EaLamKchP(27) (sys.)} \\ %d0     (LamKchP & ALamKchM)
     
     & & & & & & \\          
   \cline{2-4}
   
   & \multirow{2}{*}{10-30\%}
     & \multirow{2}{*}{\EaLamKchP(7) $\pm$ \EaLamKchP(8) (stat.) $\pm$ \EaLamKchP(9) (sys.)}       %Lambda (LamKchP 1030)
     & \multirow{2}{*}{\EaLamKchP(10) $\pm$ \EaLamKchP(11) (stat.) $\pm$ \EaLamKchP(12) (sys.)}    %Radius (LamKchP & ALamKchM 1030) 
     & & & \\
   \clineB{1-1}{4.0} 
   \clineB{5-7}{4.0}            
  \multirow{3}{*}{\LamKchP \& \ALamKchM} 
   & & & 
     & \multirow{3}{*}{\EaLamKchM(19) $\pm$ \EaLamKchM(20) (stat.) $\pm$ \EaLamKchM(21) (sys.)}     %Ref0   (LamKchM & ALamKchP)
     & \multirow{3}{*}{\EaLamKchM(22) $\pm$ \EaLamKchM(23) (stat.) $\pm$ \EaLamKchM(24) (sys.)}     %Imf0   (LamKchM & ALamKchP)
     & \multirow{3}{*}{\EaLamKchM(25) $\pm$ \EaLamKchM(26) (stat.) $\pm$ \EaLamKchM(27) (sys.)} \\ %d0     (LamKchM & ALamKchP)
   \cline{2-4}
   
   & \multirow{2}{*}{30-50\%}
     & \multirow{2}{*}{\EaLamKchP(13) $\pm$ \EaLamKchP(14) (stat.) $\pm$ \EaLamKchP(15) (sys.)}    %Lambda (LamKchP 3050)
     & \multirow{2}{*}{\EaLamKchP(16) $\pm$ \EaLamKchP(17) (stat.) $\pm$ \EaLamKchP(18) (sys.)}    %Radius (LamKchP & ALamKchM 3050)
     & & & \\
             
     & & & & & & \\  
   \hline
 \end{tabular}}
 \caption[Fit Results \LamALamKpm, with 10 residual correlations included]{Fit Results \LamALamKpm, with 10 residual correlations included.
 All \LamKpm analyses are fit simultaneously across all centralities (0-10\%, 10-30\%, 30-50\%).
 Scattering parameters ($\mathbb{R}f_{0}$, $\mathbb{I}f_{0}$, $d_{0}$) are shared between pair-conjugate systems (i.e. a parameter set describing the \LamKchP \& \ALamKchM system, and a separate set describing the \LamKchM \& \ALamKchP system).
 For each centrality, a radius and $\lambda$ parameters are shared between all pairs (\LamKchP, \ALamKchM, \LamKchM, \ALamKchP).
 Each analysis has a unique normalization parameter.
 The background is modeled by a (6$^{\mathrm{th}}$-)degree polynomial fit to THERMINATOR simulation.
 The fit is done on the data with only statistical error bars.
 The errors marked as ``stat." are those returned by MINUIT.
 The errors marked as ``sys." are those which result from my systematic analysis (as outlined in Section \ref{SystematicErrors}).}
 \label{tab:FitResultsLamKch_10Res}
\end{table}
%\end{comment}

  
%%%%%%%%%%%%%%%%%%%%%%%%%%%%%%%%%%%%%%%%%%%%%%%%%%%%%%%%%%%%%%%%%%%%%%%%%%%%%%%%%%%%%%%%%%%%%%%%%%%%%%%%%%%%%%%%%%%%%%%%%%%%%%%%%
%%%%%%%%%%%%%%%%%%%% LamKch, polynomial background, share radii, share lamconj
\begin{comment}
%\clearpage
\begin{table}[htbp]
 \centering
 \renewcommand{\arraystretch}{1.25}
 \resizebox{\paperwidth}{!}{
 \begin{tabular}{|c|c|c|c|c|c|c|}
  \multicolumn{7}{c}{Fit Results \LamALamKpm} \\
  \hline
  \multirow{2}{*}{System} & \multirow{2}{*}{Centrality} & \multicolumn{5}{c|}{Fit Parameters} \\
  \cline{3-7}
   & & $\lambda$ & $R$ & $\mathbb{R}f_{0}$ & $\mathbb{I}f_{0}$ & $d_{0}$ \\
  \hline  
  \multirow{6}{*}{\LamKchP \& \ALamKchM}  
   & \multirow{2}{*}{0-10\%} 
     & \multirow{2}{*}{\DaLamKchP(1) $\pm$ \DaLamKchP(2) (stat.) $\pm$ \DaLamKchP(3) (sys.)}    %Lambda (LamKchP 0010)
     & \multirow{2}{*}{\DaLamKchP(4) $\pm$ \DaLamKchP(5) (stat.) $\pm$ \DaLamKchP(6) (sys.)}    %Radius (LamKchP & ALamKchM 0010)
     & \multirow{6}{*}{\DaLamKchP(19) $\pm$ \DaLamKchP(20) (stat.) $\pm$ \DaLamKchP(21) (sys.)}   %Ref0   (LamKchP & ALamKchM)
     & \multirow{6}{*}{\DaLamKchP(22) $\pm$ \DaLamKchP(23) (stat.) $\pm$ \DaLamKchP(24) (sys.)}    %Imf0   (LamKchP & ALamKchM)
     & \multirow{6}{*}{\DaLamKchP(25) $\pm$ \DaLamKchP(26) (stat.) $\pm$ \DaLamKchP(27) (sys.)} \\ %d0     (LamKchP & ALamKchM)
     
     & & & & & & \\          
   \cline{2-4}
   
   & \multirow{2}{*}{10-30\%}
     & \multirow{2}{*}{\DaLamKchP(7) $\pm$ \DaLamKchP(8) (stat.) $\pm$ \DaLamKchP(9) (sys.)}       %Lambda (LamKchP 1030)
     & \multirow{2}{*}{\DaLamKchP(10) $\pm$ \DaLamKchP(11) (stat.) $\pm$ \DaLamKchP(12) (sys.)}    %Radius (LamKchP & ALamKchM 1030) 
     & & & \\
             
     & & & & & & \\  
   \cline{2-4}
   
   & \multirow{2}{*}{30-50\%}

     & \multirow{2}{*}{\DaLamKchP(13) $\pm$ \DaLamKchP(14) (stat.) $\pm$ \DaLamKchP(15) (sys.)}    %Lambda (LamKchP 3050)
     & \multirow{2}{*}{\DaLamKchP(16) $\pm$ \DaLamKchP(17) (stat.) $\pm$ \DaLamKchP(18) (sys.)}    %Radius (LamKchP & ALamKchM 3050)
     & & & \\
             
     & & & & & & \\  
   \hline
   \hline
  \multirow{6}{*}{\LamKchM \& \ALamKchP}  
   & \multirow{2}{*}{0-10\%} 
     & \multirow{2}{*}{\DaLamKchM(1) $\pm$ \DaLamKchM(2) (stat.) $\pm$ \DaLamKchM(3) (sys.)}     %Lambda (LamKchM 0010)
     & \multirow{2}{*}{\DaLamKchM(4) $\pm$ \DaLamKchM(5) (stat.) $\pm$ \DaLamKchM(6) (sys.)}     %Radius (LamKchM & ALamKchP 0010)
     & \multirow{6}{*}{\DaLamKchM(19) $\pm$ \DaLamKchM(20) (stat.) $\pm$ \DaLamKchM(21) (sys.)}     %Ref0   (LamKchM & ALamKchP)
     & \multirow{6}{*}{\DaLamKchM(22) $\pm$ \DaLamKchM(23) (stat.) $\pm$ \DaLamKchM(24) (sys.)}     %Imf0   (LamKchM & ALamKchP)
     & \multirow{6}{*}{\DaLamKchM(25) $\pm$ \DaLamKchM(26) (stat.) $\pm$ \DaLamKchM(27) (sys.)} \\ %d0     (LamKchM & ALamKchP)
     
     & & & & & & \\          
   \cline{2-4}
   
   & \multirow{2}{*}{10-30\%}
     & \multirow{2}{*}{\DaLamKchM(7) $\pm$ \DaLamKchM(8) (stat.) $\pm$ \DaLamKchM(9) (sys.)}        %Lambda (LamKchM 1030) 
     & \multirow{2}{*}{\DaLamKchM(10) $\pm$ \DaLamKchM(11) (stat.) $\pm$ \DaLamKchM(12) (sys.)}     %Radius (LamKchM & ALamKchP 1030)
     & & & \\
             
     & & & & & & \\  
   \cline{2-4}
   
   & \multirow{2}{*}{30-50\%}
     & \multirow{2}{*}{\DaLamKchM(13) $\pm$ \DaLamKchM(14) (stat.) $\pm$ \DaLamKchM(15) (sys.)}     %Lambda (LamKchM 3050) 
     & \multirow{2}{*}{\DaLamKchM(16) $\pm$ \DaLamKchM(17) (stat.) $\pm$ \DaLamKchM(18) (sys.)}     %Radius (LamKchM & ALamKchP 3050)
     & & & \\
             
     & & & & & & \\     
   \hline
 \end{tabular}}
 \caption[Fit Results \LamALamKpm, with 10 residual correlations included]{Fit Results \LamALamKpm, with 10 residual correlations included.
 All \LamKpm analyses are fit simultaneously across all centralities (0-10\%, 10-30\%, 30-50\%).
 Scattering parameters ($\mathbb{R}f_{0}$, $\mathbb{I}f_{0}$, $d_{0}$) are shared between pair-conjugate systems (i.e. a parameter set describing the \LamKchP \& \ALamKchM system, and a separate set describing the \LamKchM \& \ALamKchP system).
 For each centrality, a radius parameter is shared between all pairs (\LamKchP, \ALamKchM, \LamKchM, \ALamKchP), and a $\lambda$ parameter is shared between a pair and its conjugate.
 Each analysis has a unique normalization parameter.
 The background is modeled by a (6$^{\mathrm{th}}$-)degree polynomial fit to THERMINATOR simulation.
 The fit is done on the data with only statistical error bars.
 The errors marked as ``stat." are those returned by MINUIT.
 The errors marked as ``sys." are those which result from my systematic analysis (as outlined in Section \ref{SystematicErrors}).}
 \label{tab:FitResultsLamKch_10Res}
\end{table}  
\end{comment}
%%%%%%%%%%%%%%%%%%%%%%%%%%%%%%%%%%%%%%%%%%%%%%%%%%%%%%%%%%%%%%%%%%%%%%%%%%%%%%%%%%%%%%%%%%%%%%%%%%%%%%%%%%%%%%%%%%%%%%%%%%%%%%%%%

\end{landscape}
\pagestyle{plain}


\end{document}
