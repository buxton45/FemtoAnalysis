\documentclass[ALICE,manyauthors]{ALICE_analysis_notes}
%\documentclass[ALICE,manyauthors]{ALICE_scientific_notes}
%
%\newcommand{\jpsi}{\rm J/$\psi$}
%\newcommand{\psip}{$\psi^\prime$}
%\newcommand{\jpsiDY}{\rm J/$\psi$\,/\,DY}
%\newcommand{\dd}{\mathrm{d}}
%\newcommand{\chic}{$\chi_{\rm c}$}
%\newcommand{\ezdc}{$E_{\rm ZDC}$}
%\newcommand{\red}{\textcolor{red}}
%\newcommand{\blue}{\textcolor{blue}}
\newcommand{\slfrac}[2]{\left.#1\right/#2}
\usepackage{rotating}
\usepackage{comment} 

\usepackage{color}	%May be necessary for colored links
\usepackage{hyperref}
\hypersetup{
	colorlinks=true, %set true if you want colored links
	linktoc=all, %set to all for both sections and subsections linked
	linkcolor=red, %choose some color for links
	anchorcolor=black,
	citecolor=green,
	filecolor=cyan,
	menucolor=red,
	runcolor=cyan,
	urlcolor=magenta
}
\usepackage{pdflscape}
\textwidth=12cm
\usepackage[margin=0.25in]{geometry}
%
\begin{document}%
%%%%%%%%%%%%% ptdr definitions %%%%%%%%%%%%%%%%%%%%%
%
%%%%%%%%%%%%%%%  Title page %%%%%%%%%%%%%%%%%%%%%%%%
%
\begin{titlepage}
%
\PHnumber{} 
\PHdate{\today}

 \pagestyle{empty}
 \begin{landscape}

%
%%% Put your own title + short title here:
\title{Lambda-Kaon Femtoscopy in Pb-Pb Collisions at $\sqrt{s_{NN}}$ = 2.76 TeV with ALICE}
\ShortTitle{$\Lambda$-K Femtoscopy with ALICE}   % appears on right page headers
%
\author{J. T. Buxton and T. J. Humanic}
\author{
Department of Physics, The Ohio State University, Columbus, Ohio, USA\\
}
\author{Email: jesse.thomas.buxton@cern.ch}
%%
\begin{abstract}
We present results from a femtoscopic analysis of Lambda-Kaon correlations in Pb-Pb collisions at $\sqrt{s_{NN}}$ = 2.76 TeV by the ALICE experiment at the LHC.
The femtoscopic correlations are the result of strong final-state interactions, and are fit with a parametrization based on a model by R. Lednicky and V. L. Lyuboshitz[1].
This allows us to both characterize the emission source and measure the scattering parameters for the particle pairs.
We observe a large difference in the $\Lambda$K$^{+}$ and $\Lambda$K$^{-}$ correlations in pairs with low relative momenta (k$^{*}$ $\lesssim$ 100 MeV).
The results suggest an effect arising from different quark-antiquark interactions in the pairs, i.e. $\rm s\bar{s}$ in $\Lambda$K$^{+}$ and $\rm u\bar{u}$ in $\Lambda$K$^{-}$.

\clearpage
We present our femtoscopic analysis of $\mathrm{\Lambda}$K correlations in Pb-Pb collisions at $\sqrt{s_{\mathrm{NN}}}$ = 2.76 TeV from ALICE.
The femtoscopic correlations are the result of strong final-state interactions, and are fit with a parametrization based on a model by R. Lednicky and V. L. Lyuboshitz[1].
This allows us to both characterize the emission source and measure the scattering parameters for the particle pairs.
We observe a large difference in the $\Lambda$K$^{+}$ and $\Lambda$K$^{-}$ correlations in pairs with low relative momenta (k$^{*}$ $\lesssim$ 100 MeV/\textit{c}).
This might suggest an effect arising from different quark-antiquark interactions between the pairs, i.e. $\rm s\bar{s}$ in $\Lambda$K$^{+}$ and $\rm u\bar{u}$ in $\Lambda$K$^{-}$.
To investigate further this interesting result, we conduct a $\Xi$K analysis, for which preliminary results are shown.


\clearpage
We present our femtoscopic analysis of $\mathrm{\Lambda}$K correlations in Pb-Pb collisions at $\sqrt{s_{\mathrm{NN}}}$ = 2.76 TeV from ALICE.
The femtoscopic correlations result from strong final-state interactions, and are fit with a parametrization based on a model by Lednicky and Lyuboshitz[1].
This allows us to both characterize the emission source and measure the scattering parameters for the particle pairs.
We observe a large difference in the $\Lambda$K$^{+}$ and $\Lambda$K$^{-}$ correlations in pairs with low relative momenta.
This might suggest an effect arising from different quark-antiquark interactions between the pairs ($\rm s\bar{s}$ in $\Lambda$K$^{+}$ and $\rm u\bar{u}$ in $\Lambda$K$^{-}$), or from different net strangeness of the systems.
To investigate further, we conduct a $\Xi$K analysis, for which preliminary results are shown.


\end{abstract}


 \end{landscape}
 \pagestyle{plain}


\end{titlepage}




\end{document}
