\documentclass[../AnalysisNoteJBuxton.tex]{subfiles}
\begin{document}

\section{Systematic Errors}
\label{SystematicErrors}

This study is currently ongoing, and an estimate of my systematic uncertainties should be complete within a week.
In order to understand my systematic uncertainties, the analysis code was run many times using slightly different values for a number of important cuts, and the results were compared.
To quantify the effect, the difference in two correlation functions obtained using different values for a given cut was fit with a simple exponential decay function:

\begin{equation}
  \Delta C(k^{*}) = Ae^{-Bk^{*}}
\label{eqn:ExpDecay}
\end{equation}

The amplitude, $A$, and its associated uncertainty for the various cuts can be found in Tables \ref{tab:LamDcaLamK0} through \ref{tab:AvgSepLamKch}.
The systematic effect of the variation is marked as significant (``Sig" column) if the amplitude is not withih 2$\sigma$ of 0.
Although this proves qualitatively useful, these fits will likely not be used to quantify the systematic effects.

In order to quantify the systematic errors on the correlation functions, all correlations will be averaged, and the resulting variance will be taken as the systematic error.
Similarly, the fit parameters extracted from all of these correlation functions will be averaged, and they resulting variances will be taken as the systematic errors for the fit parameters.


\subfile{6_SystematicErrors/6.1_SysErrsLamK0.tex}
\subfile{6_SystematicErrors/6.2_SysErrsLamKch.tex}

\end{document}