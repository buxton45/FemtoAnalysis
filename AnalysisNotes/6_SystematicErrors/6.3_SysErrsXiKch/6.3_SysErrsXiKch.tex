\documentclass[/home/jesse/Analysis/FemtoAnalysis/AnalysisNotes/AnalysisNoteJBuxton.tex]{subfiles}
\begin{document}

\subsection{Systematic Errors: \texorpdfstring{$\Xi$K$^{\pm}$}{TEXT}}
\label{SysErrsXiKch}

\subsubsection{Particle and Pair Cuts}
\label{SysErrsXiKch:ParticleAndPairCuts}

The cuts included in the systematic study, as well as the values used in the variations, are listed below.  Note, the central value corresponds to that used in the analysis.




\begin{table}[htbp]
 \centering 
  \renewcommand{\arraystretch}{1.2}
  \begin{tabular}{|l|r|}
   \multicolumn{2}{c}{\XiKpm systematics} \\
   \hline  
   DCA to PV $\Xi$($\bar{\Xi}$) & $<$ [2, 3, 4] mm \\
   \hline
   DCA $\Xi$($\bar{\Xi}$) Daughters & $<$ [2, 3, 4] mm \\
   \hline
   $\cos(\theta_{PA})$ $\Xi$($\bar{\Xi}$) to PV & $>$ [0.9991, 0.9992, 0.9993] \\
   \hline
   $\cos(\theta_{PA})$ \LamALam to $\Xi$($\bar{\Xi}$) DV & $>$ [0.9992, 0.9993, 0.9994] \\
   \hline 
   DCA to PV bachelor $\pi$ & $>$ [0.5, 1, 2] mm \\
   \hline
   DCA to PV \LamALam & $>$ [1, 2, 3] mm \\
   \hline
   DCA \LamALam Daughters & $<$ [3, 4, 5] mm \\
   \hline
   DCA to PV of $p$($\bar{p}$) Daughter of \LamALam & $>$ [0.5, 1, 2] mm \\
   \hline
   DCA to PV of $\pi^{-}$($\pi^{+}$) Daughter of \LamALam & $>$ [2, 3, 4] mm \\ 
   \hline
   $\overline{\Delta\mathbf{r}}$ of \LamALam Daughter and \Kpm with like charge & $>$ [7, 8, 9] cm \\
   \hline
   $\overline{\Delta\mathbf{r}}$ of Bachelor $\pi$ and \Kpm with like charge & $>$ [7, 8, 9] cm \\
   \hline
   DCA to PV in Transverse Plane of \Kpm & $<$ [1.92, 2.4, 2.88] \\
   \hline
   DCA to PV in Longitudinal Direction of \Kpm & $<$ [2.4, 3.0, 3.6] \\
   \hline
  \end{tabular}
% \end{minipage}
 \caption[\XiKpm systematics]{\XiKpm systematics. In the table, the shorthand used is as follows: $PA$ = pointing angle; PV = primary vertex; DV = decay vertex; DCA = distance of closest approach; $\overline{\Delta\mathbf{r}}$ = average separation.}
 \label{tab:XiKchSystematics} 
\end{table}







\begin{comment}
\subsubsection{Non-Flat Background}
\label{SysErrsXiKch:NonFlatBgd}

Still needs to be done.  Currently, we fit our non-flat background with a linear function.  We will also use a quadratic form, and analyze how this choice affects our extracted parameter sets.

\subsubsection{Fit Range}
\label{SysErrsXiKch:FitRange}

Our choice of $k^{*}$ fit range was varied by $\pm$ 25\%.  The resulting uncertainties in the extracted parameter sets were combined with our uncertainties arising from our particle and pair cuts.
\end{comment}

\end{document}