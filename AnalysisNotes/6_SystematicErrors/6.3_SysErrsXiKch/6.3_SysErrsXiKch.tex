\documentclass[../AnalysisNoteJBuxton.tex]{subfiles}
\begin{document}

\subsection{Systematic Errors: \texorpdfstring{$\Xi$K$^{\pm}$}{TEXT}}
\label{SysErrsXiKch}

\subsubsection{Particle and Pair Cuts}
\label{SysErrsXiKch:ParticleAndPairCuts}

The cuts included in the systematic study, as well as the values used in the variations, are listed below.  Note, the central value corresponds to that used in the analysis.




\begin{table}[htbp]
 \centering 
  \renewcommand{\arraystretch}{1.2}
  \begin{tabular}{|c|c|}
   \multicolumn{2}{c}{\XiKpm systematics} \\
   \hline  
   Max. DCA $\Xi$($\bar{\Xi}$) & 2, 3, 4 mm \\
   \hline
   Max. DCA $\Xi$($\bar{\Xi}$) Daughters & 2, 3, 4 mm \\
   \hline
   Min. $\Xi$($\bar{\Xi}$) Cosine of Pointing Angle to Primary Vertex & 0.9991, 0.9992, 0.9993 \\
   \hline
   Min. $\Lambda$($\bar{\Lambda}$) Cosine of Pointing Angle to $\Xi$($\bar{\Xi}$) Decay Vertex & 0.9992, 0.9993, 0.9994 \\
   \hline 
   Min. DCA Bachelor $\pi$ & 0.5, 1, 2 mm \\
   \hline
   Min. DCA $\Lambda$($\bar{\Lambda}$) & 1, 2, 3 mm \\
   \hline
   Max. DCA $\Lambda$($\bar{\Lambda}$) Daughters & 3, 4, 5 mm \\
   \hline
   Min. DCA to Primary Vertex of $p$($\bar{p}$) Daughter of $\Lambda$($\bar{\Lambda}$) & 0.5, 1, 2 mm \\
   \hline
   Min. DCA to Primary Vertex of $\pi^{-}$($\pi^{+}$) Daughter of $\Lambda$($\bar{\Lambda}$) & 2, 3, 4 mm \\ 
   \hline
   Min. Average Separation of $\Lambda$($\bar{\Lambda}$) Daughter and K$^{\pm}$ with like charge & 7, 8, 9 cm \\
   \hline
   Min. Average Separation of Bachelor $\pi$ and K$^{\pm}$ with like charge & 7, 8, 9 cm \\
   \hline
   Max. DCA to Primary Vertex in Transverse Plane of K$^{\pm}$ & 1.92, 2.4, 2.88 \\
   \hline
   Max. DCA to Primary Vertex in Longitudinal Direction of K$^{\pm}$ & 2.4, 3.0, 3.6 \\
   \hline
  \end{tabular}
% \end{minipage}
 \caption{\XiKpm systematics}
 \label{tab:XiKchSystematics} 
\end{table}







\begin{comment}
\subsubsection{Non-Flat Background}
\label{SysErrsXiKch:NonFlatBgd}

Still needs to be done.  Currently, we fit our non-flat background with a linear function.  We will also use a quadratic form, and analyze how this choice affects our extracted parameter sets.

\subsubsection{Fit Range}
\label{SysErrsXiKch:FitRange}

Our choice of $k^{*}$ fit range was varied by $\pm$ 25\%.  The resulting uncertainties in the extracted parameter sets were combined with our uncertainties arising from our particle and pair cuts.
\end{comment}

\end{document}