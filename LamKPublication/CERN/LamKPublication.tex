\documentclass[ALICE,manyauthors]{cernphprep}

\usepackage[comma,square,numbers,sort&compress]{natbib}
\usepackage{hyperref}
\usepackage{lineno}
%\linenumbers

\begin{document}%

%%%%%%%%%%%%%%%  Title page %%%%%%%%%%%%%%%%%%%%%%%%
\begin{titlepage}
%
\PHyear{2015}
\PHnumber{XXX}      % required, will be obtained from PH
\PHdate{Day Month}  % required, will be obtained from PH
%

%%% Put your own title + short title here:
\title{$\Lambda$K femtoscopy in Pb-Pb collisions at $\sqrt{s_{\mathrm{NN}}} = $ 2.76 TeV}
\ShortTitle{$\Lambda$K femtoscopy in Pb-Pb collisions}   % appears on right page headers

%%% Do not change the next lines
\Collaboration{ALICE Collaboration\thanks{See Appendix~\ref{app:collab} for the list of collaboration members}}
\ShortAuthor{ALICE Collaboration} % appears on left page headers, do not change

\begin{abstract}
We present our femtoscopy analysis of $\Lambda$K correlations in Pb-Pb collisions at $\sqrt{s_{\mathrm{NN}}}$ = 2.76 TeV from ALICE.  The femtoscopic correlations result from strong final-state interactions, and are fit with a parametrization based on a model by Lednicky and Lyuboshitz.  This allows us to both characterize the emission source and measure the scattering parameters for the particle pairs.  We observe a large difference in the $\Lambda$K$^{+}$ and $\Lambda$K$^{-}$ correlations in pairs with low relative momenta.  This might suggest an effect arising from different quark-antiquark interactions between the pairs ($\mathrm{s}\bar{\mathrm{s}}$ in $\Lambda$K$^{+}$ and $\mathrm{u}\bar{\mathrm{u}}$ in $\Lambda$K$^{-}$), or from different net strangeness for each system.
\end{abstract}
\end{titlepage}
\setcounter{page}{2}

\section{Introduction}
\label{sec:Introduction}
This is where the introduction goes.

\section{Data Analysis}
\label{sec:DataAnalysis}
This is where the data analysis section goes.

\subsection{V0 selection}
\label{sec:V0Selection}
This is how we select V0s.

\subsubsection{$\Lambda$ selection}
\label{sec:LambdaSelection}
This is how we select $\Lambda$ candidates.

\subsubsection{K$^{0}_{S}$ selection}
\label{sec:K0sSelection}
This is how we select K$^{0}_{S}$ candidates.

\subsection{K$^{\pm}$ selection}
\label{sec:KchSelection}
This is how we select K$^{\mathrm{ch}}$ or K$^{\pm}$ candidates.

\section{Construction of correlation functions and fitting}
\label{sec:CfConstructionAndFitting}
This is how we do it.

\subsection{Fit Function}
\label{sec:FitFunction}
Ya boys Lednicky and Lyuboshitz!

\subsection{Systematic uncertainties}
\label{sec:SysErrs}
This is the worst.

\section{Results}
\label{sec:Results}
Hooray, finally some results!

\section{Summary}
\label{sec:Summary}
We did physics, and we found physics.

%%%%% acknowledgements
\newenvironment{acknowledgement}{\relax}{\relax}
\begin{acknowledgement}
\section*{Acknowledgements}
%\input{acknowledgements.tex}    %%%%%%% done by webmaster team
\end{acknowledgement}

%%%%%%%% Bibliography (In case of using bibtex generate the bbl requested by arXiv)
%\bibliographystyle{utphys}   % Remember we use title in the biblio
%\bibliography{biblio}
%\input {bibliography.tex}  

%%%%%%%%% appendix with author list
\newpage
\appendix
%
%\input{}               %%%%%%%%%%% put your appendices here
%
\section{The ALICE Collaboration}
\label{app:collab}
%\input{authorlist-preprint.tex}  %%%%%%% done by webmaster team
\end{document}
