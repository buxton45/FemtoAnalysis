%%% NOTE: If I want to run this standalone, uncomment out lines below
%%%       But lines must be commented out to run within larger project
\begin{comment}
\makeatletter
\def\input@path{{/home/jesse/Analysis/FemtoAnalysis/AnalysisNotes/}}
\makeatother

\documentclass[ALICE,manyauthors]{ALICE_analysis_notes}

\usepackage{MyStyle}
\usepackage{chngpage}  % for adjustwidth
\usepackage{boldline}  % to make lines in table bold
                       % V{<factor>} vertical rule in \begin{tabular} command
                       % also \clinB{<spec>}{<factor>} and \hlineB{<factor>}
\usepackage{arrayjobx} % To use the array structures stored in FitResults_cLamcKch_20180505.tex   
\end{comment}  

\newarray\ArrLamKchP
\readarray{ArrLamKchP}{
                       1.12 & 0.32 & 0.25 & 
                       6.33 & 0.99 & 0.31 & 
                       0.79 & 0.19 & 0.23 & 
                       4.77 & 0.61 & 0.17 & 
                       0.70 & 0.18 & 0.30 & 
                       3.47 & 0.46 & 0.10 & 
                       -0.66 & 0.14 & 0.13 & 
                       0.58 & 0.15 & 0.11 & 
                       0.77 & 0.47 & 1.66}

\newarray\ArrALamKchM
\readarray{ArrALamKchM}{
                       1.12 & 0.32 & 0.25 & 
                       6.33 & 0.99 & 0.31 & 
                       0.79 & 0.19 & 0.23 & 
                       4.77 & 0.61 & 0.17 & 
                       0.70 & 0.18 & 0.30 & 
                       3.47 & 0.46 & 0.10 & 
                       -0.66 & 0.14 & 0.13 & 
                       0.58 & 0.15 & 0.11 & 
                       0.77 & 0.47 & 1.66}

\newarray\ArrLamKchM
\readarray{ArrLamKchM}{
                       1.12 & 0.32 & 0.25 & 
                       6.33 & 0.99 & 0.31 & 
                       0.79 & 0.19 & 0.23 & 
                       4.77 & 0.61 & 0.17 & 
                       0.70 & 0.18 & 0.30 & 
                       3.47 & 0.46 & 0.10 & 
                       0.35 & 0.12 & 0.07 & 
                       0.44 & 0.10 & 0.08 & 
                       -4.46 & 1.53 & 1.36}

\newarray\ArrALamKchP
\readarray{ArrALamKchP}{
                       1.12 & 0.32 & 0.25 & 
                       6.33 & 0.99 & 0.31 & 
                       0.79 & 0.19 & 0.23 & 
                       4.77 & 0.61 & 0.17 & 
                       0.70 & 0.18 & 0.30 & 
                       3.47 & 0.46 & 0.10 & 
                       0.35 & 0.12 & 0.07 & 
                       0.44 & 0.10 & 0.08 & 
                       -4.46 & 1.53 & 1.36}


\newarray\ArrLamKs
\readarray{ArrLamKs}{
                       0.60 & 0.87 & 0.57 & 
                       3.15 & 0.68 & 0.45 & 
                       0.60 & 0.87 & 0.57 & 
                       2.48 & 0.54 & 0.35 & 
                       0.60 & 0.87 & 0.57 & 
                       1.80 & 0.38 & 0.17 & 
                       -0.18 & 0.04 & 0.25 & 
                       0.20 & 0.11 & 0.13 & 
                       2.72 & 1.07 & 2.12}

\newarray\ArrALamKs
\readarray{ArrALamKs}{
                       0.60 & 0.87 & 0.57 & 
                       3.15 & 0.68 & 0.45 & 
                       0.60 & 0.87 & 0.57 & 
                       2.48 & 0.54 & 0.35 & 
                       0.60 & 0.87 & 0.57 & 
                       1.80 & 0.38 & 0.17 & 
                       -0.18 & 0.04 & 0.25 & 
                       0.20 & 0.11 & 0.13 & 
                       2.72 & 1.07 & 2.12}




\begin{document}
\pagestyle{empty}

%\pagestyle{empty}
%\begin{landscape}


%%%%%%%%%%%%%%%%%%%%%%%%%%%%%%%%%%%%%%%%%%%%%%%%%%%%%%%%%%%%%%%%%%%%%%%%%%%%%%%%%%%%%%%%%%%%%%%%%%%%%%%%%%%%%%%%%%%%%%%%%%%%%%%%%
%%%%%%%%%%%%%%%%%%%%%%%%%%%%%%%%%%%%%%%%%          LamK0             %%%%%%%%%%%%%%%%%%%%%%%%%%%%%%%%%%%%%%%%%%%%%%%%%%%%%%%%%%%
%%%%%%%%%%%%%%%%%%%%%%%%%%%%%%%%%%%%%%%%%%%%%%%%%%%%%%%%%%%%%%%%%%%%%%%%%%%%%%%%%%%%%%%%%%%%%%%%%%%%%%%%%%%%%%%%%%%%%%%%%%%%%%%%%
%%%%%%%%%%%%%%%%%%%% LamK0, polynomial background, single lambda parameter
\begin{table}[htbp]
 \centering
 \renewcommand{\arraystretch}{1.25}
 \resizebox{\paperwidth}{!}{
 \begin{tabular}{|c|c|c|c|c|c|c|}
  \multicolumn{7}{c}{Fit Results \LamALamKs} \\
  \hline
  \multirow{2}{*}{System} & \multirow{2}{*}{Centrality} & \multicolumn{5}{c|}{Fit Parameters} \\
  \cline{3-7}
   & & $\lambda$ & $R$ & $\mathbb{R}f_{0}$ & $\mathbb{I}f_{0}$ & $d_{0}$ \\
  \hline  
  \multirow{3}{*}{\LamKs \& \ALamKs}  
     & 0-10\%
     & \multirow{3}{*}{\ArrLamKs(1) $\pm$ \ArrLamKs(2) (stat.) $\pm$ \ArrLamKs(3) (sys.)}    %Lambda (LamK0 & ALamK0 AllCent)
     & \ArrLamKs(4) $\pm$ \ArrLamKs(5) (stat.) $\pm$ \ArrLamKs(6) (sys.)                     %Radius (LamK0 & ALamK0 0010)
     & \multirow{3}{*}{\ArrLamKs(19) $\pm$ \ArrLamKs(20) (stat.) $\pm$ \ArrLamKs(21) (sys.)}   %Ref0   (LamK0 & ALamK0)
     & \multirow{3}{*}{\ArrLamKs(22) $\pm$ \ArrLamKs(23) (stat.) $\pm$ \ArrLamKs(24) (sys.)}    %Imf0   (LamK0 & ALamK0)
     & \multirow{3}{*}{\ArrLamKs(25) $\pm$ \ArrLamKs(26) (stat.) $\pm$ \ArrLamKs(27) (sys.)} \\ %d0     (LamK0 & ALamK0)
   
     & 10-30\%
     & & \ArrLamKs(10) $\pm$ \ArrLamKs(11) (stat.) $\pm$ \ArrLamKs(12) (sys.)                   %Radius (LamK0 & ALamK0 1030)
     & & & \\
   
     & 30-50\%
     & & \ArrLamKs(16) $\pm$ \ArrLamKs(17) (stat.) $\pm$ \ArrLamKs(18) (sys.)                   %Radius (LamK0 & ALamK0 3050)
     & & & \\
   \hline
 \end{tabular}}
 \caption[Fit Results \LamALamKs, with 3 residual correlations included]{Fit Results \LamALamKs, with 3 residual correlations included. 
 Each pair is fit simultaneously with its conjugate (ie. \LamKs with \ALamKs) across all centralities (0-10\%, 10-30\%, 30-50\%), for a total of 6 simultaneous analyses in the fit.
 A single $\lambda$ parameter is shared amongst all.
 Each analysis has a unique normalization parameter.
 The radii are shared between analyses of like centrality, as these should have similar source sizes.
 The scattering parameters ($\mathbb{R}f_{0}$, $\mathbb{I}f_{0}$, $d_{0}$) are shared amongst all.
 The background is modeled by a (6$^{\mathrm{th}}$-)degree polynomial fit to THERMINATOR simulation.
 The fit is done on the data with only statistical error bars.
 The errors marked as ``stat." are those returned by MINUIT.
 The errors marked as ``sys." are those which result from my systematic analysis (as outlined in Section \ref{SystematicErrors}).}
 \label{tab:FitResultsLamK0_3Res}
\end{table}  

%%%%%%%%%%%%%%%%%%%%%%%%%%%%%%%%%%%%%%%%%%%%%%%%%%%%%%%%%%%%%%%%%%%%%%%%%%%%%%%%%%%%%%%%%%%%%%%%%%%%%%%%%%%%%%%%%%%%%%%%%%%%%%%%%
%%%%%%%%%%%%%%%%%%%%%%%%%%%%%%%%%%%%%%%%%          LamKch             %%%%%%%%%%%%%%%%%%%%%%%%%%%%%%%%%%%%%%%%%%%%%%%%%%%%%%%%%%%
%%%%%%%%%%%%%%%%%%%%%%%%%%%%%%%%%%%%%%%%%%%%%%%%%%%%%%%%%%%%%%%%%%%%%%%%%%%%%%%%%%%%%%%%%%%%%%%%%%%%%%%%%%%%%%%%%%%%%%%%%%%%%%%%%
%%%%%%%%%%%%%%%%%%%% LamKch, polynomial background, unique radii and unique lambda
\begin{comment}
\clearpage
\begin{table}[htbp]
 \centering
 \renewcommand{\arraystretch}{1.25}
 \resizebox{\paperwidth}{!}{
 \begin{tabular}{|c|c|c|c|c|c|c|c|}
  \multicolumn{8}{c}{Fit Results \LamALamKpm} \\
  \hline
  \multirow{2}{*}{System} & \multirow{2}{*}{Centrality} & \multirow{2}{*}{Pair Type} & \multicolumn{5}{c|}{Fit Parameters} \\
  \cline{4-8}
   & & & $\lambda$ & $R$ & $\mathbb{R}f_{0}$ & $\mathbb{I}f_{0}$ & $d_{0}$ \\
  \hline  
  \multirow{6}{*}{\LamKchP \& \ALamKchM}  
   & \multirow{2}{*}{0-10\%} 
     & \LamKchP
     & \ArrLamKchP(1) $\pm$ \ArrLamKchP(2) (stat.) $\pm$ \ArrLamKchP(3) (sys.)                     %Lambda (LamKchP 0010)
     & \multirow{2}{*}{\ArrLamKchP(4) $\pm$ \ArrLamKchP(5) (stat.) $\pm$ \ArrLamKchP(6) (sys.)}    %Radius (LamKchP & ALamKchM 0010)
     & \multirow{6}{*}{\ArrLamKchP(19) $\pm$ \ArrLamKchP(20) (stat.) $\pm$ \ArrLamKchP(21) (sys.)}   %Ref0   (LamKchP & ALamKchM)
     & \multirow{6}{*}{\ArrLamKchP(22) $\pm$ \ArrLamKchP(23) (stat.) $\pm$ \ArrLamKchP(24) (sys.)}    %Imf0   (LamKchP & ALamKchM)
     & \multirow{6}{*}{\ArrLamKchP(25) $\pm$ \ArrLamKchP(26) (stat.) $\pm$ \ArrLamKchP(27) (sys.)} \\ %d0     (LamKchP & ALamKchM)
     
     & & \ALamKchM 
     & \ArrALamKchM(1) $\pm$ \ArrALamKchM(2) (stat.) $\pm$ \ArrALamKchM(3) (sys.)                     %Lambda (ALamKchM 0010)
     & & & & \\          
   \cline{2-5}
   
   & \multirow{2}{*}{10-30\%}
     & \LamKchP
     & \ArrLamKchP(7) $\pm$ \ArrLamKchP(8) (stat.) $\pm$ \ArrLamKchP(9) (sys.)                     %Lambda (LamKchP 1030)
     & \multirow{2}{*}{\ArrLamKchP(10) $\pm$ \ArrLamKchP(11) (stat.) $\pm$ \ArrLamKchP(12) (sys.)}    %Radius (LamKchP & ALamKchM 1030) 
     & & & \\
             
     & & \ALamKchM 
     & \ArrALamKchM(7) $\pm$ \ArrALamKchM(8) (stat.) $\pm$ \ArrALamKchM(9) (sys.)                     %Lambda (ALamKchM 1030)
     & & & & \\  
   \cline{2-5}
   
   & \multirow{2}{*}{30-50\%}
     & \LamKchP
     & \ArrLamKchP(13) $\pm$ \ArrLamKchP(14) (stat.) $\pm$ \ArrLamKchP(15) (sys.)                     %Lambda (LamKchP 3050)
     & \multirow{2}{*}{\ArrLamKchP(16) $\pm$ \ArrLamKchP(17) (stat.) $\pm$ \ArrLamKchP(18) (sys.)}    %Radius (LamKchP & ALamKchM 3050)
     & & & \\
             
     & & \ALamKchM
     & \ArrALamKchM(13) $\pm$ \ArrALamKchM(14) (stat.) $\pm$ \ArrALamKchM(15) (sys.)                     %Lambda (ALamKchM 3050)
     & & & & \\  
   \hline
   \hline
  \multirow{6}{*}{\LamKchM \& \ALamKchP}  
   & \multirow{2}{*}{0-10\%} 
     & \LamKchM
     & \ArrLamKchM(1) $\pm$ \ArrLamKchM(2) (stat.) $\pm$ \ArrLamKchM(3) (sys.)                      %Lambda (LamKchM 0010)
     & \multirow{2}{*}{\ArrLamKchM(4) $\pm$ \ArrLamKchM(5) (stat.) $\pm$ \ArrLamKchM(6) (sys.)}     %Radius (LamKchM & ALamKchP 0010)
     & \multirow{6}{*}{\ArrLamKchM(19) $\pm$ \ArrLamKchM(20) (stat.) $\pm$ \ArrLamKchM(21) (sys.)}     %Ref0   (LamKchM & ALamKchP)
     & \multirow{6}{*}{\ArrLamKchM(22) $\pm$ \ArrLamKchM(23) (stat.) $\pm$ \ArrLamKchM(24) (sys.)}     %Imf0   (LamKchM & ALamKchP)
     & \multirow{6}{*}{\ArrLamKchM(25) $\pm$ \ArrLamKchM(26) (stat.) $\pm$ \ArrLamKchM(27) (sys.)} \\ %d0     (LamKchM & ALamKchP)
     
     & & \ALamKchP 
     & \ArrALamKchP(1) $\pm$ \ArrALamKchP(2) (stat.) $\pm$ \ArrALamKchP(3) (sys.)                      %Lambda (ALamKchP 0010)
     & & & & \\          
   \cline{2-5}
   
   & \multirow{2}{*}{10-30\%}
     & \LamKchM
     & \ArrLamKchM(7) $\pm$ \ArrLamKchM(8) (stat.) $\pm$ \ArrLamKchM(9) (sys.)                      %Lambda (LamKchM 1030) 
     & \multirow{2}{*}{\ArrLamKchM(10) $\pm$ \ArrLamKchM(11) (stat.) $\pm$ \ArrLamKchM(12) (sys.)}     %Radius (LamKchM & ALamKchP 1030)
     & & & \\
             
     & & \ALamKchP 
     & \ArrALamKchP(7) $\pm$ \ArrALamKchP(8) (stat.) $\pm$ \ArrALamKchP(9) (sys.)                      %Lambda (ALamKchP 1030)
     & & & & \\  
   \cline{2-5}
   
   & \multirow{2}{*}{30-50\%}
     & \LamKchM
     & \ArrLamKchM(13) $\pm$ \ArrLamKchM(14) (stat.) $\pm$ \ArrLamKchM(15) (sys.)                      %Lambda (LamKchM 3050) 
     & \multirow{2}{*}{\ArrLamKchM(16) $\pm$ \ArrLamKchM(17) (stat.) $\pm$ \ArrLamKchM(18) (sys.)}     %Radius (LamKchM & ALamKchP 3050)
     & & & \\
             
     & & \ALamKchP 
     & \ArrALamKchP(13) $\pm$ \ArrALamKchP(14) (stat.) $\pm$ \ArrALamKchP(15) (sys.)                      %Lambda (ALamKchP 3050)
     & & & & \\     
   \hline
 \end{tabular}}
 \caption[Fit Results \LamALamKpm, with 3 residual correlations included]{Fit Results \LamALamKpm, with 3 residual correlations included.
 Each pair is fit simultaneously with its conjugate (ie. \LamKchP with \ALamKchM and \LamKchM with \ALamKchP) across all centralities (0-10\%, 10-30\%, 30-50\%), for a total of 6 simultaneous analyses in the fit.
 Each analysis has a unique $\lambda$ and normalization parameter.
 The radii are shared between analyses of like centrality, as these should have similar source sizes.
 The scattering parameters ($\mathbb{R}f_{0}$, $\mathbb{I}f_{0}$, $d_{0}$) are shared amongst all.
 The background is modeled by a (6$^{\mathrm{th}}$-)degree polynomial fit to THERMINATOR simulation.
 The fit is done on the data with only statistical error bars.
 The errors marked as ``stat." are those returned by MINUIT.
 The errors marked as ``sys." are those which result from my systematic analysis (as outlined in Section \ref{SystematicErrors}).}
 \label{tab:FitResultsLamKch_3Res}
\end{table}  
\end{comment}

%%%%%%%%%%%%%%%%%%%%%%%%%%%%%%%%%%%%%%%%%%%%%%%%%%%%%%%%%%%%%%%%%%%%%%%%%%%%%%%%%%%%%%%%%%%%%%%%%%%%%%%%%%%%%%%%%%%%%%%%%%%%%%%%%
%%%%%%%%%%%%%%%%%%%% LamKch, polynomial background, unique radii and share lamconj
\begin{comment}
\clearpage
\begin{table}[htbp]
 \centering
 \renewcommand{\arraystretch}{1.25}
 \resizebox{\paperwidth}{!}{
 \begin{tabular}{|c|c|c|c|c|c|c|}
  \multicolumn{7}{c}{Fit Results \LamALamKpm} \\
  \hline
  \multirow{2}{*}{System} & \multirow{2}{*}{Centrality} & \multicolumn{5}{c|}{Fit Parameters} \\
  \cline{3-7}
   & & $\lambda$ & $R$ & $\mathbb{R}f_{0}$ & $\mathbb{I}f_{0}$ & $d_{0}$ \\
  \hline  
  \multirow{6}{*}{\LamKchP \& \ALamKchM}  
   & \multirow{2}{*}{0-10\%} 
     & \multirow{2}{*}{\ArrLamKchP(1) $\pm$ \ArrLamKchP(2) (stat.) $\pm$ \ArrLamKchP(3) (sys.)}    %Lambda (LamKchP 0010)
     & \multirow{2}{*}{\ArrLamKchP(4) $\pm$ \ArrLamKchP(5) (stat.) $\pm$ \ArrLamKchP(6) (sys.)}    %Radius (LamKchP & ALamKchM 0010)
     & \multirow{6}{*}{\ArrLamKchP(19) $\pm$ \ArrLamKchP(20) (stat.) $\pm$ \ArrLamKchP(21) (sys.)}   %Ref0   (LamKchP & ALamKchM)
     & \multirow{6}{*}{\ArrLamKchP(22) $\pm$ \ArrLamKchP(23) (stat.) $\pm$ \ArrLamKchP(24) (sys.)}    %Imf0   (LamKchP & ALamKchM)
     & \multirow{6}{*}{\ArrLamKchP(25) $\pm$ \ArrLamKchP(26) (stat.) $\pm$ \ArrLamKchP(27) (sys.)} \\ %d0     (LamKchP & ALamKchM)
     
     & & & & & & \\          
   \cline{2-4}
   
   & \multirow{2}{*}{10-30\%}
     & \multirow{2}{*}{\ArrLamKchP(7) $\pm$ \ArrLamKchP(8) (stat.) $\pm$ \ArrLamKchP(9) (sys.)}       %Lambda (LamKchP 1030)
     & \multirow{2}{*}{\ArrLamKchP(10) $\pm$ \ArrLamKchP(11) (stat.) $\pm$ \ArrLamKchP(12) (sys.)}    %Radius (LamKchP & ALamKchM 1030) 
     & & & \\
             
     & & & & & & \\  
   \cline{2-4}
   
   & \multirow{2}{*}{30-50\%}
     & \multirow{2}{*}{\ArrLamKchP(13) $\pm$ \ArrLamKchP(14) (stat.) $\pm$ \ArrLamKchP(15) (sys.)}    %Lambda (LamKchP 3050)
     & \multirow{2}{*}{\ArrLamKchP(16) $\pm$ \ArrLamKchP(17) (stat.) $\pm$ \ArrLamKchP(18) (sys.)}    %Radius (LamKchP & ALamKchM 3050)
     & & & \\
             
     & & & & & & \\  
   \hline
   \hline
  \multirow{6}{*}{\LamKchM \& \ALamKchP}  
   & \multirow{2}{*}{0-10\%} 
     & \multirow{2}{*}{\ArrLamKchM(1) $\pm$ \ArrLamKchM(2) (stat.) $\pm$ \ArrLamKchM(3) (sys.)}     %Lambda (LamKchM 0010)
     & \multirow{2}{*}{\ArrLamKchM(4) $\pm$ \ArrLamKchM(5) (stat.) $\pm$ \ArrLamKchM(6) (sys.)}     %Radius (LamKchM & ALamKchP 0010)
     & \multirow{6}{*}{\ArrLamKchM(19) $\pm$ \ArrLamKchM(20) (stat.) $\pm$ \ArrLamKchM(21) (sys.)}     %Ref0   (LamKchM & ALamKchP)
     & \multirow{6}{*}{\ArrLamKchM(22) $\pm$ \ArrLamKchM(23) (stat.) $\pm$ \ArrLamKchM(24) (sys.)}     %Imf0   (LamKchM & ALamKchP)
     & \multirow{6}{*}{\ArrLamKchM(25) $\pm$ \ArrLamKchM(26) (stat.) $\pm$ \ArrLamKchM(27) (sys.)} \\ %d0     (LamKchM & ALamKchP)
     
     & & & & & & \\          
   \cline{2-4}
   
   & \multirow{2}{*}{10-30\%}
     & \multirow{2}{*}{\ArrLamKchM(7) $\pm$ \ArrLamKchM(8) (stat.) $\pm$ \ArrLamKchM(9) (sys.)}        %Lambda (LamKchM 1030) 
     & \multirow{2}{*}{\ArrLamKchM(10) $\pm$ \ArrLamKchM(11) (stat.) $\pm$ \ArrLamKchM(12) (sys.)}     %Radius (LamKchM & ALamKchP 1030)
     & & & \\
             
     & & & & & & \\  
   \cline{2-4}
   
   & \multirow{2}{*}{30-50\%}
     & \multirow{2}{*}{\ArrLamKchM(13) $\pm$ \ArrLamKchM(14) (stat.) $\pm$ \ArrLamKchM(15) (sys.)}     %Lambda (LamKchM 3050) 
     & \multirow{2}{*}{\ArrLamKchM(16) $\pm$ \ArrLamKchM(17) (stat.) $\pm$ \ArrLamKchM(18) (sys.)}     %Radius (LamKchM & ALamKchP 3050)
     & & & \\
             
     & & & & & & \\     
   \hline
 \end{tabular}}
 \caption[Fit Results \LamALamKpm, with 3 residual correlations included]{Fit Results \LamALamKpm, with 3 residual correlations included.
 Each pair is fit simultaneously with its conjugate (ie. \LamKchP with \ALamKchM and \LamKchM with \ALamKchP) across all centralities (0-10\%, 10-30\%, 30-50\%), for a total of 6 simultaneous analyses in the fit.
 A $\lambda$ parameter is shared between a pair and its conjugate for each centrality.
 Each analysis has a unique normalization parameter.
 The radii are shared between analyses of like centrality, as these should have similar source sizes.
 The scattering parameters ($\mathbb{R}f_{0}$, $\mathbb{I}f_{0}$, $d_{0}$) are shared amongst all.
 The background is modeled by a (6$^{\mathrm{th}}$-)degree polynomial fit to THERMINATOR simulation.
 The fit is done on the data with only statistical error bars.
 The errors marked as ``stat." are those returned by MINUIT.
 The errors marked as ``sys." are those which result from my systematic analysis (as outlined in Section \ref{SystematicErrors}).}
 \label{tab:FitResultsLamKch_3Res}
\end{table}  
\end{comment}


%%%%%%%%%%%%%%%%%%%%%%%%%%%%%%%%%%%%%%%%%%%%%%%%%%%%%%%%%%%%%%%%%%%%%%%%%%%%%%%%%%%%%%%%%%%%%%%%%%%%%%%%%%%%%%%%%%%%%%%%%%%%%%%%%
%%%%%%%%%%%%%%%%%%%% LamKch, polynomial background, share radii and single lambda
%\begin{comment}
\begin{table}[htbp]
 \centering
 \renewcommand{\arraystretch}{1.25}
 \resizebox{\paperwidth}{!}{
 \begin{tabular}{|c|c|c|c|c|c|c|}
  \multicolumn{7}{c}{Fit Results \LamALamKpm} \\
  \hline
  \multirow{2}{*}{System} & \multirow{2}{*}{Centrality} & \multicolumn{5}{c|}{Fit Parameters} \\
  \cline{3-7}
   & & $\lambda$ & $R$ & $\mathbb{R}f_{0}$ & $\mathbb{I}f_{0}$ & $d_{0}$ \\
  \hline  
  \multirow{3}{*}{\LamKchP \& \ALamKchM}  
   & \multirow{2}{*}{0-10\%} 
     & \multirow{2}{*}{\ArrLamKchP(1) $\pm$ \ArrLamKchP(2) (stat.) $\pm$ \ArrLamKchP(3) (sys.)}    %Lambda (LamKchP 0010)
     & \multirow{2}{*}{\ArrLamKchP(4) $\pm$ \ArrLamKchP(5) (stat.) $\pm$ \ArrLamKchP(6) (sys.)}    %Radius (LamKchP & ALamKchM 0010)
     & \multirow{3}{*}{\ArrLamKchP(19) $\pm$ \ArrLamKchP(20) (stat.) $\pm$ \ArrLamKchP(21) (sys.)}   %Ref0   (LamKchP & ALamKchM)
     & \multirow{3}{*}{\ArrLamKchP(22) $\pm$ \ArrLamKchP(23) (stat.) $\pm$ \ArrLamKchP(24) (sys.)}    %Imf0   (LamKchP & ALamKchM)
     & \multirow{3}{*}{\ArrLamKchP(25) $\pm$ \ArrLamKchP(26) (stat.) $\pm$ \ArrLamKchP(27) (sys.)} \\ %d0     (LamKchP & ALamKchM)
     
     & & & & & & \\          
   \cline{2-4}
   
   & \multirow{2}{*}{10-30\%}
     & \multirow{2}{*}{\ArrLamKchP(7) $\pm$ \ArrLamKchP(8) (stat.) $\pm$ \ArrLamKchP(9) (sys.)}       %Lambda (LamKchP 1030)
     & \multirow{2}{*}{\ArrLamKchP(10) $\pm$ \ArrLamKchP(11) (stat.) $\pm$ \ArrLamKchP(12) (sys.)}    %Radius (LamKchP & ALamKchM 1030) 
     & & & \\
   \clineB{1-1}{4.0} 
   \clineB{5-7}{4.0}            
  \multirow{3}{*}{\LamKchP \& \ALamKchM} 
   & & & 
     & \multirow{3}{*}{\ArrLamKchM(19) $\pm$ \ArrLamKchM(20) (stat.) $\pm$ \ArrLamKchM(21) (sys.)}     %Ref0   (LamKchM & ALamKchP)
     & \multirow{3}{*}{\ArrLamKchM(22) $\pm$ \ArrLamKchM(23) (stat.) $\pm$ \ArrLamKchM(24) (sys.)}     %Imf0   (LamKchM & ALamKchP)
     & \multirow{3}{*}{\ArrLamKchM(25) $\pm$ \ArrLamKchM(26) (stat.) $\pm$ \ArrLamKchM(27) (sys.)} \\ %d0     (LamKchM & ALamKchP)
   \cline{2-4}
   
   & \multirow{2}{*}{30-50\%}
     & \multirow{2}{*}{\ArrLamKchP(13) $\pm$ \ArrLamKchP(14) (stat.) $\pm$ \ArrLamKchP(15) (sys.)}    %Lambda (LamKchP 3050)
     & \multirow{2}{*}{\ArrLamKchP(16) $\pm$ \ArrLamKchP(17) (stat.) $\pm$ \ArrLamKchP(18) (sys.)}    %Radius (LamKchP & ALamKchM 3050)
     & & & \\
             
     & & & & & & \\  
   \hline
 \end{tabular}}
 \caption[Fit Results \LamALamKpm, with 3 residual correlations included]{Fit Results \LamALamKpm, with 3 residual correlations included.
 All \LamKpm analyses are fit simultaneously across all centralities (0-10\%, 10-30\%, 30-50\%).
 Scattering parameters ($\mathbb{R}f_{0}$, $\mathbb{I}f_{0}$, $d_{0}$) are shared between pair-conjugate systems (i.e. a parameter set describing the \LamKchP \& \ALamKchM system, and a separate set describing the \LamKchM \& \ALamKchP system).
 For each centrality, a radius and $\lambda$ parameters are shared between all pairs (\LamKchP, \ALamKchM, \LamKchM, \ALamKchP).
 Each analysis has a unique normalization parameter.
 The background is modeled by a (6$^{\mathrm{th}}$-)degree polynomial fit to THERMINATOR simulation.
 The fit is done on the data with only statistical error bars.
 The errors marked as ``stat." are those returned by MINUIT.
 The errors marked as ``sys." are those which result from my systematic analysis (as outlined in Section \ref{SystematicErrors}).}
 \label{tab:FitResultsLamKch_3Res}
\end{table}
%\end{comment}

  
%%%%%%%%%%%%%%%%%%%%%%%%%%%%%%%%%%%%%%%%%%%%%%%%%%%%%%%%%%%%%%%%%%%%%%%%%%%%%%%%%%%%%%%%%%%%%%%%%%%%%%%%%%%%%%%%%%%%%%%%%%%%%%%%%
%%%%%%%%%%%%%%%%%%%% LamKch, polynomial background, share radii, share lamconj
\begin{comment}
%\clearpage
\begin{table}[htbp]
 \centering
 \renewcommand{\arraystretch}{1.25}
 \resizebox{\paperwidth}{!}{
 \begin{tabular}{|c|c|c|c|c|c|c|}
  \multicolumn{7}{c}{Fit Results \LamALamKpm} \\
  \hline
  \multirow{2}{*}{System} & \multirow{2}{*}{Centrality} & \multicolumn{5}{c|}{Fit Parameters} \\
  \cline{3-7}
   & & $\lambda$ & $R$ & $\mathbb{R}f_{0}$ & $\mathbb{I}f_{0}$ & $d_{0}$ \\
  \hline  
  \multirow{6}{*}{\LamKchP \& \ALamKchM}  
   & \multirow{2}{*}{0-10\%} 
     & \multirow{2}{*}{\ArrLamKchP(1) $\pm$ \ArrLamKchP(2) (stat.) $\pm$ \ArrLamKchP(3) (sys.)}    %Lambda (LamKchP 0010)
     & \multirow{2}{*}{\ArrLamKchP(4) $\pm$ \ArrLamKchP(5) (stat.) $\pm$ \ArrLamKchP(6) (sys.)}    %Radius (LamKchP & ALamKchM 0010)
     & \multirow{6}{*}{\ArrLamKchP(19) $\pm$ \ArrLamKchP(20) (stat.) $\pm$ \ArrLamKchP(21) (sys.)}   %Ref0   (LamKchP & ALamKchM)
     & \multirow{6}{*}{\ArrLamKchP(22) $\pm$ \ArrLamKchP(23) (stat.) $\pm$ \ArrLamKchP(24) (sys.)}    %Imf0   (LamKchP & ALamKchM)
     & \multirow{6}{*}{\ArrLamKchP(25) $\pm$ \ArrLamKchP(26) (stat.) $\pm$ \ArrLamKchP(27) (sys.)} \\ %d0     (LamKchP & ALamKchM)
     
     & & & & & & \\          
   \cline{2-4}
   
   & \multirow{2}{*}{10-30\%}
     & \multirow{2}{*}{\ArrLamKchP(7) $\pm$ \ArrLamKchP(8) (stat.) $\pm$ \ArrLamKchP(9) (sys.)}       %Lambda (LamKchP 1030)
     & \multirow{2}{*}{\ArrLamKchP(10) $\pm$ \ArrLamKchP(11) (stat.) $\pm$ \ArrLamKchP(12) (sys.)}    %Radius (LamKchP & ALamKchM 1030) 
     & & & \\
             
     & & & & & & \\  
   \cline{2-4}
   
   & \multirow{2}{*}{30-50\%}

     & \multirow{2}{*}{\ArrLamKchP(13) $\pm$ \ArrLamKchP(14) (stat.) $\pm$ \ArrLamKchP(15) (sys.)}    %Lambda (LamKchP 3050)
     & \multirow{2}{*}{\ArrLamKchP(16) $\pm$ \ArrLamKchP(17) (stat.) $\pm$ \ArrLamKchP(18) (sys.)}    %Radius (LamKchP & ALamKchM 3050)
     & & & \\
             
     & & & & & & \\  
   \hline
   \hline
  \multirow{6}{*}{\LamKchM \& \ALamKchP}  
   & \multirow{2}{*}{0-10\%} 
     & \multirow{2}{*}{\ArrLamKchM(1) $\pm$ \ArrLamKchM(2) (stat.) $\pm$ \ArrLamKchM(3) (sys.)}     %Lambda (LamKchM 0010)
     & \multirow{2}{*}{\ArrLamKchM(4) $\pm$ \ArrLamKchM(5) (stat.) $\pm$ \ArrLamKchM(6) (sys.)}     %Radius (LamKchM & ALamKchP 0010)
     & \multirow{6}{*}{\ArrLamKchM(19) $\pm$ \ArrLamKchM(20) (stat.) $\pm$ \ArrLamKchM(21) (sys.)}     %Ref0   (LamKchM & ALamKchP)
     & \multirow{6}{*}{\ArrLamKchM(22) $\pm$ \ArrLamKchM(23) (stat.) $\pm$ \ArrLamKchM(24) (sys.)}     %Imf0   (LamKchM & ALamKchP)
     & \multirow{6}{*}{\ArrLamKchM(25) $\pm$ \ArrLamKchM(26) (stat.) $\pm$ \ArrLamKchM(27) (sys.)} \\ %d0     (LamKchM & ALamKchP)
     
     & & & & & & \\          
   \cline{2-4}
   
   & \multirow{2}{*}{10-30\%}
     & \multirow{2}{*}{\ArrLamKchM(7) $\pm$ \ArrLamKchM(8) (stat.) $\pm$ \ArrLamKchM(9) (sys.)}        %Lambda (LamKchM 1030) 
     & \multirow{2}{*}{\ArrLamKchM(10) $\pm$ \ArrLamKchM(11) (stat.) $\pm$ \ArrLamKchM(12) (sys.)}     %Radius (LamKchM & ALamKchP 1030)
     & & & \\
             
     & & & & & & \\  
   \cline{2-4}
   
   & \multirow{2}{*}{30-50\%}
     & \multirow{2}{*}{\ArrLamKchM(13) $\pm$ \ArrLamKchM(14) (stat.) $\pm$ \ArrLamKchM(15) (sys.)}     %Lambda (LamKchM 3050) 
     & \multirow{2}{*}{\ArrLamKchM(16) $\pm$ \ArrLamKchM(17) (stat.) $\pm$ \ArrLamKchM(18) (sys.)}     %Radius (LamKchM & ALamKchP 3050)
     & & & \\
             
     & & & & & & \\     
   \hline
 \end{tabular}}
 \caption[Fit Results \LamALamKpm, with 3 residual correlations included]{Fit Results \LamALamKpm, with 3 residual correlations included.
 All \LamKpm analyses are fit simultaneously across all centralities (0-10\%, 10-30\%, 30-50\%).
 Scattering parameters ($\mathbb{R}f_{0}$, $\mathbb{I}f_{0}$, $d_{0}$) are shared between pair-conjugate systems (i.e. a parameter set describing the \LamKchP \& \ALamKchM system, and a separate set describing the \LamKchM \& \ALamKchP system).
 For each centrality, a radius parameter is shared between all pairs (\LamKchP, \ALamKchM, \LamKchM, \ALamKchP), and a $\lambda$ parameter is shared between a pair and its conjugate.
 Each analysis has a unique normalization parameter.
 The background is modeled by a (6$^{\mathrm{th}}$-)degree polynomial fit to THERMINATOR simulation.
 The fit is done on the data with only statistical error bars.
 The errors marked as ``stat." are those returned by MINUIT.
 The errors marked as ``sys." are those which result from my systematic analysis (as outlined in Section \ref{SystematicErrors}).}
 \label{tab:FitResultsLamKch_3Res}
\end{table}  
\end{comment}
%%%%%%%%%%%%%%%%%%%%%%%%%%%%%%%%%%%%%%%%%%%%%%%%%%%%%%%%%%%%%%%%%%%%%%%%%%%%%%%%%%%%%%%%%%%%%%%%%%%%%%%%%%%%%%%%%%%%%%%%%%%%%%%%%

%\end{landscape}
%\pagestyle{plain}


\end{document}
