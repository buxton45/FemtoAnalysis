%%% NOTE: If I want to run this standalone, uncomment out lines below
%%%       But lines must be commented out to run within larger project
\begin{comment}
\makeatletter
\def\input@path{{/home/jesse/Analysis/FemtoAnalysis/AnalysisNotes/}}
\makeatother

\documentclass[ALICE,manyauthors]{ALICE_analysis_notes}

\usepackage{MyStyle}
\usepackage{chngpage}  % for adjustwidth
\usepackage{boldline}  % to make lines in table bold
                       % V{<factor>} vertical rule in \begin{tabular} command
                       % also \clinB{<spec>}{<factor>} and \hlineB{<factor>}
\usepackage{arrayjobx} % To use the array structures stored in FitResults_cLamcKch_20180505.tex   
\end{comment}  

\newarray\ArrLamKchP
\readarray{ArrLamKchP}{
                       1.12 & 0.32 & 0.25 & 
                       6.33 & 0.99 & 0.31 & 
                       0.79 & 0.19 & 0.23 & 
                       4.77 & 0.61 & 0.17 & 
                       0.70 & 0.18 & 0.30 & 
                       3.47 & 0.46 & 0.10 & 
                       -0.66 & 0.14 & 0.13 & 
                       0.58 & 0.15 & 0.11 & 
                       0.77 & 0.47 & 1.66}

\newarray\ArrALamKchM
\readarray{ArrALamKchM}{
                       1.12 & 0.32 & 0.25 & 
                       6.33 & 0.99 & 0.31 & 
                       0.79 & 0.19 & 0.23 & 
                       4.77 & 0.61 & 0.17 & 
                       0.70 & 0.18 & 0.30 & 
                       3.47 & 0.46 & 0.10 & 
                       -0.66 & 0.14 & 0.13 & 
                       0.58 & 0.15 & 0.11 & 
                       0.77 & 0.47 & 1.66}

\newarray\ArrLamKchM
\readarray{ArrLamKchM}{
                       1.12 & 0.32 & 0.25 & 
                       6.33 & 0.99 & 0.31 & 
                       0.79 & 0.19 & 0.23 & 
                       4.77 & 0.61 & 0.17 & 
                       0.70 & 0.18 & 0.30 & 
                       3.47 & 0.46 & 0.10 & 
                       0.35 & 0.12 & 0.07 & 
                       0.44 & 0.10 & 0.08 & 
                       -4.46 & 1.53 & 1.36}

\newarray\ArrALamKchP
\readarray{ArrALamKchP}{
                       1.12 & 0.32 & 0.25 & 
                       6.33 & 0.99 & 0.31 & 
                       0.79 & 0.19 & 0.23 & 
                       4.77 & 0.61 & 0.17 & 
                       0.70 & 0.18 & 0.30 & 
                       3.47 & 0.46 & 0.10 & 
                       0.35 & 0.12 & 0.07 & 
                       0.44 & 0.10 & 0.08 & 
                       -4.46 & 1.53 & 1.36}


\newarray\ArrLamKs
\readarray{ArrLamKs}{
                       0.60 & 0.87 & 0.57 & 
                       3.15 & 0.68 & 0.45 & 
                       0.60 & 0.87 & 0.57 & 
                       2.48 & 0.54 & 0.35 & 
                       0.60 & 0.87 & 0.57 & 
                       1.80 & 0.38 & 0.17 & 
                       -0.18 & 0.04 & 0.25 & 
                       0.20 & 0.11 & 0.13 & 
                       2.72 & 1.07 & 2.12}

\newarray\ArrALamKs
\readarray{ArrALamKs}{
                       0.60 & 0.87 & 0.57 & 
                       3.15 & 0.68 & 0.45 & 
                       0.60 & 0.87 & 0.57 & 
                       2.48 & 0.54 & 0.35 & 
                       0.60 & 0.87 & 0.57 & 
                       1.80 & 0.38 & 0.17 & 
                       -0.18 & 0.04 & 0.25 & 
                       0.20 & 0.11 & 0.13 & 
                       2.72 & 1.07 & 2.12}




\begin{document}
\pagestyle{empty}

%\pagestyle{empty}
%\begin{landscape}


%%%%%%%%%%%%%%%%%%%%%%%%%%%%%%%%%%%%%%%%%%%%%%%%%%%%%%%%%%%%%%%%%%%%%%%%%%%%%%%%%%%%%%%%%%%%%%%%%%%%%%%%%%%%%%%%%%%%%%%%%%%%%%%%%
%%%%%%%%%%%%%%%%%%%%%%%%%%%%%%%%%%%%%%%%%          LamK0             %%%%%%%%%%%%%%%%%%%%%%%%%%%%%%%%%%%%%%%%%%%%%%%%%%%%%%%%%%%
%%%%%%%%%%%%%%%%%%%%%%%%%%%%%%%%%%%%%%%%%%%%%%%%%%%%%%%%%%%%%%%%%%%%%%%%%%%%%%%%%%%%%%%%%%%%%%%%%%%%%%%%%%%%%%%%%%%%%%%%%%%%%%%%%
%%%%%%%%%%%%%%%%%%%% LamK0, polynomial background, single lambda parameter
\begin{table}[htbp]
 \centering
 \renewcommand{\arraystretch}{1.25}
 \resizebox{\paperwidth}{!}{
 \begin{tabular}{|c|c|c|c|c|c|c|}
  \multicolumn{7}{c}{Fit Results \LamALamKs} \\
  \hline
  \multirow{2}{*}{System} & \multirow{2}{*}{Centrality} & \multicolumn{5}{c|}{Fit Parameters} \\
  \cline{3-7}
   & & $\lambda$ & $R$ & $\mathbb{R}f_{0}$ & $\mathbb{I}f_{0}$ & $d_{0}$ \\
  \hline  
  \multirow{3}{*}{\LamKs \& \ALamKs}  
     & 0-10\%
     & \multirow{3}{*}{\ArrLamKs(1) $\pm$ \ArrLamKs(2) (stat.) $\pm$ \ArrLamKs(3) (sys.)}    %Lambda (LamK0 & ALamK0 AllCent)
     & \ArrLamKs(4) $\pm$ \ArrLamKs(5) (stat.) $\pm$ \ArrLamKs(6) (sys.)                     %Radius (LamK0 & ALamK0 0010)
     & \multirow{3}{*}{\ArrLamKs(19) $\pm$ \ArrLamKs(20) (stat.) $\pm$ \ArrLamKs(21) (sys.)}   %Ref0   (LamK0 & ALamK0)
     & \multirow{3}{*}{\ArrLamKs(22) $\pm$ \ArrLamKs(23) (stat.) $\pm$ \ArrLamKs(24) (sys.)}    %Imf0   (LamK0 & ALamK0)
     & \multirow{3}{*}{\ArrLamKs(25) $\pm$ \ArrLamKs(26) (stat.) $\pm$ \ArrLamKs(27) (sys.)} \\ %d0     (LamK0 & ALamK0)
   
     & 10-30\%
     & & \ArrLamKs(10) $\pm$ \ArrLamKs(11) (stat.) $\pm$ \ArrLamKs(12) (sys.)                   %Radius (LamK0 & ALamK0 1030)
     & & & \\
   
     & 30-50\%
     & & \ArrLamKs(16) $\pm$ \ArrLamKs(17) (stat.) $\pm$ \ArrLamKs(18) (sys.)                   %Radius (LamK0 & ALamK0 3050)
     & & & \\
   \hline
 \end{tabular}}
 \caption[Fit Results \LamALamKs, with 3 residual correlations included]{Fit Results \LamALamKs, with 3 residual correlations included. 
 Each pair is fit simultaneously with its conjugate (ie. \LamKs with \ALamKs) across all centralities (0-10\%, 10-30\%, 30-50\%), for a total of 6 simultaneous analyses in the fit.
 A single $\lambda$ parameter is shared amongst all.
 Each analysis has a unique normalization parameter.
 The radii are shared between analyses of like centrality, as these should have similar source sizes.
 The scattering parameters ($\mathbb{R}f_{0}$, $\mathbb{I}f_{0}$, $d_{0}$) are shared amongst all.
 The background is modeled by a (6$^{\mathrm{th}}$-)degree polynomial fit to THERMINATOR simulation.
 The fit is done on the data with only statistical error bars.
 The errors marked as ``stat." are those returned by MINUIT.
 The errors marked as ``sys." are those which result from my systematic analysis (as outlined in Section \ref{SystematicErrors}).}
 \label{tab:FitResultsLamK0_3Res}
\end{table}  



%\end{landscape}
%\pagestyle{plain}


\end{document}
